\section{コンピュータチップス}\label{s1:コンピュータチップス} %{
	\subsection{RSAとDSA}\label{s2:RSAとDSA} %{
		RSAとDSAはどちらも公開鍵暗号方式であるが、次のような違いがある。
		\begin{description}\setlength{\itemsep}{-1mm} %{
			\item[RSA] ユーザー認証と通信の暗号化をサポートする。
			RSAは特許の問題で普及が遅れたが、特許の期限が切れたために、
			現在は、RSAを用いるのが主流になっている。
			\item[DSA] ユーザー認証だけをサポートする。
			署名の長さが$320$ビットに固定、鍵の長さが1024ビットまでに制限
			されている。
		\end{description} %}
	%s2:RSAとDSA}
%s1:コンピュータチップス}
\section{英語チップス}\label{s1:英語チップス} %{
	覚えておくべき単語や熟語などを書いておく。
	\begin{description}\setlength{\itemsep}{-1mm} %{
		\item[friend of ten years] 十年来の友人 \\
		日本語の場合と同じ言い回しをする。
		\item[postponement] 延期 \\
		日本語では、延期には肯定的、先送りや後回しには否定的な意味合いが
		含まれるが、英語ではその区別はないようだ。
		\item[coined by] 造語 \\
		a term conined by Sylvester. シルベスターによって作られた造語 \\
		分野によってはnamedよりもよく使われるようだ。
		\item[dabbed] あだ名をつける。 \\
		Bill is dabbed 'tiny', because he is so big.
		\item[realm] 領域 \\
		the realm of nature. 自然界 \\
		the realm of England. イングランド王国
		\item[a reputation for] という評価 \\
		Stack machines have a reputation for being slow.
		\item[incarnation] 変化
		\item[fall into] 分類される \\
		types fall into several broad categories.
		\item[a slap on the wrist] お灸を据える
		\item[disjoint union] 集合の直和 \\
		非交差という言葉も使われるようだが厳しい感じがする。
		日本語の方が難しい例である。
		\item[tableau] 表 \\
		table+auと覚える。日常英語としては絵画という意味もあるらしい。
		日本語ではタブローという言うらしい。
		\item[arrive on] Skype arrives on Microsoft phones.
		\item[palindrome] 回文 \\
		英語も日本語も普段使わない単語である。右から読んでも左から読んでも
		同じ単語を回文という。漢字を文字とした時の”山本山”などが回文になる。
		\item[put out a call for bids] 入札にする
		\item[roll out] 展開する \\
		UK Police Roll Out On-the-Spot Mobile Data Extraction System.
		\item[send someone packing] 追い払う、解雇する \\
		When her husband came home late, she sent hiim packing. \\
		彼女は遅く帰ってきた夫を叩き出しました。
		\item[goosebumps] 鳥肌
		\item[stumble out of the gate] いきなりつまづく \\
		stumbleはよろけるという意味で、stumble out of the gateで門から出てところ
		でよろけるという意味になる。
		\item[be supposed to] its supposed to be spring. \\
		もう春なののに。
		\item[herbicide] 枯葉剤 \\
		ベトナム戦争で用いられた枯葉剤は数種類あり、代表的なものにオレンジ剤
		(Agent Orange)というものがある。
		\item[relevant] 関連する。
		The appearance of an object relevant to widely diffent fields.
		\item[stunning] 見事な
		\item[feasible] 可能であれば \\
		Because the classification of representations is no longer feasible.
		\item[glimpse] 垣間見る \\
		We will just give a glimpse of this area.
		\item[tame] 飼いならす \\
		\item[up to] 二つの意味がある \\
		what are you up to? $=$  what up? \\
		it's up to you.
	\end{description} %}

	業界やインターネットで使われる英語の省略形を書いておく。
	\begin{description}\setlength{\itemsep}{-1mm} %{
		\item[WOLOG] Without Loss Of Generality
		\item[LOL] Lot Of Laugh
		\item[IMHO] In My Humble Opinion
		\item[IMO] In My Opinion
		\item[CoC] Convention over Configuration
		\item[WTF] what the fuck
	\end{description} %}
%s1:英語チップス}
