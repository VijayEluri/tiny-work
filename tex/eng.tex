\section{コンピュータチップス}\label{s1:コンピュータチップス} %{
	\subsection{RSAとDSA}\label{s2:RSAとDSA} %{
		RSAとDSAはどちらも公開鍵暗号方式であるが、次のような違いがある。
		\begin{description}\setlength{\itemsep}{-1mm} %{
			\item[RSA] ユーザー認証と通信の暗号化をサポートする。
			RSAは特許の問題で普及が遅れたが、特許の期限が切れたために、
			現在は、RSAを用いるのが主流になっている。
			\item[DSA] ユーザー認証だけをサポートする。
			署名の長さが$320$ビットに固定、鍵の長さが1024ビットまでに制限
			されている。
		\end{description} %}
	%s2:RSAとDSA}
%s1:コンピュータチップス}
\section{英語チップス}\label{s1:英語チップス} %{
	覚えておくべき熟語などを書いておく。
	\begin{description}\setlength{\itemsep}{-1mm} %{
		\item[friend of ten years] 十年来の友人 \\
		日本語の場合と同じ言い回しをする。
		\item[postponement] 延期 \\
		日本語では、延期には肯定的、先送りや後回しには否定的な意味合いが
		含まれるが、英語ではその区別はないようだ。
		\item[coined by] 造語 \\
		a term conined by Sylvester. シルベスターによって作られた造語
		\item[dabbed] あだ名をつける。 \\
		Bill is dabbed 'tiny', because he is so big.
		\item[realm] 領域
		\begin{itemize}\setlength{\itemsep}{-1mm} %{
			\item the realm of nature. 自然界
			\item the realm of England. イングランド王国
		\end{itemize} %}
		\item[a reputation for] という評価
		\begin{itemize}\setlength{\itemsep}{-1mm} %{
			\item Stack machines have a reputation for being slow.
		\end{itemize} %}
		\item[incarnation] 変化
	\end{description} %}

	業界やインターネットで使われる英語の省略形を書いておく。
	\begin{description}\setlength{\itemsep}{-1mm} %{
		\item[WOLOG] Without Loss Of Generality
		\item[LOL] Lot Of Laugh
		\item[IMHO] In My Humble Opinion
		\item[IMO] In My Opinion
	\end{description} %}

	訳に迷った英語や英語と日本語が$1:1$に対応しないものを書いておく。
	\begin{description}\setlength{\itemsep}{-1mm} %{
		\item[disjoint union] 集合の直和 \\
		非交差という言葉も使われるようだが厳しい感じがする。
	\end{description} %}
%s1:英語チップス}
