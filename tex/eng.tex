\section{コンピュータチップス}\label{s1:コンピュータチップス} %{
	\subsection{RSAとDSA}\label{s2:RSAとDSA} %{
		RSAとDSAはどちらも公開鍵暗号方式であるが、次のような違いがある。
		\begin{description}\setlength{\itemsep}{-1mm} %{
			\item[RSA] ユーザー認証と通信の暗号化をサポートする。
			RSAは特許の問題で普及が遅れたが、特許の期限が切れたために、
			現在は、RSAを用いるのが主流になっている。
			\item[DSA] ユーザー認証だけをサポートする。
			署名の長さが$320$ビットに固定、鍵の長さが1024ビットまでに制限
			されている。
		\end{description} %}
	%s2:RSAとDSA}
%s1:コンピュータチップス}
\section{英語チップス}\label{s1:英語チップス} %{
	業界やインターネットで使われる英語の省略形を書いておく。
	\begin{table}[htbp] %{
		\begin{center}\begin{tabular}{rll} \hline
			省略形 & 英語 & 日本語 \\ \hline
			WOLOG & \underline{w}ithout \underline{l}oss \underline{o}f
				\underline{g}enerality \\
			LOL & \underline{l}ot \underline{o}f \underline{l}augh \\
			IMHO & \underline{i}n \underline{m}y \underline{h}umble
				\underline{o}pinion & 個人的な意見では \\
		\end{tabular}\end{center}
		\caption{省略形}
	\end{table} %}

	訳に迷った英語や英語と日本語が$1:1$に対応しないものを書いておく。
	\begin{table}[htbp] %{
		\begin{center}\begin{tabular}{lll} \hline
			英語 & 日本語 & メモ \\ \hline
			disjoint union & 集合の直和、非公和 & 集合の圏での余直積になる \\
		\end{tabular}\end{center}
		\caption{業界用語}
	\end{table} %}
%s1:英語チップス}
