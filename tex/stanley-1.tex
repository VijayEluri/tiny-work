\begingroup %{

{\setlength\arraycolsep{2pt}
%
\section{記号など}\label{s1:記号など} %{
	\begin{table}[H] %{
		\begin{center}\begin{tabular}{l|ll} \hline
			記号 & 説明 & 代数 \\ \hline
			$K\blr{x}$ & $K$上の多項式 & 環 \\
			$K\bblr{x}$ & $K$上の形式級数 & 環 \\
			$K\plr{x}$ & $K$上の有理多項式 & 体 \\
			& $K\plr{x} := \set{\frac{f}{g}\mid f,g\neq0\in K\blr{x}}$ \\
			$K\pplr{x}$ & $K$上のLaurent級数 & 体 \\
			& $K\pplr{x} := \set{\frac{f}{g}\mid f,g\neq0\in K\bblr{x}}$ \\
			$K\plr{x}^*$ & $K$上のPuiseux多項式 & 少なくとも環 \\
			& $K\plr{x}^*:=\cup_{n\in\sizen_+}K\plr{x^{\frac{1}{n}}}$ \\
			$K\pplr{x}^*$ & $K$上のPuiseux級数 & 少なくとも環 \\
			& $K\pplr{x}^*:=\cup_{n\in\sizen_+}K\pplr{x^{\frac{1}{n}}}$ \\
		\end{tabular}\end{center}
		\caption{多項式を表す記号}
	\end{table} %}

	\begin{itemize}\setlength{\itemsep}{-1mm} %{
		\item $K\bblr{x}$はほぼ体になっている。\\
		定数項が$0$でない形式級数は逆元が定まり、定数項が$0$となる形式級数のみが
		$K\bblr{x}$内に逆元を持たない。
	\end{itemize} %}

	\begin{todo}[示すべきこと]\label{todo:示すべきこと} %{
		示すべきことを書いておく。
		\begin{itemize}\setlength{\itemsep}{-1mm} %{
			\item $K\pplr{x}=K\bblr{x}[x^{-1}]$となることを示す。
			\item 形式級数は体になるか。
			\item Puseux多項式は体になるか。
		\end{itemize} %}
	\end{todo} %todo:示すべきこと}
%s1:記号など}
\section{体の基礎事項}\label{s1:体の基礎事項} %{
	\cite{artin1959}\,p.58\,を参考にする。
\subsection{有限標数の体}\label{s2:有限標数の体} %{
	\begin{observation}[素数の剰余類]\label{obs:素数の剰余類} %{
		$p\in\sizen_+$を素数として、$\sei_p:=\sei/p\sei$を考える。
		$\sei_p^\times:=\sei_p-\set{0}$とする。
		\begin{enumerate}\setlength{\itemsep}{-1mm} %{
			\item $\sei_p^\times$は群になり、
			\item 任意の$m\in\sei_p^\times$に対して$m^p=m$が成り立つから、
		\end{enumerate} %}
		任意の$m\in\sei_p$に対して$m^p=m$が成り立つ。
	\end{observation} %obs:素数の剰余類}
	\begin{proof} %{
		\begin{enumerate}\setlength{\itemsep}{-1mm} %{
			\item $\sei_p^\times$が有限集合だから、任意の$m\in\sei_p^\times$に
			対して写像$m-:\sei_p\to\sei_p:= n\mapsto mn$が$1:1$になることを示せば
			よい。次の式は$m-$が$1:1$になることを示す。
			\begin{equation*}\begin{split}
				r\le s\in\sei_p^\times\mid mr = ms &\iff m(s-r) = 0 \\
					&\implies m(s-r) \text{は$p$の倍数} \implies s=r
			\end{split}\end{equation*}
			\item $\sei_p^\times$は有限群だから、任意の$m\in\sei_p^\times$に対して
			ある$k\in\sizen_+$があって、$m^k=1$となる。
			$k=\op{argmin}_{r\in\sizen_+}m^r=1$とすると、Lagrangeの定理から、
			$k$は$p$の約数となる。$m=1$の時は、$k=1$となり、$m\neq1$の時は、
			$k=p$となる。
		\end{enumerate} %}
	\end{proof} %}
	
	$K$を標数$p>0$の体とし、$1_K$を$K$の乗法の単位元とする。
	$K$の部分集合$\sizen1_K$を次のように定義すると、
	\begin{equation*}\begin{split}
		\sizen1_K:=\set{0,1_K,21_K,\dots,(p-1)1_K}
	\end{split}\end{equation*}
	$\sizen1_K$は$K$の部分体となる。そして、$K$の任意の部分体は$\sizen1_K$を
	部分体に持つから、$\sizen1_K$は$K$の最小の部分体となる。
	また、写像$\rho:\sei_p\to\sizen1_K:=m\mapsto m1_K$は$1:1$になり、
	加法と乗法についてそれぞれ準同型性が成り立つから、
	体同型$\rho:\sei_p\simeq\sizen1_K$が成り立つ。

	任意の$a,b\in K$に対して次の式が成り立ち、
	\begin{equation*}\begin{split}
		(a + b)^p = a^p + b^p \quad\text{because }
		\binom{p}{k}\in p\sizen  \quad\text{for all } k=1,\dots,p-1
	\end{split}\end{equation*}
	$(ab)^p=a^pb^p$だから、写像$\mu:K\to K:=a\mapsto a^p$は体準同型となる。
	そして、$p$が素数、特に奇数、だから、$(a-b)^p=a^p-b^p$となり、
	$\mu$は$1:1$となることがわかる。
%s2:有限標数の体}
\subsection{標数0の体}\label{s2:標数0の体} %{
	$K$を標数$p>0$の体とし、$1_K$を$K$の乗法の単位元とする。
	$K$の部分集合$\bun1_K$を次のように定義すると、
	\begin{equation*}\begin{split}
		\bun1_K:=\set{q1_K\mid q\in\bun}
	\end{split}\end{equation*}
	$\bun1_K$は$K$の部分体となる。そして、$K$の任意の部分体は$\bun1_K$を
	部分体に持つから、$\bun1_K$は$K$の最小の部分体となる。
	また、写像$\rho:\bun\to\bun1_K:=q\mapsto q1_K$は$1:1$になり、
	加法と乗法についてそれぞれ準同型性が成り立つから、
	体同型$\rho:\bun\simeq\bun1_K$が成り立つ。
%s2:標数0の体}
%s1:体の基礎事項}
\section{多項式の重根}\label{s1:多項式の重根} %{
	$K$を体とする。

	観察\ref{obs:素数の剰余類}からわかるように、$K$の標数が$0$でない場合は、
	係数がすべてゼロの多項式$f=0\in K\blr{x}$の他にも、
	任意の$K$の元を根に持つ多項式が存在する。
	紛らわしいので多項式に特別の名前を付けておく。

	\begin{description}\setlength{\itemsep}{-1mm} %{
		\item[ゼロ多項式] すべての係数が$0$の多項式 \\
		$f=0$と書く。
		\item[定数多項式] 次数が$0$の多項式 \\
		ゼロ多項式$\implies$定数多項式
		\item[ゼロ関数多項式] 係数体のすべての元が根になる多項式 \\
		ゼロ多項式$\implies$零関数多項式
	\end{description} %}

	ゼロ関数多項式は、多項式を$K$の自己写像としてみた時、$0$への定数写像全体
	の作る部分代数となる。
	ゼロ多項式でないがゼロ関数多項式になる多項式は、可約な多項式となる。

	零関数多項式は、標数が有限な有限体に特有な現象になっている。
	標数が$0$の場合は、体の大きさに関わらず、ゼロ関数多項式$=$ゼロ多項式となる。
	一般に、多項式の根の数は多項式の次数以下となる。
	したがって、標数に関わらず無限体の場合は、
	多項式は無限個の根を持ち得えないから、零関数多項式$=$ゼロ多項式となり、
	標数が有限な有限体の場合には、ゼロでない零関数多項式があったとしても、
	その次数は体の大きさ以上になる\cite{cox:2000}。

	\begin{todo}[証明すべきこと]\label{todo:証明すべきこと} %{
		この文言には、幾つか証明すべきことが残されている。
		\begin{itemize}\setlength{\itemsep}{-1mm} %{
			\item 標数が$0$の時、ゼロ関数多項式$=$ゼロ多項式となる。
			\item 多項式の根の数は多項式の次数以下となる。
		\end{itemize} %}
	\end{todo} %todo:証明すべきこと}

	多項式の重根について次の命題が成り立つ。

	\begin{proposition}[多項式の重根]\label{prop:多項式の重根} %{
		$f\in K\blr{x}$に対して次の性質は同値である。
		\begin{enumerate}\setlength{\itemsep}{-1mm} %{
			\item\label{item:mult-1} $f$は重根を持つ。
			\item\label{item:mult-2} $f$と$\partial f$が同一の根を持つ。
			\item\label{item:mult-3} $f$と$\partial f$が$K\blr{x}$に次数が
			$1$以上の共通因数を持つ。
		\end{enumerate} %}
	\end{proposition} %prop:多項式の重根}
	\begin{proof} %{
		$a\in E\supseteq K\mid fa=0$とする。
		\begin{itemize}\setlength{\itemsep}{-1mm} %{
			\item \ref{item:mult-1}$\iff$\ref{item:mult-2} \\
			ある$g\in E\blr{x}$があって、$fx=(x-a)(gx)$と書くことができる。
			すると、$(\der f)x=(gx)+(x-a)\plrg{(\der g)x}$となって、
			$a$が$f$の重根$\iff$$ga=0\iff(\der f)a=0$となる。
			%
			\item \ref{item:mult-2}$\iff$\ref{item:mult-3} \\
			$fa=(\der f)a=0$ならば、$a$を根に持つ$K\blr{x}$の既約多項式は、
			$f$と$\der f$の共通因数となる。逆に、$f$と$\der f$が共通因数を持てば、
			共通因数の根は$f$と$\der f$の根になる。
		\end{itemize} %}
	\end{proof} %}

	$K$の標数を$p\le2$をとすると、多項式$x^p\in K\blr{x}$は$0$を重根として
	持ち、その微分は、$px^{p-1}=0$から、ゼロ多項式となる。次の命題は、
	このような状況を考慮している。

	\begin{proposition}[多項式の重根その二]\label{prop:多項式の重根その二} %{
		既約多項式$f\in K\blr{x}$が重根を持たない。
		$\iff$ $\der f$がゼロ多項式でない。
	\end{proposition} %prop:多項式の重根その二}
	\begin{proof} %{
		$\der f=0$かそうでないかで場合分けする。
		\begin{itemize}\setlength{\itemsep}{-1mm} %{
			\item $\der f$がゼロ多項式でない。$\implies$$f$が重根を持たない。 \\
			$\der f$がゼロ多項式でなければ、$\der f$の次数は$f$の次数より小さい。
			したがって、$\der f$と$f$の共通因数は定数になる。
			よって、命題\ref{prop:多項式の重根}から、$f$は重根を持たない。
			%
			\item $\der f$がゼロ多項式である。$\implies$$f$が重根を持つ。 \\
			$\der f$がゼロ多項式ならば、$f$の任意の根が$\der f$の根になり、
			命題\ref{prop:多項式の重根}から、$f$は重根を持つ。
		\end{itemize} %}
	\end{proof} %}

	\begin{proposition}[多項式の重根その三]\label{prop:多項式の重根その三} %{
		$K$の標数が$0$の時は、$K\blr{x}$の既約多項式は重根を持たない。
	\end{proposition} %prop:多項式の重根その三}
	\begin{proof} %{
		この場合は、微分がゼロ多項式になるのは定数多項式のみになる。
	\end{proof} %}
%s1:多項式の重根}
\section{Stanley}\label{s1:Stanley} %{
	\cite{stanley-2}六章の有理言語に関する部分を読む。
	形式級数の無限和と無限積を扱うための準備で\cite{stanley-1}が必要になる。

\subsection{可換な形式級数}\label{s2:可換な形式級数} %{
	$K$を体、$K\blr{x}$を$K$上の1変数の多項式環、$K\bblr{x}$を$K$上の1変数の
	形式級数環とする。不定元の文字に依らずに係数を得るために、$K\bblr{x}$の元
	について線形な写像$\dlr{\here,\here}:K\bblr{x}\times\sizen\to K$を
	次のように定義する。
	\begin{equation*}\begin{split}
		\dlr{f,n} := f_n \quad\text{where } fx = \sum_{n\in\sizen} f_nx^n
		\text{ with } f_n\in K \text{ for all } n\in\sizen
	\end{split}\end{equation*}
	$K\bblr{x}$の加法を$+$、乗法を$\cdot$で表すと、次の式が成り立つ。
	\begin{equation*}\begin{split}
		\dlr{f + g, n} &= \dlr{f, n} + \dlr{g, n} \\
		\dlr{f \cdot g, n} &= \dlr{f, n}\dlr{g, n} \\
	\end{split}
	\quad\text{for all } f,g\in K\bblr{x},\; n\in\sizen
	\end{equation*}

	$K\bblr{x}$は係数の無限和や無限積は無制限には許さない。
	例えば、$\set{\sum_{k\in\sizen}a_k}x^n$のような係数が無限和によって定義
	された項は無制限には許さない。どのような無限和や無限積を許すかをはっきり
	させるために、$K\bblr{x}$に位相と収束を定義する。係数体の位相に依らずに
	形式級数の位相を定義するために、まず、形式級数の次数というものを定義する。

	\begin{definition}[形式級数の次数]\label{def:形式級数の次数} %{
		写像$\deg_\bot:K\bblr{x}\to\sizen+\set{\infty}$を次のように定義する。
		\begin{equation*}\begin{split}
			\deg_\bot f := \begin{cases}
				\infty, &\text{ iff } f = 0 \\
				\op{argmin}_{n\in\sizen} \dlr{f,n}\neq0, &\text{ otherwise } \\
			\end{cases} \quad\text{for all } f\in K\bblr{x}
		\end{split}\end{equation*}
		つまり、$\deg_\bot f$を$\dlr{f,n}\neq0$となる最小の$n$とする。\EOP
	\end{definition} %def:形式級数の次数}

	ここで定義した$\deg_\bot$は、通常の多項式の次数とは逆の意味になっている
	ことに注意する。通常の多項式の次数は$\dlr{f,n}\neq0$となる最大の$n$である。

	$\deg_\bot$は、$n=\deg_\bot(f-g)$とした時、$f$と$g$は$0..(n-1)$次の各係数
	が一致しているということを表す。例えば、$f,g\in K\bblr{x}$として、
	$\deg_\bot(f-g)=n$なら、$f$と$g$は互いに、
	\begin{itemize}\setlength{\itemsep}{-1mm} %{
		\item $0..(n-1)$次の各係数が一致して、
		\item $n$次の係数が異なり、
		\item $n+1$次以上の係数は一致してるかどうかわからない
	\end{itemize} %}
	となる。また、$\deg_\bot$によって、$K\bblr{x}$は次数付き$K$-代数となる。

	$(f)_\bot^n\subseteq K\bblr{x}$を次のように定義すると、
	\begin{equation*}\begin{split}
		(f)_\bot^n := \set{g\in K\bblr{x}\mid n\le \deg_\bot(g-f)}
		\quad\text{for all } f\in K\bblr{x},\; n\in\sizen\cup\set{\infty}
	\end{split}\end{equation*}
	次の部分集合の降下列が成り立つ。
	\begin{equation*}\begin{split}
		K\bblr{x} = (f)_\bot^0 \supseteq (f)_\bot^1 \supseteq (f)_\bot^2
		\supseteq \cdots \supseteq (f)_\bot^\infty = \set{f}
	\end{split}\end{equation*}
	例えば、係数体を標数$2$の体$\sei_2$、$f:=1+x\in\sei_2\bblr{x}$とすると、
	次のようになる。
	\begin{equation*}\begin{split}
		(f)_\bot^0 = \sei_2\bblr{x} 
		\supset (f)_\bot^1 = \set{1 + f_1 x + \cdots}
		\supset (f)_\bot^2 = \set{1 + x + f_2x^2 + \cdots} \\
		\supset \cdots \supset (f)_\bot^\infty = \set{1 + x}
	\end{split}\end{equation*}

	係数体が複素数の場合には、$\deg_\bot$は次のように書ける。
	\begin{equation*}\begin{split}
		\deg_\bot f = \begin{cases}
			\infty, &\text{ iff } f = 0 \\
			\lim_{x\to0}\cfrac{\ln f}{\ln x}, &\text{ otherwise } \\
		\end{cases} \quad\text{for all } f\in \fukuso\bblr{x}
	\end{split}\end{equation*}
	Maslov量子化に似ている気がするが話を続ける。

	$\deg_\bot$は次の式を満たす。
	\begin{alignat*}{2}
		\deg_\bot f= \infty &\iff f = 0 
			&&\quad\text{for all } f\in K\bblr{x} \\
		\deg_\bot(f + g) &\ge \min\plrg{\deg_\bot f,\; \deg_\bot g}
			&&\quad\text{for all } f,g\in K\bblr{x} \\
		\deg_\bot(f\cdot g) &= (\deg_\bot f) + (\deg_\bot g)
			&&\quad\text{for all } f,g\in K\bblr{x} %\\
	\end{alignat*}
	係数体が複素数の場合からわかるように、対数に似た演算になる。
	そして、ほぼ距離の定義を満たす。
	\begin{description}\setlength{\itemsep}{-1mm} %{
		\item[非負] $0\le \deg_\bot(f - g)$
		\item[反射] $f = g\iff \deg_\bot(f - g) = \infty$
		\item[対称] $\deg_\bot(f - g) = \deg_\bot(g - f)$
		\item[三角] $\deg_\bot(f + g)\ge\min\plrg{\deg_\bot f,\;\deg_\bot g}$
	\end{description} %}
	距離の定義の$0$を$\infty$に置き換えて不等号の向きを反転した形になっている。
	距離$d$の場合は、反射律が$f=g\implies d(f,g)=0$となり、
	三角不等式が$d(f,g)\le d(f,h)+g(h,g)$となる。そして、三角不等式が
	$d(f,g)\le\max\plrg{d(f,h),\;d(h,g)}$となっている時は、
	$d(f,g)\le\max\plrg{d(f,h),\;d(h,g)}\le d(f,h)+d(h,g)$となるから、
	通常の三角不等式より強い不等式となる。このように$\max$で表される三角不等式
	を非アルキメデス距離というらしい\footnote{
		wikipediaの距離空間
	}。

	形式級数の次数を使って収束を定義する。

	\begin{definition}[形式級数の収束]\label{def:形式級数の収束} %{
		$\set{f_n}:=\set{f_n\in K\bblr{x}\mid n\in\sizen}$が次の性質を満たす時、
		\begin{itemize}\setlength{\itemsep}{-1mm} %{
			\item 任意の$N\in\sizen$に対してある$n\in\sizen$が存在して、
			次の式を満たす。
			\begin{equation*}\begin{split}
				N\le \deg_\bot(f_{k+1} - f_k) \quad\text{for all } n\le k\in\sizen
			\end{split}\end{equation*}
		\end{itemize} %}
		$\set{f_n}$は収束するという。
		また、ある$f\in K\bblr{x}$に対して次の性質を満たす時、
		\begin{itemize}\setlength{\itemsep}{-1mm} %{
			\item 任意の$N\in\sizen$に対してある$n\in\sizen$が存在して、
			次の式を満たす。
			\begin{equation*}\begin{split}
				N\le \deg_\bot(f - f_k) \quad\text{for all } n\le k\in\sizen
			\end{split}\end{equation*}
		\end{itemize} %}
		$\set{f_n}$は$f$に収束するとか、$f$を$\set{f_n}$を収束点という。\EOP
	\end{definition} %def:形式級数の収束}

	$\set{f_n}$が$f\in K\bblr{x}$に収束するならば、$\set{f_n}$は収束する。
	\begin{proof} %{
		任意の$N\in\sizen$に対してある$n\in\sizen$があって、
		\begin{equation*}\begin{split}
			N\le\deg_\bot(f - f_k) \quad\text{for all }n\le k\in\sizen
		\end{split}\end{equation*}
		となるが、
		\begin{equation*}\begin{split}
			\deg_\bot(f_{k+1} - f_k)
			&\ge\min\plrg{\deg_\bot(f_{k+1} - f),\;\deg_\bot(f - f_k)} \\
			&\ge\min\plrg{N,\;N} = N \quad\text{for all }n\le k\in\sizen
		\end{split}\end{equation*}
		となる。
	\end{proof} %}
	この逆、
	\begin{itemize}\setlength{\itemsep}{-1mm} %{
		\item $\set{f_n}$が収束するならば、ある$f\in K\bblr{x}$が唯一つ定まり、
		$\set{f_n}$が$f$に収束する
	\end{itemize} %}
	という文言が成り立つ時、$K\bblr{x}$は完備となる。
	$\set{f_n}$が収束するならば、$f_\infty:=\lim_{n\to\infty}f_n$とおいてしまえば、
	次の意味で$f\in K\bblr{x}$となる。
	\begin{equation*}\begin{split}
		\dlr{f_\infty,n} = \lim_{p\to\infty} \dlr{f_p,n}\in K
		\quad\text{for all } n\in\sizen
	\end{split}\end{equation*}
	また、${f_n}$が$g$に収束するならば、任意の$N\in\sizen$に対してある
	$n\in\sizen$が存在して、次の式を満たし、
	\begin{equation*}\begin{split}
		N\le\deg_\bot(f_\infty - f_k) \text{ and } N\le\deg_\bot(g - f_k)
		\quad\text{for all }n\le k\in\sizen
	\end{split}\end{equation*}
	次の式が成り立つ。
	\begin{equation*}\begin{split}
		\deg_\bot(f_\infty - g)
		&\ge\min\plrg{\deg_\bot(f_\infty - f_k),\;\deg_\bot(g - f_k)} \\
		&\ge\min\plrg{N,\;N} = N \quad\text{for all }n\le k\in\sizen
	\end{split}\end{equation*}
	したがって、$K\bblr{x}$が完備になることがわかる。

	厳密には、多項式環$K\blr{x}$を$\deg_\bot$で完備化したものを$K\bblr{x}$
	とすべきなのだろうが、\cite{stanley-1}では直感的であることを優先して、
	集合同型$K\bblr{x}\simeq\cat{Set}(\sizen,K)$によって形式級数を定義しておいて、
	後から位相を入れるという手順を踏んでいる。

	\begin{proposition}[形式級数の収束その一]\label{prop:形式級数の収束その一} %{
		(\cite{stanley-1} p.15 1.1.8)
		$\set{f_n}$を$K\bblr{x}$の加算部分集合とする。
		$\sum_{n\in\sizen}f_n$が収束することと、$\lim_{n\to\infty}f_n=0$となる
		ことは同値である。\EOP
	\end{proposition} %prop:形式級数の収束その一}
	\begin{proof} %{
		$F_n:=\sum_{k=0}^n f_k$とすると、$F_{n+1}-F_n=f_{n+1}$となるから、
		$\deg_\bot(F_{n+1}-F_n)=\deg_\bot f_{n+1}=\deg_\bot(f_{n+1}-0)$となる。
	\end{proof} %}

	\begin{proposition}[形式級数の収束その二]\label{prop:形式級数の収束その二} %{
		(\cite{stanley-1} p.15 1.1.9)
		$\set{f_n}$を$K\bblr{x}$の加算部分集合とし、各$n\in\sizen$で
		$\dlr{f_n,0}=0$となるとする。
		この時、$\prod_{n\in\sizen}(1 + f_n)$が収束することと、
		$\lim_{n\to\infty}f_n=0$となることは同値である。\EOP
	\end{proposition} %prop:形式級数の収束その二}
	\begin{proof} %{
		$F_n:=\prod_{k=0}^n(1 + f_k)$とすると、$F_{n+1}-F_n=f_{n+1}F_n$となる。
		\begin{itemize}\setlength{\itemsep}{-1mm} %{
			\item $f_\infty=0$ならば$\set{F_n}$は収束する。\\
			任意の$N\in\sizen$に対してある$n\in\sizen$が存在して、次の式を満たし、
			\begin{equation*}\begin{split}
				N\le \deg_\bot f_k \quad\text{for all } n\le k\in\sizen
			\end{split}\end{equation*}
			次の式を満たす。
			\begin{equation*}\begin{split}
				\deg_\bot(F_{k+1}-F_k) &= \deg_\bot f_k + \deg_\bot F_k \\
				&\ge N + \deg_\bot F_k\ge N \quad\text{for all } n\le k\in\sizen
			\end{split}\end{equation*}
			\item $\set{F_n}$が収束するならば$f_\infty=0$となる。\\
			任意の$N\in\sizen$に対してある$n\in\sizen$が存在して、次の式を満たすが、
			\begin{equation*}\begin{split}
				N\le \deg_\bot(F_{k+1} - F_k)
				= \deg_\bot f_{k+1} + \deg_\bot F_k
				\quad\text{for all } n\le k\in\sizen
			\end{split}\end{equation*}
			$\dlr{F_k,0}=1$だから、$\deg_\bot F_k=0$となり、次の式が成り立つ。
			\begin{equation*}\begin{split}
				N\le\deg_\bot f_{k+1} \quad\text{for all } n\le k\in\sizen
			\end{split}\end{equation*}
		\end{itemize} %}
	\end{proof} %}

	この命題から次の命題が得られる。

	\begin{proposition}[形式級数の収束その三]\label{prop:形式級数の収束その三} %{
		$f,g\in K\bblr{x}$とし、$\dlr{g,0}=0$とする。この時、合成$f\circ g$
		\begin{equation*}\begin{split}
			(f\circ g)x := \sum_{n\in\sizen} f_n\plr{gx}^n
		\end{split}\end{equation*}
		は収束する。\EOP
	\end{proposition} %prop:形式級数の収束その三}
	\begin{proof} %{
		$g0=0$から$1\le\deg_\bot g$となり、次の式が成り立つ。
		\begin{equation*}\begin{split}
			\deg_\bot g^n = n\deg_\bot g \ge n \quad\text{for all } n\in\sizen
		\end{split}\end{equation*}
		したがって、次の式が成り立ち、
		\begin{equation*}\begin{split}
			\lim_{n\to\infty} \deg_\bot\plrg{f_n\plr{gx}^n}
			&= \lim_{n\to\infty}\plrg{\deg_\bot f_n + n \deg_\bot g} \\
			&\ge \lim_{n\to\infty}\plrg{\deg_\bot f_n + n} = \infty
		\end{split}\end{equation*}
		命題\ref{prop:形式級数の収束その一}から$f\circ g$が収束することがわかる。
	\end{proof} %}

	この命題を使って、直感に反して形式級数でない例を挙げることができる。
	
	\begin{example}[収束しない形式級数]\label{eg:収束しない形式級数} %{
		$f,g\in\fukuso\bblr{x}$を次のように定義して、
		\begin{equation*}\begin{split}
			fx := \sum_{n\in\sizen}\frac{x^n}{n!}
			,\quad gx = 1 + x
		\end{split}\end{equation*}
		合成$f\circ g$を考える。$\deg_\bot g^n=0$かつ$\deg_\bot f_n=0$なので、
		$\lim_{n\to\infty}\deg_\bot\plrg{f_n(gx)}=0$となり、
		命題\ref{prop:形式級数の収束その一}から、$f\circ g$は収束しない。
		一方、$f\circ g$を通常の写像として解釈すると、
		$(f\circ g)x=e^{1+x}=e(fx)$となり、$e(fx)\in\fukuso\bblr{x}$となる。
		つまり、次の奇妙な関係になっている。
		\begin{equation*}\begin{split}
			\sum_{n\in\sizen}\frac{ex^n}{n!}\in\fukuso\bblr{x}
			\text{ but }
			\sum_{n\in\sizen}\frac{(1 + x)^n}{n!}\not\in\fukuso\bblr{x}
		\end{split}\end{equation*}
		一つ目の級数と二つ目の級数が一致することを示すためには
		$e=\sum_{n\in\sizen}(n!)^{-1}$という無限和を含む式が必要になる。
		一方、$\deg_\bot$は係数体$K$の性質に依存せずに定義されていて、
		一般には、$K$の元の無限和が$K$の元になるという保証はないので、
		形式級数の範囲内では係数の無限和を取り扱うことができない。
		このために、上記のような直感に反したことになる。\EOP
	\end{example} %eg:収束しない形式級数}
%s2:可換な形式級数}
%s1:Stanley}
%
}\endgroup %}
