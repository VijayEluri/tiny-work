\begingroup %{
{\setlength\arraycolsep{2pt}
%
\section{Stanley}\label{s1:Stanley} %{
	\cite{stanley-2}六章の有理言語に関する部分を読む。
	形式級数の無限和と無限積を扱うための準備で\cite{stanley-1}が必要になる。

\subsection{可換な形式級数}\label{s2:可換な形式級数} %{
	$K$を体、$K\blr{x}$を$K$上の1変数の多項式環、$K\bblr{x}$を$K$上の1変数の
	形式級数環とする。不定元の文字に依らずに係数を得るために、$K\bblr{x}$の元
	について線形な写像$\dlr{\here,\here}:K\bblr{x}\times\sizen\to K$を
	次のように定義する。
	\begin{equation*}\begin{split}
		\dlr{f,n} := f_n \quad\text{where } fx = \sum_{n\in\sizen} f_nx^n
		\text{ with } f_n\in K \text{ for all } n\in\sizen
	\end{split}\end{equation*}
	$K\bblr{x}$の加法を$+$、乗法を$\cdot$で表すと、次の式が成り立つ。
	\begin{equation*}\begin{split}
		\dlr{f + g, n} &= \dlr{f, n} + \dlr{g, n} \\
		\dlr{f \cdot g, n} &= \dlr{f, n}\dlr{g, n} \\
	\end{split}
	\quad\text{for all } f,g\in K\bblr{x},\; n\in\sizen
	\end{equation*}

	$K\bblr{x}$は係数の無限和や無限積は無制限には許さない。
	例えば、$\set{\sum_{k\in\sizen}a_k}x^n$のような係数が無限和によって定義
	された項は無制限には許さない。どのような無限和や無限積を許すかをはっきり
	させるために、$K\bblr{x}$に位相と収束を定義する。係数体の位相に依らずに
	形式級数の位相を定義するために、まず、形式級数の次数というものを定義する。

	\begin{definition}[形式級数の次数]\label{def:形式級数の次数} %{
		写像$\deg_\bot:K\bblr{x}\to\sizen+\set{\infty}$を次のように定義する。
		\begin{equation*}\begin{split}
			\deg_\bot f := \begin{cases}
				\infty, &\text{ iff } f = 0 \\
				\op{argmin}_{n\in\sizen} \dlr{f,n}\neq0, &\text{ otherwise } \\
			\end{cases} \quad\text{for all } f\in K\bblr{x}
		\end{split}\end{equation*}
		つまり、$\deg_\bot f$を$\dlr{f,n}\neq0$となる最小の$n$とする。\EOP
	\end{definition} %def:形式級数の次数}

	ここで定義した$\deg_\bot$は、通常の多項式の次数とは逆の意味になっている
	ことに注意する。通常の多項式の次数は$\dlr{f,n}\neq0$となる最大の$n$である。

	$\deg_\bot$は、$n=\deg_\bot(f-g)$とした時、$f$と$g$は$0..(n-1)$次の各係数
	が一致しているということを表す。また、$\deg_\bot$によって、$K\bblr{x}$は
	次数付き$K$-代数となる。

	係数体が複素数の場合には、$\deg_\bot$は次のように書ける。
	\begin{equation*}\begin{split}
		\deg_\bot f = \begin{cases}
			\infty, &\text{ iff } f = 0 \\
			\lim_{x\to0}\cfrac{\ln f}{\ln x}, &\text{ otherwise } \\
		\end{cases} \quad\text{for all } f\in \fukuso\bblr{x}
	\end{split}\end{equation*}
	Maslov量子化に似ている気がするが話を続ける。

	$\deg_\bot$は次の式を満たす。
	\begin{alignat*}{2}
		\deg_\bot f= \infty &\iff f = 0 
			&&\quad\text{for all } f\in K\bblr{x} \\
		\deg_\bot(f + g) &\ge \min\plrg{\deg_\bot f,\; \deg_\bot g}
			&&\quad\text{for all } f,g\in K\bblr{x} \\
		\deg_\bot(f\cdot g) &= (\deg_\bot f) + (\deg_\bot g)
			&&\quad\text{for all } f,g\in K\bblr{x} %\\
	\end{alignat*}
	係数体が複素数の場合からわかるように、対数に似た演算になる。
	そして、ほぼ距離の定義を満たす。
	\begin{description}\setlength{\itemsep}{-1mm} %{
		\item[非負] $0\le \deg_\bot(f - g)$
		\item[反射] $f = g\iff \deg_\bot(f - g) = \infty$
		\item[対称] $\deg_\bot(f - g) = \deg_\bot(g - f)$
		\item[三角] $\deg_\bot(f + g)\ge\min\plrg{\deg_\bot f,\;\deg_\bot g}$
	\end{description} %}
	距離の定義の$0$を$\infty$に置き換えて不等号の向きを反転した形になっている。
	距離$d$の場合は、反射律が$f=g\implies d(f,g)=0$となり、
	三角不等式が$d(f,g)\le d(f,h)+g(h,g)$となる。そして、三角不等式が
	$d(f,g)\le\max\plrg{d(f,h),\;d(h,g)}$となっている時は、
	$d(f,g)\le\max\plrg{d(f,h),\;d(h,g)}\le d(f,h)+d(h,g)$となるから、
	通常の三角不等式より強い不等式となる。このように$\max$で表される三角不等式
	を非アルキメデス距離というらしい\footnote{
		wikipediaの距離空間
	}。

	形式級数の次数を使って収束を定義する。

	\begin{definition}[形式級数の収束]\label{def:形式級数の収束} %{
		$\set{f_n}:=\set{f_n\in K\bblr{x}\mid n\in\sizen}$が次の性質を満たす時、
		\begin{itemize}\setlength{\itemsep}{-1mm} %{
			\item 任意の$N\in\sizen$に対してある$n\in\sizen$が存在して、
			次の式を満たす。
			\begin{equation*}\begin{split}
				N\le \deg_\bot(f_{k+1} - f_k) \quad\text{for all } n\le k\in\sizen
			\end{split}\end{equation*}
		\end{itemize} %}
		$\set{f_n}$は収束するという。
		また、ある$f\in K\bblr{x}$に対して次の性質を満たす時、
		\begin{itemize}\setlength{\itemsep}{-1mm} %{
			\item 任意の$N\in\sizen$に対してある$n\in\sizen$が存在して、
			次の式を満たす。
			\begin{equation*}\begin{split}
				N\le \deg_\bot(f - f_k) \quad\text{for all } n\le k\in\sizen
			\end{split}\end{equation*}
		\end{itemize} %}
		$\set{f_n}$は$f$に収束するとか、$f$を$\set{f_n}$を収束点という。\EOP
	\end{definition} %def:形式級数の収束}

	$\set{f_n}$が$f\in K\bblr{x}$に収束するならば、$\set{f_n}$は収束する。
	\begin{proof} %{
		任意の$N\in\sizen$に対してある$n\in\sizen$があって、
		\begin{equation*}\begin{split}
			N\le\deg_\bot(f - f_k) \quad\text{for all }n\le k\in\sizen
		\end{split}\end{equation*}
		となるが、
		\begin{equation*}\begin{split}
			\deg_\bot(f_{k+1} - f_k)
			&\ge\min\plrg{\deg_\bot(f_{k+1} - f),\;\deg_\bot(f - f_k)} \\
			&\ge\min\plrg{N,\;N} = N \quad\text{for all }n\le k\in\sizen
		\end{split}\end{equation*}
		となる。
	\end{proof} %}
	この逆、
	\begin{itemize}\setlength{\itemsep}{-1mm} %{
		\item $\set{f_n}$が収束するならば、ある$f\in K\bblr{x}$が唯一つ定まり、
		$\set{f_n}$が$f$に収束する
	\end{itemize} %}
	という文言が成り立つ時、$K\bblr{x}$は完備となる。
	$\set{f_n}$が収束するならば、$f_\infty:=\lim_{n\to\infty}f_n$とおいてしまえば、
	次の意味で$f\in K\bblr{x}$となる。
	\begin{equation*}\begin{split}
		\dlr{f_\infty,n} = \lim_{p\to\infty} \dlr{f_p,n}\in K
		\quad\text{for all } n\in\sizen
	\end{split}\end{equation*}
	また、${f_n}$が$g$に収束するならば、任意の$N\in\sizen$に対してある
	$n\in\sizen$が存在して、次の式を満たし、
	\begin{equation*}\begin{split}
		N\le\deg_\bot(f_\infty - f_k) \text{ and } N\le\deg_\bot(g - f_k)
		\quad\text{for all }n\le k\in\sizen
	\end{split}\end{equation*}
	次の式が成り立つ。
	\begin{equation*}\begin{split}
		\deg_\bot(f_\infty - g)
		&\ge\min\plrg{\deg_\bot(f_\infty - f_k),\;\deg_\bot(g - f_k)} \\
		&\ge\min\plrg{N,\;N} = N \quad\text{for all }n\le k\in\sizen
	\end{split}\end{equation*}
	したがって、$K\bblr{x}$が完備になることがわかる。

	厳密には、多項式環$K\blr{x}$を$\deg_\bot$で完備化したものを$K\bblr{x}$
	とすべきなのだろうが、\cite{stanley-1}では直感的であることを優先して、
	集合同型$K\bblr{x}\simeq\cat{Set}(\sizen,K)$によって形式級数を定義しておいて、
	後から位相を入れるという手順を踏んでいる。

	\begin{proposition}[形式級数の収束その一]\label{prop:形式級数の収束その一} %{
		(\cite{stanley-1} p.15 1.1.8)
		$\set{f_n}$を$K\bblr{x}$の加算部分集合とする。
		$\sum_{n\in\sizen}f_n$が収束することと、$\lim_{n\to\infty}f_n=0$となる
		ことは同値である。\EOP
	\end{proposition} %prop:形式級数の収束その一}
	\begin{proof} %{
		$F_n:=\sum_{k=0}^n f_k$とすると、$F_{n+1}-F_n=f_{n+1}$となるから、
		$\deg_\bot(F_{n+1}-F_n)=\deg_\bot f_{n+1}=\deg_\bot(f_{n+1}-0)$となる。
	\end{proof} %}

	\begin{proposition}[形式級数の収束その二]\label{prop:形式級数の収束その二} %{
		(\cite{stanley-1} p.15 1.1.9)
		$\set{f_n}$を$K\bblr{x}$の加算部分集合とし、各$n\in\sizen$で
		$\dlr{f_n,0}=0$となるとする。
		この時、$\prod_{n\in\sizen}(1 + f_n)$が収束することと、
		$\lim_{n\to\infty}f_n=0$となることは同値である。\EOP
	\end{proposition} %prop:形式級数の収束その二}
	\begin{proof} %{
		$F_n:=\prod_{k=0}^n(1 + f_k)$とすると、$F_{n+1}-F_n=f_{n+1}F_n$となる。
		\begin{itemize}\setlength{\itemsep}{-1mm} %{
			\item $f_\infty=0$ならば$\set{F_n}$は収束する。\\
			任意の$N\in\sizen$に対してある$n\in\sizen$が存在して、次の式を満たし、
			\begin{equation*}\begin{split}
				N\le \deg_\bot f_k \quad\text{for all } n\le k\in\sizen
			\end{split}\end{equation*}
			次の式を満たす。
			\begin{equation*}\begin{split}
				\deg_\bot(F_{k+1}-F_k) &= \deg_\bot f_k + \deg_\bot F_k \\
				&\ge N + \deg_\bot F_k\ge N \quad\text{for all } n\le k\in\sizen
			\end{split}\end{equation*}
			\item $\set{F_n}$が収束するならば$f_\infty=0$となる。\\
			任意の$N\in\sizen$に対してある$n\in\sizen$が存在して、次の式を満たすが、
			\begin{equation*}\begin{split}
				N\le \deg_\bot(F_{k+1} - F_k)
				= \deg_\bot f_{k+1} + \deg_\bot F_k
				\quad\text{for all } n\le k\in\sizen
			\end{split}\end{equation*}
			$\dlr{F_k,0}=1$だから、$\deg_\bot F_k=0$となり、次の式が成り立つ。
			\begin{equation*}\begin{split}
				N\le\deg_\bot f_{k+1} \quad\text{for all } n\le k\in\sizen
			\end{split}\end{equation*}
		\end{itemize} %}
	\end{proof} %}

	この命題から次の命題が得られる。

	\begin{proposition}[形式級数の収束その三]\label{prop:形式級数の収束その三} %{
		$f,g\in K\bblr{x}$とし、$\dlr{g,0}=0$とする。この時、合成$f\circ g$
		\begin{equation*}\begin{split}
			(f\circ g)x := \sum_{n\in\sizen} f_n\plr{gx}^n
		\end{split}\end{equation*}
		は収束する。\EOP
	\end{proposition} %prop:形式級数の収束その三}
	\begin{proof} %{
		$g0=0$から$1\le\deg_\bot g$となり、次の式が成り立つ。
		\begin{equation*}\begin{split}
			\deg_\bot g^n = n\deg_\bot g \ge n \quad\text{for all } n\in\sizen
		\end{split}\end{equation*}
		したがって、次の式が成り立ち、
		\begin{equation*}\begin{split}
			\lim_{n\to\infty} \deg_\bot\plrg{f_n\plr{gx}^n}
			&= \lim_{n\to\infty}\plrg{\deg_\bot f_n + n \deg_\bot g} \\
			&\ge \lim_{n\to\infty}\plrg{\deg_\bot f_n + n} = \infty
		\end{split}\end{equation*}
		命題\ref{prop:形式級数の収束その一}から$f\circ g$が収束することがわかる。
	\end{proof} %}

	この命題を使って、直感に反して形式級数でない例を挙げることができる。
	
	\begin{example}[収束しない形式級数]\label{eg:収束しない形式級数} %{
		$f,g\in\fukuso\bblr{x}$を次のように定義して、
		\begin{equation*}\begin{split}
			fx := \sum_{n\in\sizen}\frac{x^n}{n!}
			,\quad gx = 1 + x
		\end{split}\end{equation*}
		合成$f\circ g$を考える。$\deg_\bot g^n=0$かつ$\deg_\bot f_n=0$なので、
		$\lim_{n\to\infty}\deg_\bot\plrg{f_n(gx)}=0$となり、
		命題\ref{prop:形式級数の収束その一}から、$f\circ g$は収束しない。
		一方、$f\circ g$を通常の写像として解釈すると、
		$(f\circ g)x=e^{1+x}=e(fx)$となり、$e(fx)\in\fukuso\bblr{x}$となる。
		つまり、次の奇妙な関係になっている。
		\begin{equation*}\begin{split}
			\sum_{n\in\sizen}\frac{ex^n}{n!}\in\fukuso\bblr{x}
			\text{ but }
			\sum_{n\in\sizen}\frac{(1 + x)^n}{n!}\not\in\fukuso\bblr{x}
		\end{split}\end{equation*}
		一つ目の級数と二つ目の級数が一致することを示すためには
		$e=\sum_{n\in\sizen}(n!)^{-1}$という無限和を含む式が必要になる。
		一方、$\deg_\bot$は係数体$K$の性質に依存せずに定義されていて、
		一般には、$K$の元の無限和が$K$の元になるという保証はないので、
		形式級数の範囲内では係数の無限和を取り扱うことができない。
		このために、上記のような直感に反したことになる。\EOP
	\end{example} %eg:収束しない形式級数}
%s2:可換な形式級数}
%s1:Stanley}
%
}\endgroup %}
