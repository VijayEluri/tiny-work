\usepackage{amsmath,amssymb,amscd}
\usepackage[obeyspaces]{url}
\usepackage[all]{xy}
% installed in the system by hand
\usepackage{shuffle}
\usepackage{listings}
\usepackage{float}
\lstset{language=java %
%	, numbers=left %
%	, numberstyle ={\tiny \emph} %
%	, numbersep=10pt %
%	, breaklines=true %
%	, breakindent=40pt %
%	, frame=tlRB %
	, frame=leftline %
%	, frameround=ffft %
%	, framesep=3pt %
%	, rulesep=1pt %
%	, rulecolor={\color{blue}} %
%	, rulesepcolor={\color{blue}} %
	, flexiblecolumns=true %
	, keepspaces=false %
	, basicstyle=\scriptsize %
	, identifierstyle=\itshape\scriptsize %
	, commentstyle=\fontfamily{ptm}\selectfont\scriptsize %
	, stringstyle=\scshape\scriptsize %
	, tabsize=2 %
}
\renewcommand{\lstlistingname}{プログラム}
\usepackage{sty/simplewick}
\usepackage[enableskew]{sty/youngtab}
\usepackage{sty/jumoline}
\usepackage{sty/boxedminipage}
\usepackage{stmaryrd}
\usepackage{fancybox}
\usepackage{lscape}
%{
\usepackage[dvipdfm,dvips]{color}
\usepackage[dvipdfm,dvips]{graphicx}
\usepackage[dvipdfm %
  , hypertex %
  , colorlinks=true %
  , bookmarks=true %
  , bookmarksnumbered=false %
  , bookmarkstype=toc %
  , pdfkeywords={TeX; dvipdfmx; hyperref; color;} %
	]{hyperref}
\ifnum 42146=\euc"A4A2
  \AtBeginDvi{\special{pdf:tounicode EUC-UCS2}}% platex-utf8 でも OK
\else
  \AtBeginDvi{\special{pdf:tounicode 90ms-RKSJ-UCS2}}%"
\fi
\usepackage{verbatim}
\usepackage{sty/myarrow}
\usepackage{sty/mybraket}
%}
\usepackage[amsmath,hyperref]{sty/ntheorem}
\newtheorem{theorem}{定理}[section]
\newtheorem{definition}{定義}[section]
\newtheorem{proposition}{命題}[section]
\newtheorem{procedure}{手続き}[section]
\newtheorem{todo}{課題}[section]
\newtheorem{note}{ノート}[section]
\newtheorem{example}{例}[section]
\newtheorem{observation}{観察}[section]
\newtheorem{problem}{問題}[section]
%
\def\proof{\rm \trivlist \item[\hskip \labelsep{証明}] }
\def\endproof{{\large$\Box$}\endtrivlist}
%
% common
%
\newcommand{\bool}{\ensuremath{\mathbf{B}}}
\newcommand{\sizen}{\ensuremath{\mathbf{N}}}
\newcommand{\sei}{\ensuremath{\mathbf{Z}}}
\newcommand{\jitu}{\ensuremath{\mathbf{R}}}
\newcommand{\fukuso}{\ensuremath{\mathbf{C}}}
\newcommand{\bun}{\ensuremath{\mathbf{Q}}}
\newcommand{\is}[1]{\mathinner{[\![#1]\!]}}
%
\newcommand{\op}[1]{\mathinner{\operatorname{#1}}}
\newcommand{\id}{\op{id}}
\newcommand{\dfn}{\op{def}}
\newcommand{\onto}{\op{onto}}
\newcommand{\der}{{\partial}}
\newcommand{\xiff}[2][]{\xLongleftrightarrow[#1]{#2}}
\newcommand{\ximplies}[2][]{\xRightarrow[#1]{#2}}
\newcommand{\ximpliedby}[2][]{\xLeftarrow[#1]{#2}}
\newcommand{\here}{{\ensuremath{\mathchar`-}}}
\newcommand{\opp}{{\op{op}}}
%
\newcommand{\cat}[1]{{\mathbf{#1}}}
\newcommand{\obj}{{\mathfrak{O}}}
\newcommand{\arr}{{\mathfrak{A}}}
%
\newcommand{\qbinom}[2]{\genfrac{[}{]}{0pt}{0}{#1}{#2}}
%
\newcommand{\mvec}[2]{\begin{matrix}{#1}\\{#2}\end{matrix}}
\newcommand{\pvec}[2]{\begin{pmatrix}{#1}\\{#2}\end{pmatrix}}
\newcommand{\bvec}[2]{\begin{bmatrix}{#1}\\{#2}\end{bmatrix}}
\newcommand{\tr}{\op{tr}}
%
\newcommand{\what}[1]{{\widehat{#1}}}
\newcommand{\wbar}[1]{{\widebar{#1}}}
\newcommand{\wtilde}[1]{{\widetilde{#1}}}
%
\newcommand{\clA}{{\mathcal{A}}}
\newcommand{\clB}{{\mathcal{B}}}
\newcommand{\clC}{{\mathcal{C}}}
\newcommand{\clD}{{\mathcal{D}}}
\newcommand{\clF}{{\mathcal{F}}}
\newcommand{\clG}{{\mathcal{G}}}
\newcommand{\clH}{{\mathcal{H}}}
\newcommand{\clI}{{\mathcal{I}}}
\newcommand{\clJ}{{\mathcal{J}}}
\newcommand{\clK}{{\mathcal{K}}}
\newcommand{\clL}{{\mathcal{L}}}
\newcommand{\clM}{{\mathcal{M}}}
\newcommand{\clN}{{\mathcal{N}}}
\newcommand{\clO}{{\mathcal{O}}}
\newcommand{\clP}{{\mathcal{P}}}
\newcommand{\clQ}{{\mathcal{Q}}}
\newcommand{\clR}{{\mathcal{R}}}
\newcommand{\clS}{{\mathcal{S}}}
\newcommand{\clT}{{\mathcal{T}}}
\newcommand{\clU}{{\mathcal{U}}}
\newcommand{\clV}{{\mathcal{V}}}
\newcommand{\clW}{{\mathcal{W}}}
\newcommand{\clX}{{\mathcal{X}}}
\newcommand{\clY}{{\mathcal{Y}}}
\newcommand{\clZ}{{\mathcal{Z}}}
%
\newcommand{\EOP}{\hspace{\fill}\P}
%
\newcommand{\hen}{\ar@{-}}
\newcommand{\arto}{\ar@{|->}}
%
\usepackage{pifont}
%
\DeclareMathOperator*{\Res}{Res}
%
\newcommand{\citepage}[2]{{``\cite{#1}}\,{#2}''}
