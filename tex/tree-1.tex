\begingroup %{
\newcommand{\qa}[1]{{\blra{#1}}}
\newcommand{\q}[1]{{\blr{#1}}}
\newcommand{\qg}[1]{{\blrg{#1}}}
\newcommand{\qgg}[1]{{\blrgg{#1}}}
\newcommand{\qggg}[1]{{\blrggg{#1}}}
\newcommand{\qgggg}[1]{{\blrgggg{#1}}}
\newcommand{\opN}{{\op{N}}}
\newcommand{\opL}{{\op{L}}}
\newcommand{\opR}{{\op{R}}}
\newcommand{\hf}{{\what{f}}}
\newcommand{\hF}{{\what{F}}}
\newcommand{\hm}{{\what{m}}}
\newcommand{\hM}{{\what{M}}}
\newcommand{\hx}{{\what{x}}}
\newcommand{\hX}{{\what{X}}}
\newcommand{\hy}{{\what{y}}}
\newcommand{\hY}{{\what{y}}}
\newcommand{\halpha}{{\what{\alpha}}}
\newcommand{\hbeta}{{\what{\beta}}}
\newcommand{\hgamma}{{\what{\gamma}}}
\newcommand{\hmu}{{\what{\mu}}}
\newcommand{\bou}{{\,|\,}}
\newcommand{\uw}{{\uparrow}}
\newcommand{\dw}{{\downarrow}}
\newcommand{\sumw}{{\op{sum}}}
\newcommand{\word}[1]{{\lfloor{#1}\rfloor}}
\newcommand{\code}{{\op{code}}}
\newcommand{\grow}{{\op{grow}}}
\newcommand{\graft}{{\op{graft}}}
\newcommand{\cut}{{\op{cut}}}
\newcommand{\negw}{{\op{neg}}}
%\newcommand{\Deg}{{\op{Deg}}}
%\newcommand{\incr}{{\op{inc}_R}}
%\newcommand{\decl}{{\op{dec}_L}}
%\newcommand{\incr}{{R_+}}
%\newcommand{\decl}{{L_-}}
\newcommand{\toN}{{\op{to}^\sizen}}
{\setlength\arraycolsep{2pt}
%
\section{二分木の配列}\label{s1:二分木の配列} %{
	二分木と二分木を文字とする文字列は次の図のような同型対応がある。
	\begin{equation}\label{eq:木の生成関数その一}\begin{split}
		\xymatrix@R=4pt@C=4pt{
			\bullet\hen[d]\hen[r] & \cdots\hen[r] & \circ\hen[d]\hen[r] & \circ \\
			t_1 & \cdots & t_k  
		}\mapsto \word{t_1,\dots,t_k}
	\end{split}\end{equation}
	この対応を写像の言葉に書き直そう。
	$\iota:\clB_*\to\clB_*^*\mid t\mapsto\word{t}$を文字列への標準入射とし、
	写像$\cut:\clB_*\to\clB_*^*$を次のように定義する。
	\begin{equation*}\begin{split}
		\cut\,\bullet := 1,\quad \cut\beta = m_0\plr{\iota\otimes\cut}
	\end{split}\end{equation*}
	すると、次の式が成り立ち、
	\begin{equation*}\begin{split}
		\plr{\hbeta\,\cut\,\bullet}\bullet = \bullet,\quad
		\plr{\hbeta\,\cut\,\beta\plr{s\otimes t}}
		= \plr{\hbeta s}\plr{\hbeta\,\cut\,t}\bullet
		\quad\text{for all } s,t\in\clB_*
	\end{split}\end{equation*}
	次数についての帰納法によって、
	$\phi:\clB_*^*\to\clB_*\mid w\mapsto\plr{\hbeta w}\bullet$とすると、
	$\phi\,\cut=\id$となることがわかる。また、次の式から、
	\begin{equation*}\begin{split}
		\cut\,\phi1 = 1,\quad
		\cut\,\phi\plr{t\otimes w} 
		%= \cut\beta\plr{t\otimes\phi w}
		= m_0\plrg{\word{t}\otimes\cut\,\phi w}
		\quad\text{for all } t\in\clB_*,\; w\in\clB_*^*
	\end{split}\end{equation*}
	やはり、次数についての帰納法によって、$\cut\,\phi=\id$となることが
	わかる。したがって、$\cut$は同型射となり、逆射$\cut^{-1}$は次のように
	なることがわかる。
	\begin{equation*}\begin{split}
		\cut^{-1}\,w := \plr{\hbeta w}\bullet \quad\text{for all } w\in\clB_*^*
	\end{split}\end{equation*}

	$\clB_*$の積$m_0$を次の畳み込みで定義する。
	\begin{equation*}\begin{split}
		m_0 := \cut^{-1}\,m_0\plr{\cut\otimes\cut}
	\end{split}\end{equation*}
	$m_0$は次のように、一つ目の木の右端の葉を二つ目の木で置き換えるという
	操作になる。
	\begin{equation*}\begin{split}
		\xymatrix@R=4pt@C=4pt{
			\bullet\hen[d]\hen[r] & \cdots\hen[r] & \circ\hen[d]\hen[r] & \circ \\
			s_1 & \cdots & s_k  
		}\otimes\xymatrix@R=4pt@C=4pt{
			\bullet\hen[d]\hen[r] & \cdots\hen[r] & \circ\hen[d]\hen[r] & \circ \\
			t_1 & \cdots & t_l  
		}\xmapsto{m_0}\xymatrix@R=4pt@C=4pt{
			\bullet\hen[d]\hen[r] & \cdots\hen[r] & \circ\hen[d]\hen[r] 
			& \circ\hen[d]\hen[r] & \cdots\hen[r] & \circ\hen[d]\hen[r] 
			& \circ \\
			s_1 & \cdots & s_k & t_1 & \cdots & t_k
		}
	\end{split}\end{equation*}
	$m_0$は$\bullet$を単位元に持ち、次数を保存する。
	\begin{equation*}\begin{split}
		\deg m_0\plr{s\otimes t} = \plr{\deg s} + \plr{\deg t}
		\quad\text{for all } s,t\in\clB_*
	\end{split}\end{equation*}

	二分木の成長$\grow$は配列にそのまま移行される。
	線形射$\grow^\clW:R_q\clB_*^*\to R_q\clB_*^*$を次のように定義する。
	\begin{equation*}\begin{split}
		\grow^\clW := \cut\,\grow\,\cut^{-1}
	\end{split}\end{equation*}
	すると、次の交換関係が成り立つが、
	\begin{equation*}\begin{split}
		\grow^\clW\plr{\hm_0t} = q^{\deg t}\plr{\hm_0t}\grow^\clW
			+ \hm_0\,\grow\,t \quad\text{for all } t\in\clB_*
	\end{split}\end{equation*}
	この交換関係は$\grow$の交換関係と全く同じ形をしている。
	\begin{equation*}\begin{split}
		\grow\plr{\hbeta t} = q^{\deg t}\grow\plr{\hbeta t}\grow
			+ \hbeta\,\grow\,t \quad\text{for all } t\in\clB_*
	\end{split}\end{equation*}
	$\grow^\clW$は次のようになっている。
	\begin{equation*}\begin{split}
		\grow^\clW1 = \bullet
		,\quad \grow^\clW\bullet = \xymatrix@R=4pt@C=4pt{
			\bullet\hen[r]\hen[d] & \circ \\ \circ
		} + \bullet\otimes\bullet \\
		\grow^\clW\xymatrix@R=4pt@C=4pt{
			\bullet\hen[r]\hen[d] & \circ \\ \circ
		} = \xymatrix@R=4pt@C=4pt{
			\bullet\hen[r]\hen[d] & \circ \\ \circ\hen[r]\hen[d] & \circ \\ \circ
		} + \xymatrix@R=4pt@C=4pt{
			\bullet\hen[r]\hen[d] & \circ\hen[r]\hen[d] & \circ \\ \circ & \circ
		} + q\xymatrix@R=4pt@C=4pt{
			\bullet\hen[r]\hen[d] & \circ \\ \circ
		}\otimes\bullet \\
		\grow^\clW\bullet\otimes\bullet = \xymatrix@R=4pt@C=4pt{
			\bullet\hen[r]\hen[d] & \circ \\ \circ
		}\otimes\bullet + \bullet\otimes\xymatrix@R=4pt@C=4pt{
			\bullet\hen[r]\hen[d] & \circ \\ \circ
		} + \bullet\otimes\bullet\otimes\bullet
	\end{split}\end{equation*}
	一般には、Leibniz則によく似た計算になる。
	\begin{equation*}\begin{split}
		\grow^\clW\word{t_1,\dots, t_k}
		&= \word{\grow\,t_1, t_2, \dots, t_k} \\
		&\,+ q^{\deg t_1}\word{t_1, \grow\,t_2, \dots, t_k} \\
		&\,+ \cdots \\
		&\,+ q^{\deg t_1 +\cdots+ \deg t_{k-1}}\word{t_1, \dots, t_{k-1}, \grow\,t_k} \\
		&\,+ q^{\deg t_1 +\cdots+ \deg t_k}\word{t_1, \dots, t_k, \bullet} 
	\end{split}\end{equation*}
%s1:二分木の配列}
\section{二分木の符号化}\label{s1:二分木の符号化} %{
	二分木を符号化する方法はいろいろあるが、ここでは、葉に数字を当てはめる
	符号化を考える。次のように、左の葉には負、右の葉には正の数字を当てはめる
	ことを考える。
	\begin{equation*}\begin{split}
		\xymatrix@R=4pt@C=4pt{
			&& \bullet\hen[dl]\hen[dr] \\
			& \circ\hen[dl]\hen[dr] && 1 \\
			-2 && 1 \\
		}\,\quad\xymatrix@R=4pt@C=4pt{
			& \bullet\hen[dl]\hen[dr] \\
			-1 && \circ\hen[dl]\hen[dr] \\
			& -1 && 2
		}
	\end{split}\end{equation*}
	写像$\code_\sei:\clB_*\to\sei^+$を次のように定義する。
	\begin{equation*}\begin{split}
		\code_\sei\bullet := \word{0},\quad
		\code_\sei\beta = m_0\plr{L_-\,\code_\sei\otimes R_+\,\code}
	\end{split}\end{equation*}
	ここで、$\opL_\pm,\opR_\pm:\sei^+\to\sei^+$は次のように定義する。
	\begin{equation*}\begin{split}
		\opL_\pm\plr{\word{n}w} := \word{n\pm1}w,\quad
		\opR_\pm\plr{w\word{n}} := w\word{n\pm1}
		\quad\text{for all } w\in\sei^*,\; n\in\sei
	\end{split}\end{equation*}
	$\code_\sei$の像は次の性質を持つ。
	\begin{itemize}\setlength{\itemsep}{-1mm} %{
		\item 文字列の長さは次数に$1$を足したものになる。
		\begin{equation*}\begin{split}
			|\code_\sei t| = \deg t + 1 \quad\text{for all } t\in\clB_*
		\end{split}\end{equation*}
		\item 文字の和は$0$になる。
		\begin{equation*}\begin{split}
			\sumw\,\code\sei t = 0 \quad\text{for all } t\in\clB_*
		\end{split}\end{equation*}
		ここで、$\sumw:\sei^*\to\sei$は次のように定義する。
		\begin{equation*}\begin{split}
			\sumw\,1 = 0,\quad \sumw\word{n_1,\dots,n_k} = n_1 +\cdots+ n_k
			\quad\text{for all } n_1,\dots,n_k\in\sei
		\end{split}\end{equation*}
		\item 正の文字の和は次数になる。
		\begin{equation*}\begin{split}
			\sumw\,\toN\,\code_\sei t = \deg t \quad\text{for all } t\in\clB_*
		\end{split}\end{equation*}
		ここで、$\toN:\sei^*\to\sizen^*$は次のように定義する。
		\begin{equation*}\begin{split}
			\toN\,1 = 1,\quad \toN\word{n_1,\dots,n_k} 
			= \word{\max(n_1,0),\dots,\max(n_k,0)} \\
			\quad\text{for all } n_1,\dots,n_k\in\sei
		\end{split}\end{equation*}
	\end{itemize} %}

	この符号化が$1:1$であることを証明しよう。
	$\clD_n:=\code_\sei\clB_n$とする。$\clD_n\subseteq\sei^{n+1}$だから、
	$m\neq n\in\sizen$ならば、$\clD_m\cap\clD_n=\emptyset$となる。
	そして、$\code_\sei$は、負の文字は右へ、正の文字は下に絶対値分だけ移動する
	ようにすると、次のようにDyck経路と対応する。
	\begin{equation*}\begin{split}
		\xymatrix@R=4pt@C=4pt{
			&& \bullet\hen[dl]\hen[dr] \\
			& \circ\hen[dl]\hen[dr] && 1 \\
			-2 && 1 \\
		} \mapsto \xymatrix@R=2ex@C=2ex{
			\circ\ar[rr] && \circ\ar[d] \\
			&& \circ\ar[d] \\
			&& \circ \\
		},\quad \xymatrix@R=4pt@C=4pt{
			& \bullet\hen[dl]\hen[dr] \\
			-1 && \circ\hen[dl]\hen[dr] \\
			& -1 && 2
		} \mapsto \xymatrix@R=2ex@C=2ex{
			\circ\ar[r] & \circ\ar[r] & \circ \ar[dd] \\
			&& \\
			&& \circ \\
		}
	\end{split}\end{equation*}
	したがって、次の命題が成り立つ。

	\begin{proposition}[符号化のDyck経路]\label{prop:符号化のDyck経路} %{
		任意の$w\in\code_\sei\clB_+$に対して、ある$w_1,w_2\in\sei^*$があって、
		$w=w_1w_2$ならば、$\sumw\,w_1\le 0$となり、$\sumw\,w_1=0$となるのは、
		$w_1=w$または$w_1=1$となる時だけである。
	\EOP\end{proposition} %prop:符号化のDyck経路}
	\begin{proof} %{
		次数についての帰納法で証明する。次数が$1$の時は、
		$\lambda\,\code_\sei\clB_1=\word{-1,1}$だから、命題は成り立つ。
		次数が$n$以下で命題が成り立つとすると、任意の
		$w_1,w_2\in\oplus_{k=0}^n\code_\sei\clB_k$に対して、
		ある$x,y\in\sei^+$があって、$\plr{L_-w_1}\plr{R_+w_2}=xy$ならば、
		帰納法の仮定によって、$\sumw\,x\le-1$となる。
		したがって、次数が$n+1$でも命題が成り立つ。
	\end{proof} %}

	この命題を使うと、$\code_\sei$が$1:1$となることが証明できる。

	\begin{proposition}[符号化は$1:1$]\label{prop:符号化は1:1} %{
		$\code_\sei$は$1:1$になる。
	\end{proposition} %prop:符号化は1:1}
	\begin{proof} %{
		次数についての帰納法で証明する。$\lambda\,\code_\sei\clB_0=\word{0}$
		だから、次数が$0$の時は$\code_\sei$は$1:1$になる。
		ある次数$n$以下で$\code_\sei$が$1:1$になると仮定する。すると、
		任意の$s_i,t_j\in\oplus_{k=0}^n\clB_k$に対して次の式が成り立つ。
		\begin{equation*}\begin{split}
			&\code_\sei\beta\plr{s_1\otimes s_2}
			= \code_\sei\beta\plr{t_1\otimes t_2} \\
			\implies& \plr{\code_\sei s_1}\plr{\code_\sei s_2}
			= \plr{\code_\sei t_1}\plr{\code_\sei t_2}
		\end{split}\end{equation*}
		$|\code_\sei s_1|<|\code_\sei t_1|$とすると、
		$\code_\sei t_1=\plr{\code_\sei s_1}w$となる$w\in\sei^+$があって、
		$\sumw\,w=0$となるが、これは、上記の命題\ref{prop:符号化のDyck経路}に反する。
		したがって、$|\code_\sei s_1|=|\code_\sei t_1|$となり、帰納法の仮定から、
		$s_1=t_1$かつ$s_2=t_2$となる。したがって、次数が$n+1$でも命題が成り立つ。
	\end{proof} %}

	\begin{todo}[復号化]\label{todo:復号化} %{
		$\code_\sei^{-1}:R\clD_*\to R\clB_*$は次の式を満たす。
		\begin{equation*}\begin{split}
			\code_\sei^{-1}\word{0} = \bullet,\quad
			\code_\sei^{-1}\,m_0\plr{L_-\otimes R_+} 
			= \beta\plr{\code_\sei^{-1}\otimes\code_\sei^{-1}}
		\end{split}\end{equation*}
		$\code_\sei^{-1}$の定義域を$\sei^+$に拡大する。できれば、写像の形で、
		できなければ、アルゴリズムの形で書いておきたい。\EOP
	\end{todo} %todo:復号化}

	線形射$\grow_\sei:R_q\code_\sei\clB_*\to R_q\code_\sei\clB_*$を次の式を満たす
	ように定義する。
	\begin{equation*}\begin{split}
		\grow_\sei\code_\sei = \code_\sei\grow
	\end{split}\end{equation*}
	線形射$\opN_\sizen:R\sei^*\to R\sei^*$を次のように定義すると、
	\begin{equation*}\begin{split}
		\opN_\sizen w = \plr{\sumw\,\toN w} w \quad\text{for all } w\in\sei^+
	\end{split}\end{equation*}
	$\grow_\sei$は次の式を満たす。
	\begin{equation*}\begin{split}
		\grow_\sei\word{0} &= \word{-1, 1} \\
		\grow_\sei m_0\plr{L_-\otimes R_+} 
		&= m_0\plr{L_-\grow_\sei\otimes1 + q^{\opN_\sizen}\otimes R_+\grow_\sei} \\
	\end{split}\end{equation*}
	ここで、$\grow_\sei$と$L_-,R_+$が可換だと仮定すると、次のようになり、
	\begin{equation*}\begin{split}
		\grow_\sei\word{n} &= \begin{cases}
			\word{-1, n+1}, &\text{ if } 0\le n \\
			\word{n - 1, 1}, &\text{ otherwise } \\
		\end{cases} \quad\text{for all } n\in\sizen \\
		\grow_\sei m_0 
		&= m_0\plr{\grow_\sei\otimes1 + q^{\opN_\sizen}\otimes\grow_\sei}
	\end{split}\end{equation*}
	定義域を$\sei^*$に拡大することができる。したがって、改めて、
	線形射$\grow_\sei:R_q\sei^*\to R_q\sei^*$をこの式で定義する。
	この時、$\grow_\sei$は成長する操作にも関わらず、空配列への作用は$0$になる、
	$\grow_\sei\word{}=0$、ことに注意する。このことから、$\grow_\sei$は
	$R_q\sei^+$で閉じていて、実質$R_q\sei^+\to R_q\sei^+$となる。

\begin{todo}[符号の対称化]\label{todo:符号の対称化} %{
	$\clD_n$を文字の並びについて対称化$D_n$すると、次のような重複が現れる。
	\begin{equation*}\begin{split}
D_0 &= [0],\quad D_1 = [-1, 1],\quad D_2 = [-1, -1, 2] + [-2, 1, 1] \\
D_3 &= [-1, -1, -1, 3] + 3[-2, -1, 1, 2] + [-3, 1, 1, 1] \\
D_4 &= [-1, -1, -1, -1, 4] + [-4, 1, 1, 1, 1]
	+ 2[-2, -1, -1, 2, 2] + 2[-2, -2, 1, 1, 2] \\
&\,+ 4[-2, -1, -1, 1, 3] + 4[-3, -1, 1, 1, 2] \\
D_5 &= [-1, -1, -1, -1, -1, 5] + [-5, 1, 1, 1, 1, 1] \\
&\,+ 5[-2, -1, -1, -1, 1, 4] + 5[-2, -1, -1, -1, 2, 3] + 5[-2, -2, -1, 1, 1, 3] \\
&\,+ 5[-2, -2, -1, 1, 2, 2] + 5[-3, -1, -1, 1, 1, 3] + 5[-3, -1, -1, 1, 2, 2] \\
&\,+ 5[-3, -2, 1, 1, 1, 2] + 5[-4, -1, 1, 1, 1, 2] \\
	\end{split}\end{equation*}
	数の合成$\clC_{n,k}$を対称化した数の分割$\clP_{n,k}$を用いると、
	一般に$D_n$は、ある$d_{n,r}\in\sizen$を用いて、次のように書ける。
	\begin{equation*}\begin{split}
		D_n = \sum_{r=1}^n d_{n,r} \lambda\plr{\negw\,\clP_{n,n+1-r}}\clP_{n,r}
	\end{split}\end{equation*}
	$d_{n,r}$をどうやって計算すればよいだろうか?\EOP
\end{todo} %todo:符号の対称化}
%s1:二分木の符号化}
%
}\endgroup %}
