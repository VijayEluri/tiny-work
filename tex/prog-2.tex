\begingroup %{
	\newcommand{\Pow}{\mycal{P}}
	{\setlength\arraycolsep{2pt}
\section{文字列の検索}\label{s1:文字列の検索} %{
\subsection{Boyer-Mooreの方法}\label{s2:Boyer-Mooreの方法} %{
	Boyer-Mooreの方法は文字列を検索するとき、
	\begin{itemize}\setlength{\itemsep}{-1mm} %{
		\item 如何にして無駄な文字のマッチングを避けるか
	\end{itemize} %}
	という問題に対して基本的な考え方を与えてくれる。

	文章を先頭から一文字づつずらしながら単語を照合していく方法では、
	文章が$AACAAB$だった場合、単語$AAB$が出現する位置を見つけ出すまで$9$回
	の文字の照合を行うことになる。
	\begin{equation*}\begin{array}{rcrcrcrcr}
		3 &+& 2 &+& 1 &+& 3 &=& 9 \\
		\young(AACAAB,AAB) &\to& \young(AACAAB,:AA) &\to&
		\young(AACAAB,::A) &\to& \young(AACAAB,:::AAB)
	\end{array}\end{equation*}
	文字集合を$\Sigma$として、単語$\Sigma^*AAB$の状態遷移図を書くと
	次のようになる。
	\begin{equation*}\begin{split}
		\xymatrix{
			x_0 \ar@(lu,ld)_{\neg a} \ar[r]^A
			& x_1 \ar[r]^A \ar@(ul,ur)[l]_{A^\neg} 
			& x_2 \ar[r]^B \ar@(ul,ur)[l]_A \ar@(dl,dr)[ll]^{(A+B)^\neg} & 1
		} \quad \begin{split}
			x_0 &= \Sigma^*AAB \\
			x_1 &= AB + x_0 \\
			x_2 &= B + x_1 \\
		\end{split}
	\end{split}\end{equation*}
	したがって、状態遷移図を用いると$7$回の文字の照合で済む。
	\begin{equation*}\begin{array}{rcrcrcrcr}
		3 &+& 1 &+& 3 &=& 7 \\
		\young(AACAAB,AAB) &\to& C\neq A &\to& \young(AACAAB,:::AAB)
	\end{array}\end{equation*}
	さらに、単語の右から左へと文字の照合をおこなっていくと、$5$回の文字の
	照合で済む。
	\begin{equation*}\begin{array}{rcrcrcrcr}
		1 &+& 1 &+& 3 &=& 5 \\
		\young(AACAAB,AAB) &\to& C\neq A &\to& \young(AACAAB,:::AAB)
	\end{array}\end{equation*}
	この例からわかるように、文章の中から単語$w$が出現する位置を見つけ出す
	には、文章中のカーソル上の文字に対して、$w$の文字を右から左への照合
	していけば効率がよい。

	Boyer-Mooreの方法を解説\cite{lecroq}にしたがって書いておく。
	Boyer-Mooreの方法は、単語の右から左へと文字の照合を行なっていき、文字の
	ミスマッチが起こったところで、予め計算しておいた次の二つの関数を利用して
	次にとるべき動作を決める。
	\begin{description}\setlength{\itemsep}{-1mm} %{
		\item[good suffix shift] 既にマッチした部分が残る場合 \\
		文章中の文字$b$と単語中の文字$a$がミスマッチしたとする。
		\begin{itemize}\setlength{\itemsep}{-1mm} %{
			\item パターンその一
			\begin{equation*}\begin{array}{rcrcrcrcrcrcrcrcrcr}
			\text{text} &\quad& &\cdots& && \cdots && \cdots &|& B &|& \text{matched} &|& \cdots \\
			\text{word} &\quad& && &|& \cdots && \cdots &|& A &|& \text{matched} &|& \\
			\text{word} &\quad& && && \xto{\text{shift}} &|& \cdots &|& C &|& \text{matched} &|& \cdots &|& \\
			\end{array}\end{equation*}
			\item パターンその二
			\begin{equation*}\begin{array}{rcrcrcrcrcrcrcrcr}
			\text{text} &\quad& && \cdots && \cdots &|& B &|& \text{matched-1} &|& \text{matched-2} &|& \cdots \\
			\text{word} &\quad& && &|& \cdots &|& A &|& \text{matched-1} &|& \text{matched-2} &|& \\
			\text{word} &\quad& && && && && \xto{\text{shift}} &|& \text{matched-2}  &|& \cdots &|& \\
			\end{array}\end{equation*}
		\end{itemize} %}
		\item[bad character shift] 既にマッチした部分が残らない場合 \\
		\begin{itemize}\setlength{\itemsep}{-1mm} %{
			\item パターンその一
			\begin{equation*}\begin{array}{rcrcrcrcrcrcrcrcrcr}
			\text{text} &\quad& &\cdots& && \cdots && \cdots &|& B &|& \text{matched} &|& \cdots \\
			\text{word} &\quad& && &|& \cdots && \cdots &|& A &|& \text{matched} &|& \\
			\text{word} &\quad& && && \xto{\text{shift}} &|& \cdots &|& B &|& \text{no $b$s} &|& \text{no $b$s} &|& \\
			\end{array}\end{equation*}
			\item パターンその二
			\begin{equation*}\begin{array}{rcrcrcrcrcrcrcrcrcr}
			\text{text} &\quad& &\cdots& && \cdots && \cdots &|& B &|& \text{matched} &|& \cdots \\
			\text{word} &\quad& && &|& \cdots && \cdots &|& A &|& \text{matched} &|& \\
			\text{word} &\quad& && && && && \xto{\text{shift}} &|& \text{no $B$s} &|& \text{no $B$s} &|& \\
			\end{array}\end{equation*}
		\end{itemize} %}
	\end{description} %}
	\cite{lecroq}の例を用いてgood-suffixとbad-characterの表を作ってみる。
	使っている{\TeX}のライブラリの都合上、$a:=A..Z-\set{A}$のように、
	大文字の補集合として小文字を用いる。単語$GCAGAGAG$のgood-suffixの表
	$\myop{bmG}$は次のようにして作る。
	\begin{equation*}\begin{array}{rcl}
		\text{シフトの計算} && \myop{bmG} \\
		\young(GCAGAGAG,::::::g) &\implies& \begin{array}{r|rrrrrrrr}
			\text{index} & 0 & 1 & 2 & 3 & 4 & 5 & 6 & 7 \\\hline
			\text{shift} &   &   &   &   &   &   &   & 1 \\
		\end{array} \\
		\young(:GCAGAGAG,aG) &\implies& \begin{array}{r|rrrrrrrr}
			\text{index} & 0 & 1 & 2 & 3 & 4 & 5 & 6 & 7 \\\hline
			\text{shift} &   &   &   &   &   &   & 7 & 1 \\
		\end{array} \\
		\young(GCAGAGAG,:gAG) &\implies& \begin{array}{r|rrrrrrrr}
			\text{index} & 0 & 1 & 2 & 3 & 4 & 5 & 6 & 7 \\\hline
			\text{shift} &   &   &   &   &   & 4 & 7 & 1 \\
		\end{array} \\
		\young(:::GCAGAGAG,aGAG) &\implies& \begin{array}{r|rrrrrrrr}
			\text{index} & 0 & 1 & 2 & 3 & 4 & 5 & 6 & 7 \\\hline
			\text{shift} &   &   &   &   & 7 & 4 & 7 & 1 \\
		\end{array} \\
		\young(GCAGAGAG,:gAGAG) &\implies& \begin{array}{r|rrrrrrrr}
			\text{index} & 0 & 1 & 2 & 3 & 4 & 5 & 6 & 7 \\\hline
			\text{shift} &   &   &   & 2 & 7 & 4 & 7 & 1 \\
		\end{array} \\
		\young(:::::GCAGAGAG,aGAGAG) &\implies& \begin{array}{r|rrrrrrrr}
			\text{index} & 0 & 1 & 2 & 3 & 4 & 5 & 6 & 7 \\\hline
			\text{shift} &   &   & 7 & 2 & 7 & 4 & 7 & 1 \\
		\end{array} \\
		\young(::::::GCAGAGAG,cAGAGAG) &\implies& \begin{array}{r|rrrrrrrr}
			\text{index} & 0 & 1 & 2 & 3 & 4 & 5 & 6 & 7 \\\hline
			\text{shift} &   & 7 & 7 & 2 & 7 & 4 & 7 & 1 \\
		\end{array} \\
		\young(:::::::GCAGAGAG,gCAGAGAG) &\implies& \begin{array}{r|rrrrrrrr}
			\text{index} & 0 & 1 & 2 & 3 & 4 & 5 & 6 & 7 \\\hline
			\text{shift} & 7 & 7 & 7 & 2 & 7 & 4 & 7 & 1 \\
		\end{array} \\
	\end{array}\end{equation*}
	単語$w$のgood-suffixの値は$1$以上$|w|$以下の自然数となる。
	単語$GCAGAGAG$の場合は、単語の長さは$8$だが、左端と右端の文字が等しい
	から、$1$以上$7$以下の自然数のみが現れていることに注意する。このように
	して作られた$\myop{bmG}$は次のようになる。
	\begin{equation*}\begin{split}
		\myop{bmG} := \begin{array}{r|rrrrrrrr}
			\text{alph.} & G & C & A & G & A & G & A & G \\
			\text{index} & 0 & 1 & 2 & 3 & 4 & 5 & 6 & 7 \\\hline
			\text{shift} & 7 & 7 & 7 & 2 & 7 & 4 & 7 & 1 \\
		\end{array}
	\end{split}\end{equation*}
	bad-characterの表$\myop{bmB}$は単語に現れる文字の右端の位置になる。
	\begin{equation*}\begin{split}
		\myop{bmB} := \begin{array}{r|rrrr}
			\text{alph.} & G & A & C & \text{その他} \\
			\text{index} & 7 & 6 & 1 & 8 \\
		\end{array}
	\end{split}\end{equation*}
	$\myop{bmG}$と$\myop{bmB}$は次のように組み合わせて使う。
	\begin{equation*}\begin{array}{l}
		\young(AAAAAAGCAGAGAG,GCAGAGAG,:::::::\uparrow) \\
		\downarrow \max(\myop{bmG}_7,7-\myop{bmB}_C) = \max(1,7-1) = 6 \\
		\young(AAAAAAGCAGAGAG,::::::GCAGAGAG) \\
	\end{array}\end{equation*}
	\cite{lecroq}にある少し長い例を挙げておく。
	\begin{equation*}\begin{array}{ll}
		\young(GCATCGCAGAGAGTATACAGTACG,GCAGAGAG,:::::::\uparrow) \\
		\downarrow \myop{bmG}_7 = 1,\quad 7 - \myop{bmB}_A = 1 \\
		\young(GCATCGCAGAGAGTATACAGTACG,:GCAGAGAG,::::::\uparrow) \\
		\downarrow \myop{bmG}_5 = 4,\quad 5 - \myop{bmB}_C = 4 \\
		\young(GCATCGCAGAGAGTATACAGTACG,:::::GCAGAGAG),\quad\text{found} \\
		\downarrow \myop{bmG}_0 = 7 \\
		\young(GCATCGCAGAGAGTATACAGTACG,::::::::::::GCAGAGAG,:::::::::::::::::\uparrow) \\
		\downarrow \myop{bmG}_5 = 4,\quad 5 - \myop{bmB}_C = 4 \\
		\young(GCATCGCAGAGAGTATACAGTACG,::::::::::::::::GCAGAGAG,::::::::::::::::::::::\uparrow) \\
		\downarrow \myop{bmG}_6 = 7,\quad 6 - \myop{bmB}_C = 5 \\
	\end{array}\end{equation*}
	この例では、単語$GCAGAGAG$の位置が一つ見つかった後もその次の位置を探して
	いるが、その際に、bad-characterがないので$\myop{bmB}$を参照をしていない
	ことに注意する。

	\begin{note}[文字列検索での言葉]\label{note:文字列検索での言葉} %{
		文字列検索では次のような言葉が用いられるようだ。
		\begin{description}\setlength{\itemsep}{-1mm} %{
			\item[needle] パターン
			\item[haystack] パターンを含むかもしれない文章
		\end{description} %}
	\end{note} %note:文字列検索での言葉}
	\begin{todo}[やりたいこと]\label{todo:やりたいこと} %{
		\begin{description}\setlength{\itemsep}{-1mm} %{
			\item[Sundayの方法] 実用的な検索方法。通常のコンピュータでは 
			一文字同士の比較とmemcmpのスピードがあまり変わらないことを利用
			する。good-suffixを使わない単純だが効率的な検索方法のようだ。
			\item[DFA] DFAを用いた場合にスキップをどうやって取り入れるかを
			考える。
		\end{description} %}
	\end{todo} %todo:やりたいこと}
%s2:Boyer-Mooreの方法}
\subsection{Sundayの方法}\label{s2:Sundayの方法} %{
	Sundayの方法はBoyer-Mooreの方法で使ったbad-characterの表だけを使って
	次のようにシフトしていく。
	\begin{equation*}\begin{array}{l}
		\young(CATCGCAGAGTAGATACAGTACG,GCAGAGAG,:::::\uparrow) \\
		\downarrow \quad\text{shift} \\
		\young(CATCGCAGAGTAGATACAGTACG,::GCAGAGAG,::::::::\uparrow) \\
		\downarrow \\
		\young(CATCGCAGAGTAGATACAGTACG,::GCAGAGAG,:::::\uparrow) \\
		\downarrow \quad\text{shift} \\
		\young(CATCGCAGAGTAGATACAGTACG,:::::::::::GCAGAGAG,::::::::::\uparrow) \\
	\end{array}\end{equation*}
%s2:Sundayの方法}
%s1:文字列の検索}
\section{バイト集合}\label{s1:バイト集合} %{
	$1$バイト($8$ビット)で表される集合$\myop{uint8\_t}$の演算考える。
	$|\myop{uint8\_t}|=2^8=256$だから、冪集合$\Pow\myop{uint8\_t}$を
	\myop{uint64\_t}の配列で表すと、
	$|\myop{uint8\_t}|/64=4$だから、長さ$4$の配列が必要になる。

	ASCIIは$7$ビットで表される集合の部分集合である。
	wikipediaによると、$7$ビットを次のような領域に分割している。
	\begin{equation*}\begin{split}
		\left.\begin{split}
			000\;0000 \\
			001\;1111
		\end{split}\;\right\} & \quad\text{制御コード} \\
		\left.\begin{split}
			010\;0000 \\
			111\;1110
		\end{split}\;\right\} & \quad\text{印字可能な文字} \\
		\left.\begin{split}
			111\;1112
		\end{split}\;\right\} & \quad\text{制御コード} \\
	\end{split}\end{equation*}
%s1:バイト集合}
\section{CoC}\label{s1:CoC} %{
	設定によってサブシステム間の結合の自由度の高めることよりも、
	規約によってサブシステム間の結合の仕方を決め打ちしてしまった方が
	開発効率が高まることが往々にしてある。
	例えば、次のような自由度は高いが冗長な関数は、
	\begin{lstlisting}[caption=冗長な例, label=code:冗長な例]
	sum: (begin: P, end: P
		, eq: (P, P) -> boolean, inc: P -> P, val: P -> T
		, plus: (T, T) -> T, zero: T) -> T = {
		for (; eq(p, end); p = inc(p)) {
			zero = plus (zero, val p);
		}
		return zero;
	}
	\end{lstlisting}
	イテレータ型を取り入れることでもう少し単純な次のような関数をつくることが
	できる。
	\begin{lstlisting}[caption=単純化した例, label=code:単純化した例]
	sum: (begin: (iterator T), end: (iterator T)
		, plus: (T, T) -> T, zero: T) -> T = {
		return sum (begin, end, (iterator T).equals (iterator T).increment
		, (iteartor T).get, plus, zero);
	}
	\end{lstlisting}
	この場合は、イテレータ型というのが規約になる。
	\begin{lstlisting}[caption=イテレータ型, label=code:イテレータ型]
	iterator (T): type = {
		equals: ((iterator T) x) -> boolean;
		increment: () -> self;
		get: () -> T;
	}
	\end{lstlisting}
%s1:CoC}
\section{Diffie-Hellmannのカギ共有}\label{s1:Diffie-Hellmannのカギ共有} %{
	Diffie-Hellmannのカギ共有は、可換半群$A=(A,*,1)$の元$a,b,c$を用いて、
	次の図で表すことができる。
	\begin{equation*}\xymatrix{
		\text{person }\alpha & \text{common} & \text{person }\beta \\
		a\in A \ar[d] & c\in A \ar[ld] \ar[rd] & b\in A \ar[d] \\
		\ar[d] \ar[rrd]^(0.2){a*c} & & \ar[d] \ar[lld]_(0.2){b*c} \\
		a*(b*c) & & b*(a*c) \\
	}\end{equation*}
	この図にある要素には次のような名称が与えられている。
	\begin{description}\setlength{\itemsep}{-1mm} %{
		\item[公開アルゴリズム] 可換半群$A$
		\item[公開カギ] $c\in A$
		\item[秘密カギ] $a,b\in A$
		\item[共有カギ] $a*b*c\in A$ \\
		$A$が可換半群だから、$a*(b*c)=b*(a*c)$となり、$\alpha$と$\beta$で
		カギ$a*b*c$を共有できる。
		\item[交換カギ] $c*a,c*b\in A$ \\
		交換カギ$a*c$と$b*c$の両方が盗まれた場合でも、共有カギ$a*b*c$を
		計算することが困難であれば、共有カギを秘密にすることができる。
	\end{description} %}
	$A$が群であった場合は、公開カギ$c$の逆元$c^{-1}$が予め計算されてしまう
	ので、共有鍵を秘密にすることができなくなる。
	\begin{equation*}\begin{split}
		a = c^{-1}*a*c \quad\text{for all }a,c\in A
	\end{split}\end{equation*}
	幸か不幸か、計算しやすいものは群になることが多い。
	そこで、$A$の作用域を持ち込んで交換カギから秘密カギの逆元を計算できなく
	する。集合$C$を可換半群$A$の作用域として、$A$の$C$への作用$\rhd$で書く。
	\begin{equation*}\begin{split}
		a\rhd b\rhd c = (a*b)\rhd c \quad\text{for all } a,b\in A,\; c\in C
	\end{split}\end{equation*}
	カギ交換のプロトコルは次のような図で表される。
	\begin{equation*}\xymatrix{
		\text{person }\alpha & \text{common} & \text{person }\beta \\
		a\in A \ar[d] & c\in C \ar[ld] \ar[rd] & b\in A \ar[d] \\
		\ar[d] \ar[rrd]^(0.2){a\rhd c} & & \ar[d] \ar[lld]_(0.2){b\rhd c} \\
		a\rhd b\rhd c & & b\rhd a\rhd c \\
	}\end{equation*}
	この場合、交換カギから秘密カギを求めるアルゴリズムは次のように定式化する
	ことができる。
	\begin{equation*}\begin{split}
		\text{for given } c,d\in C 
		\text{ find } a\in A \text{ such that } d = a\rhd c
	\end{split}\end{equation*}
	この処理が困難なことが保証されていれば、$A$として簡単な半群を使うことが
	できる。
	$C$を$\sei/p\sei$、$A$を$\sizen$として、$\rhd$を$k\rhd n=n^k\mod p$
	としたものが、Diffie-Hellmannのオリジナルらしい。この場合でも、
	公開カギ$c$の逆数$c^{-1}$を用いると、$c^{-a}c^a=1\mod p$となり、
	交換カギ$c^a$に対して$c^{-1}$を$a$回掛けると秘密カギ$a$がわかってしまう。
	秘密カギが$\sei_p$上に一様分布しているとすると、秘密カギを見つける手数
	は$p$程度になる。
	\begin{equation}\label{eq:離散対数問題}\begin{split}
		\text{for given } c,d\in\sei/p\sei 
		\text{ find } a\in\sizen \text{ such that } d = c^a\mod p
	\end{split}\end{equation}
	この問題を効率良く解く方法を見つけることを離散対数問題というらしい。

	Wikipediaによると、RSAとは離散対数問題の困難さを利用した暗号化の方法で
	ある。Diffie-HellmannのオリジナルとRSAの両者共に離散対数問題の困難さを
	利用しているために、両者は混同されがちだが、その目的は異なる。
	$n\in N$をパラメーター、$A_n$を$N$でパラメトライズされたカギとなる群、
	$M$を平文とし、暗号化を作用$\rhd_n:A_n\times M\to M$で定義する。
	そして、次のようなプロトコルによって、暗号化を定義したものがRSAである。
	\begin{equation*}\xymatrix{
		\text{person }\alpha & \text{person }\beta \\
		a,a_n^{-1}\in A_n \ar[d] & m\in M \ar[dd] \\
		\ar[dd] \ar[rd]^{a,n} & & \\
		& \ar[ld]_{a\rhd_n m} \\
		a_n^{-1}\rhd_n a\rhd_n m & \\
	}\end{equation*}
	暗号化のカギ$a$から復号化のカギ$a_n^{-1}$がバレなければ、$\beta$から
	$\alpha$へ送信される平文の秘匿性が保たれる。この場合は、暗号化の頑強性を
	群の逆元を求める難しさに帰着させている。
	\begin{equation*}\begin{split}
		p,q\text{が素数} \implies (m^a\bmod n)^b\bmod n = m \quad\text{where }
		n := pq \\
		\quad\text{for all } m<pq,\quad ab=1\mod (p-1)(q-1)
	\end{split}\end{equation*}
%s1:Diffie-Hellmannのカギ共有}
}\endgroup %}
