\begingroup %{
\newcommand{\bou}{{\,|\,}}
\newcommand{\boug}{{\,\big|\,}}
\newcommand{\bougg}{{\,\bigg|\,}}
\newcommand{\qD}{{\op{D}}}
\newcommand{\qI}{{\op{I}}}
{\setlength\arraycolsep{2pt}
%
\section{On Fixed Point Equations over Commutative Semirings}\label{s1:EKL07} %{
	\cite{EKL07}のノートを書いておく。
	\cite{EKL07}は\cite{Hopkins99}に触発させて書かれたものと思われる。

\subsubsection{$\omega$-連続}\label{s3:オメガ-連続} %{
	\begin{itemize}\setlength{\itemsep}{-1mm} %{
		\item 自然な順序 \\
		半環$R$に次の半順序$\le$が定義できるとき、$\le$を自然な順序という。
		\begin{equation*}\begin{split}
			r_1\le r_2 \iff \text{ there exists } s\in R \text{ such that } 
			r_1 + s = r_2
		\end{split}\end{equation*}
		自然数には自然な順序が定義できるが、整数には自然な順序が定義できない。
		\item $*$-完備 \\
		任意の$R$の元に対してKleeneスターが定義できると仮定する。
		$1^*=\sum_{n\in\sizen}1\not\in\sizen$だから自然数は$*$-完備でないが、
		自然数に無限大を付け加えたものは$*$-完備になる。
		\item $\omega$-連続 \\
		自然な順序が定義され、$*$-完備であり、任意の$n\in\sizen$に対して
		$r_n\in R$ならば、$\sum_{n\in\sizen}r_n\in R$となる。
		$*$-完備はKleeneスターが収束することを要求するが、$\omega$-連続は、
		もっと強く、任意の可算和が収束することを要求する。非常に強い要請で、
		通常扱う半環では満たされない。
		自然数に無限大を付け加えたものは$\omega$-連続になる。
	\end{itemize} %}
%s3:オメガ-連続}

\subsubsection{根の逐次近似}\label{s3:根の逐次近似} %{
	\begin{itemize}\setlength{\itemsep}{-1mm} %{
		\item Kleeneの逐次近似 \\
		$\omega$-連続な半環上の任意の形式級数$f$に対して、
		$x=\plr{f\bou x}$の最小不動点は存在して、
		$\sup_{n\in\sizen}\plr{f^n\bou 0}$で与えられる。
		\begin{equation*}\begin{split}
			x_0 &= \plr{f\bou 0} \\
			x_1 &= \plr{f\bou x_0} \\
			\vdots \\
			x_{n+1} &= \plr{f\bou x_n} \\
		\end{split}\end{equation*}
		%
		\item Hopkins-Kozenの逐次近似 \\
		ベキ等なcc-半環上の任意の形式級数$f$に対して、
		$x=\plr{f\bou x}$の最小不動点$x_1$は存在して、
		次のように与えられる。
		\begin{equation*}\begin{split}
			x_0 &= \plr{f\bou 0} \\
			x_1 &= x_0 + \plr{\partial f\bou x_0}x_1 \\
		\end{split}\end{equation*}
		%
		\item Newtonの逐次近似 \\
		\begin{equation*}\begin{split}
			x_0 &= \plr{f\bou 0} \\
			x_1 &= \plr{f\bou x_0} + \plr{\partial f\bou x_0}\plr{x_1 - x_0} \\
			\vdots \\
			x_{n+1} &= \plr{f\bou x_n} + \plr{\partial f\bou x_n}\plr{x_{n+1} - x_n} \\
		\end{split}\end{equation*}
	\end{itemize} %}

	ベキ等半環の場合、Newtonの逐次近似からHopkins-Kozenの逐次近似が導かれる。
	\begin{equation*}\begin{split}
			x_{n+1} 
			&= \plr{f\bou x_n} + \plr{\partial f\bou x_n}\plr{x_{n+1} - x_n} \\
			&= \plr{f\bou x_n} - \plr{\partial f\bou x_n}x_n
				+ \plr{\partial f\bou x_n}x_{n+1} \\
			&= \plr{f\bou 0} + \plr{\partial f\bou x_n}x_{n+1} \\
	\end{split}\end{equation*}
	最後の式変形で、ベキ等半環の場合、次の式が成り立つことを使っている。
	\begin{equation*}\begin{split}
		\plr{f\bou x} - \plr{\partial f\bou x}x = \plr{f\bou 0}
	\end{split}\end{equation*}
%s3:根の逐次近似}

	\begin{proposition}[Taylorの定理]\label{prop:Taylorの定理} %{
		$R$をcc-半環とする。
		任意の$\plr{f\bou x}\in R\bblr{x}$と$y\in R$に対して次の不等式が
		成り立つ。
		\begin{equation*}\begin{split}
			\plr{f\bou x} + \plr{\partial f\bou x}y \le \plr{f\bou x + y}
			\le \plr{f\bou x} + \plr{\partial f\bou x + y}y
		\end{split}\end{equation*}
	\end{proposition} %prop:Taylorの定理}
	\begin{proof} %{
		まず、$R\blr{x}$が$x$で生成されることを用いて、多項式に対して
		命題の不等式が成り立つことを証明する。
		\begin{itemize}\setlength{\itemsep}{-1mm} %{
			\item $\plr{f\bou x}=a\in R$の時
			\begin{equation*}\begin{split}
				\plr{f\bou x} + \plr{\partial f\bou x}y = \plr{f\bou x + y}
				= \plr{f\bou x} + \plr{\partial f\bou x + y}y = a
			\end{split}\end{equation*}
			\item $\plr{f\bou x}=x$の時
			\begin{equation*}\begin{split}
				\plr{f\bou x} + \plr{\partial f\bou x}y = \plr{f\bou x + y}
				= \plr{f\bou x} + \plr{\partial f\bou x + y}y = x + y
			\end{split}\end{equation*}
			\item $\plr{f\bou x}=\plr{g\bou x} + \plr{h\bou x}$の時 \\
			多項式の項数についての帰納法で証明する。
			\begin{equation*}\begin{split}
				\plr{f\bou x + y} &= \plr{g\bou x + y} + \plr{h\bou x + y} \\
				&\ge \plr{g\bou x} + \plr{\partial g\bou x}y
					+ \plr{h\bou x} + \plr{\partial h\bou x}y \\
				&= \plr{f\bou x} + \plr{\partial f\bou x}y \\
				%
				\plr{f\bou x + y} &= \plr{g\bou x + y} + \plr{h\bou x + y} \\
				&\le \plr{g\bou x} + \plr{\partial g\bou x + y}y
					+ \plr{h\bou x} + \plr{\partial h\bou x + y}y \\
				&= \plr{f\bou x} + \plr{\partial f\bou x + y}y \\
			\end{split}\end{equation*}
			\item $\plr{f\bou x}=\plr{g\bou x}\plr{h\bou x}$の時 \\
			多項式の次数についての帰納法で証明する。
			$y$の二次の項を捨てることで次の式が成り立つことがわかる。
			\begin{equation*}\begin{split}
				\plr{f\bou x + y} &= \plr{g\bou x + y}\plr{h\bou x + y} \\
				&\ge \plrgg{\plr{g\bou x} + \plr{\partial g\bou x}y}
					\plrgg{\plr{h\bou x} + \plr{\partial h\bou x}y} \\
				&\ge \plr{g\bou x}\plr{h\bou x}
					+ \plrgg{\plr{\partial g\bou x}\plr{h\bou x} 
					+ \plr{g\bou x}\plr{\partial h\bou x}}y \\
				&= \plr{f\bou x} + \plr{\partial f\bou x}y
			\end{split}\end{equation*}
			また、$\plr{g\bou x}\le\plr{g\bou x + y}$により、次の式が成り立つ
			ことがわかる。
			\begin{equation*}\begin{split}
				\plr{f\bou x + y} &= \plr{g\bou x + y}\plr{h\bou x + y} \\
				&\le \plrgg{\plr{g\bou x} + \plr{\partial g\bou x + y}y}
					\plr{h\bou x + y} \\
				&= \plr{g\bou x}\plr{h\bou x + y} 
					+ \plr{\partial g\bou x + y}\plr{h\bou x + y}y \\
				&= \plr{g\bou x}\plr{h\bou x} + \plrgg{
					\plr{g\bou x}\plr{\partial h\bou x + y}
					+ \plr{\partial g\bou x + y}\plr{h\bou x + y}}y \\
				&\le \plr{g\bou x}\plr{h\bou x} + \plrgg{
					\plr{g\bou x + y}\plr{\partial h\bou x + y}
					+ \plr{\partial g\bou x + y}\plr{h\bou x + y}}y \\
				&= \plr{f\bou x} + \plr{\partial f\bou x + y}y
			\end{split}\end{equation*}
		\end{itemize} %}
		次に、可算個の多項式の和によって任意の形式級数を書くことが
		できるとして($\omega$-連続の定義)、形式級数に対して証明する。
	\end{proof} %}

	\begin{proposition}[Taylorの定理(ベキ等)]\label{prop:Taylorの定理(ベキ等)} %{
		$R$をベキ等なcc-半環とする。
		任意の$\plr{f\bou x}\in R\bblr{x}$と$y\in R$に対して次の式が成り立つ。
		\begin{equation*}\begin{split}
			\plr{f\bou x + y} = \plr{f\bou x} + \plr{\partial f\bou x + y}y
		\end{split}\end{equation*}
	\end{proposition} %prop:Taylorの定理(ベキ等)}
	\begin{proof} %{
		ベキ等半環では、任意の$n\in\sizen$に対して次の式が成り立つ。
		\begin{equation*}\begin{split}
			\plr{x+y}^{n+1} = \sum_{k=0}^{n+1} x^ky^{n+1-k}
			= x^{n+1} + y\sum_{k=0}^n x^ky^{n-k}
			= x^{n+1} + y\plr{x+y}^n
		\end{split}\end{equation*}
		したがって、任意の単項式に対して命題が成り立つ。
		そして、係数環が$\omega$-連続だから、単項式の無限和に対しても命題が
		成り立つ。
	\end{proof} %}
%s1:EKL07}
}\endgroup %}
