\begingroup %{
\newcommand{\bfg}{\mathbf{g}}
\newcommand{\calT}{\mathcal{T}}
{\setlength\arraycolsep{2pt}
%
\section{ベキ等半環上の文法}\label{s1:ベキ等半環上の文法} %{
	\begin{todo}[一般化]\label{todo:一般化} %{
		$R$を可換環、$V$を$R$-代数とする。$V$に代数の加法と乗法の他に、
		双線型射$\beta:V\otimes V\to V$が定義されているとする。
		次の式を満たす$x_t\in V\bblr{t}$を考える。
		\begin{equation*}\begin{split}
			x_t = 1 + t\beta\plr{x_t\otimes x_t},\quad x_0 = 1
		\end{split}\end{equation*}
	\end{todo} %todo:一般化}
\subsection{多項式の摂動解}\label{s2:多項式の摂動解} %{
	$R$を可換環、$\bfg:=\set{g_0,g_1,\dots}$を可算集合とする。
	$x_t\in R\dlr{\bfg}\bblr{t}$を次の式を満たす形式級数とし、
	\begin{equation*}\begin{split}
		x_t = 1 + t(g|x_t)x_t \quad\text{where }
		(g|x) := \sum_{n\in\sizen} g_nx^n\in R\dlr{\bfg,x}
	\end{split}\end{equation*}
	$x_t$の摂動解を求めることを考える。
	\begin{equation*}\begin{split}
		x_t = \sum_{n\in\sizen} t^nx_n
		\quad\text{where } x_n = \epsilon_t\delta_t^nx_t \in R\dlr{\bfg}
	\end{split}\end{equation*}
	ここで、$\delta_t$を$t$についてのBrzozowski微分、
	$\epsilon_t$を$t\to0$の極限をとる代数射とする。
	係数$x_n$を求めるために、微分を計算すると次のようになる。
	\begin{equation*}\begin{split}
		\delta_tx_t &= (g|x_t)x_t \\
		\delta_t^2x_t &= (\delta_g g|x_t)x_t + (\epsilon_t g|x_t)(g|x_t)x_t \\
	\end{split}\end{equation*}
	ここで、$R\dlr{\bfg,x}$の線形二項演算$\delta$を次のように定義する。
	\begin{equation*}\begin{split}
		(\delta_\alpha\beta|x) &= \sum_{n\in\sizen} \beta_{n+1}\sum_{k=0}
		1^{\otimes(n-k)}\otimes (\alpha|x)x\otimes x^{\otimes k}
		\quad\text{for all } \alpha,\beta\in R\dlr{\bfg,x}
	\end{split}\end{equation*}
	一般の多項式に対する$\delta$の作用は後で考えるとして、
	可換環上の多項式に近い、次の形の多項式に限定すると、
	\begin{equation*}\begin{split}
		(f|x) = \sum_{n\in\sizen} f_nx^n
		\quad\text{where } f_n\in R\dlr{\bfg}
	\end{split}\end{equation*}
	$\delta$の作用は次のように、行列の形で書くことができる。
	\begin{equation*}\begin{split}
		(\delta_\alpha f|x) = F^T(\alpha|x)SX \quad\text{where } \\
		F := \begin{pmatrix}
			f_0 \\ f_1 \\ f_2 \\ f_3 \\ \vdots
		\end{pmatrix} ,\quad X := \begin{pmatrix}
			1 \\ x \\ x^2 \\ x^3 \\ \vdots
		\end{pmatrix},\quad S := \begin{pmatrix}
			0 & 0 & 0 & 0 & \cdots \\
			0 & 1 & 0 & 0 & \cdots \\
			0 & 1 & 1 & 0 & \cdots \\
			0 & 1 & 1 & 1 & \cdots \\
			\vdots & \vdots & \vdots & \vdots & \cdots \\
		\end{pmatrix}
	\end{split}\end{equation*}
	したがって、系列$E_n,G,X_t$を次のように定義すると、
	\begin{equation*}\begin{split}
		E_n^T := \begin{pmatrix}
			\cdots & 0 & 1 & 0 & \cdots
		\end{pmatrix},\quad G := \sum_{n\in\sizen} g_nE_n\quad 
		X_t := \sum_{n\in\sizen} x_t^nE_n
	\end{split}\end{equation*}
	$(g|x_t)=G^TX_t$と書けて、微分は次のようになる。
	\begin{equation*}\begin{split}
		\delta_tX_t &= \plr{G^TX_t}SX_t \\
		\delta_t^2X_t &= \plrg{G^T\plr{G^TX_t}SX_t}SX_t
		+ \plr{G^T\epsilon_tX_t}S\plr{G^TX_t}SX_t \\
	\end{split}\end{equation*}

	$\delta_t^nX_t$を計算するために、二分木に対応させる。
	葉でない頂点が$n$個ある二分木の集合を$\calT_n$、
	$\calT_*:=\cup_{n\in\sizen}\calT_n$とし、写像
	$\beta:\calT_*\times\calT_*\to\calT_*$を次のように定義する。
	\begin{equation*}\begin{split}
		\beta(t_1, t_2) := \vcenter{\xymatrix@R=4pt@C=4pt{
			& \bullet \hen[ld] \hen[rd] \\
			t_1 & & t_2
		}} \quad\text{for all } t_1,t_2\in\calT_*
	\end{split}\end{equation*}
	これらを$R$線形に拡張して、線形射$\phi_t:R\calT_*\to R\dlr{\bfg}\bblr{t}$
	を次のように定義する。
	\begin{equation*}\begin{split}
		\phi_t\bullet := X_t,\quad \phi_t\beta\plr{t_1\otimes t_2}
		:= \plr{G^T\epsilon_t^{\deg t_2}\phi_tt_1}S\plr{\phi_tt_2}
		\quad\text{for all } t_1,t_2\in\calT_*
	\end{split}\end{equation*}
	そして、$\phi_t\calT_n$を次のように定義すると、
	\begin{equation*}\begin{split}
		\phi_t\calT_n := \sum_{t\in\calT_n}\phi_tt
		\quad\text{for all } n\in\sizen
	\end{split}\end{equation*}
	$\phi_t\calT_0=X_t$かつ$\delta_t\phi_t\calT_0=\phi_t\calT_1=(G^TX_t)SX_t$
	が成り立つ。そして、帰納法によって、任意の$n\in\sizen$に対して、
	次の式が成り立つことが示され、
	\begin{equation*}\begin{split}
		\delta_t\phi\calT_{n+1} &= \delta_t\sum_{k=0}^n
			\plr{G^T\epsilon_t^k\phi_t\calT_{n-k}}S\phi_t\calT_k \\
		&= \plr{G^T\delta_t\phi_t\calT_n}S\phi_t\calT_0 + \sum_{k=0}^n
			\plr{G^T\epsilon_t^{k+1}\phi_t\calT_{n-k}}S\delta_t\phi_t\calT_k \\
		&= \plr{G^T\phi_t\calT_{n+1}}S\phi_t\calT_0 + \sum_{k=0}^n
			\plr{G^T\epsilon_t^{k+1}\phi_t\calT_{n-k}}S\phi_t\calT_{k+1} \\
		&= \sum_{k=0}^{n+1} 
			\plr{G^T\epsilon_t^k\phi_t\calT_{n-k}}S\phi_t\calT_k \\
	\end{split}\end{equation*}
	$\delta_t^nX_t=\phi_t\calT_n$となることがわかる。したがって、
	$X_t$は$X_t=\sum_{n\in\sizen}t^n\epsilon_t\phi_t\calT_n$と書くことができる。
	$\epsilon_t\phi_t$は二分木を次のように行きがけ順に辿ればよい。
	\begin{equation}\label{eq:二分木から多項式その一}\begin{split}
		\xymatrix@R=2ex@C=2ex{
			& & \bullet \ar[ld]_{G^T} \ar[rd]_S \\
			& \bullet \ar[ld]_{G^T} \ar[rd]_S & & \mathbf{1} \\
			\mathbf{1} & & \mathbf{1}
		} \sim G^TG^T\mathbf{1}S\mathbf{1}S\mathbf{1}
	\end{split}\end{equation}
	ここで、$\mathbf{1}:=\epsilon_tX_t=(1\;1\;\cdots)^T$とする。
	低次の項を見ると次のようになっている。
	\begin{equation*}\begin{array}{rclclcl}
		\epsilon_t\phi_t\vcenter{\xymatrix@R=4pt@C=4pt{
			\bullet
		}} &=& \mathbf{1} & & &\sim& (0) \\
		\epsilon_t\phi_t\vcenter{\xymatrix@R=4pt@C=4pt{
			& \bullet \hen[ld] \hen[rd] \\
			\bullet & & \bullet
		}} &=& G^T\mathbf{1}S\mathbf{1}
			&=& S\mathbf{1}(G^T\mathbf{1}) &\sim& (1,0) \\
		\epsilon_t\phi_t\vcenter{\xymatrix@R=4pt@C=4pt{
			& & \bullet \hen[ld] \hen[rd] \\
			& \bullet \hen[ld] \hen[rd] & & \bullet \\
			\bullet & & \bullet
		}} &=& G^TG^T\mathbf{1}S\mathbf{1}S\mathbf{1}
			&=& S\mathbf{1}(G^TS\mathbf{1})(G^T\mathbf{1}) &\sim& (1,1,0) \\
		\epsilon_t\phi_t\vcenter{\xymatrix@R=4pt@C=4pt{
			& \bullet \hen[ld] \hen[rd] \\
			\bullet & & \bullet \hen[ld] \hen[rd] \\
			& \bullet & & \bullet
		}} &=& G^T\mathbf{1}SG^T\mathbf{1}S\mathbf{1}
			&=& S^2\mathbf{1}(G^T\mathbf{1})(G^T\mathbf{1}) &\sim& (2,0,0) \\
	\end{array}\end{equation*}
	この例と図\eqref{eq:二分木から多項式その一}から、二分木を左右反転して、
	次のように行きがけ順に辿っても良いことがわかる。
	\begin{equation}\label{eq:二分木から多項式その二}\begin{split}
		\xymatrix@R=2ex@C=2ex{
			& \bullet \ar[ld]_S \ar[rd]^{G^T} \\
			\mathbf{1} & & \bullet \ar[ld]_S \ar[rd]^{G^T} \\
			& \mathbf{1} & & \mathbf{1}
		} \sim S\mathbf{1}(G^TS\mathbf{1})(G^T\mathbf{1})
	\end{split}\end{equation}
	この写像を$\psi$とすると、$R$を二分木を左右反転する操作として、
	$\psi$は次のように定義され、
	\begin{equation*}\begin{split}
		\psi := \epsilon_t\phi_tR
	\end{split}\end{equation*}
	次の漸化式が得られる。
	\begin{equation*}\begin{split}
		\psi\bullet = \mathbf{1},\quad 
		\psi\beta\plr{t_1\otimes t_2} = S\psi t_1\plr{G^T\psi t_2}
		\quad\text{for all } t_1,t_2\in\calT_*
	\end{split}\end{equation*}
	任意の$k\in\sizen$に対して$S_k:=S^k\mathbf{1}$とおくと、
	任意の$t\in\clT_n$に対して$\psi t$は次の形になる。
	\begin{equation*}\begin{split}
		\psi t = S_{k_0}\plr{G^TS_{k_1}}\cdots\plr{G^TS_{k_{n-1}}}\plr{G^TS_0}
		\quad\text{where } k_0 + k_1 +\cdots+ k_{n-1} = n
	\end{split}\end{equation*}
	$\chi:\sizen^+\to R\dlr{\bfg}$を次のように定義すると、
	\begin{equation*}\begin{split}
		\chi(k_0,\dots,k_n) := S_{k_0}\plr{G^TS_{k_1}}\cdots\plr{G^TS_{k_n}}
		\quad\text{for all } k_i\in\sizen
	\end{split}\end{equation*}
	任意の$t\in\clT_n$に対して$\psi t=\chi w$となる$w\in\sizen^{n+1}$が唯一つ
	定まる。これを$\chi^{-1}\psi t$と書くことにする。
	$\psi$の定義から、次の式が成り立つ。
	\begin{equation*}\begin{split}
		\left\{\begin{split}
			\chi^{-1}\psi u &= (k_0,\dots,k_m) \\
			\chi^{-1}\psi v &= (l_0,\dots,l_n)
		\end{split}\right.\implies \chi^{-1}\psi\beta\plr{u\otimes v} 
		= (k_0+1,k_1,\dots,k_m,l_0,\dots,l_n) \\
		\quad\text{for all } u\in\clT_m,\; v\in\clT_n
	\end{split}\end{equation*}
	文字列を連結して、左端の数字を一つインクリメントすればよい。

	以下、$\psi\clT_n := \sum_{t\in\clT_n}\psi t$と略記する。
	$X_t = \sum_{n\in\sizen}t^n\psi\clT_n$と書け、特に、
	$x_t = \sum_{n\in\sizen}t^n E_1^T\psi\clT_n$となる。

	各$\psi\clT_n$は
	、ある$G^n_k\in R\dlr{\bfg}$があって、次の形になる。
	\begin{equation*}\begin{split}
		\psi\clT_n = \sum_{k=0}^n S_{k}G^n_k
	\end{split}\end{equation*}
	$G^n_k$は$\chi^{-1}\psi\clT_n$から直ちに求まり、$\chi^{-1}\psi\clT_n$は
	簡単に計算できて、次のようになる。
	\begin{equation*}\begin{split}
		\chi^{-1}\psi\clT_0 &= (0) \\
		\chi^{-1}\psi\clT_1 &= (1, 0) \\
		\chi^{-1}\psi\clT_2 &= (1, 1, 0) + (2, 0, 0) \\
		\chi^{-1}\psi\clT_3 &= (1, 1, 1, 0) + (1, 2, 0, 0) + (2, 0, 1, 0) 
			+ (2, 1, 0, 0) \\
			&\, + (3, 0, 0, 0) \\
		\chi^{-1}\psi\clT_4 &= (1, 1, 1, 1, 0) + (1, 1, 2, 0, 0) 
			+ (1, 2, 0, 1, 0) + (1, 2, 1, 0, 0) \\
			&\, + (1, 3, 0, 0, 0) + (2, 0, 1, 1, 0) + (2, 0, 2, 0, 0) 
			+ (2, 1, 0, 1, 0) \\
			&\, + (2, 1, 1, 0, 0) + (2, 2, 0, 0, 0) + (3, 0, 0, 1, 0) 
			+ (3, 0, 1, 0, 0) \\
			&\,+ (3, 1, 0, 0, 0) + (4, 0, 0, 0, 0)
	\end{split}\end{equation*}
	$\bfg$が互いに可換の場合には、数字の文字列の左端の文字以外は可換になり、
	次のようになる。
	\begin{equation*}\begin{split}
		\chi^{-1}\psi\clT_0 &= (0) \\
		\chi^{-1}\psi\clT_1 &= (1, 0) \\
		\chi^{-1}\psi\clT_2 &= (1, 1, 0) + (2, 0, 0) \\
		\chi^{-1}\psi\clT_3 &= (1, 1, 1, 0) + (1, 2, 0, 0) + 2(2, 1, 0, 0) 
			+ (3, 0, 0, 0) \\
		\chi^{-1}\psi\clT_4 &= (1, 1, 1, 1, 0) + 3(1, 2, 1, 0, 0) 
			+ (1, 3, 0, 0, 0) \\
			&\, + 3(2, 1, 1, 0, 0) + 2(2, 2, 0, 0, 0) + 3(3, 1, 0, 0, 0) 
			+ (4, 0, 0, 0, 0)
	\end{split}\end{equation*}

	ここでは、$G^n_k$の満たす漸化式を求めよう。$n<k$で$G^n_k=0$とすると、
	Dyck分解による次の式から、
	\begin{equation*}\begin{split}
		\psi\calT_{n+1} &= \sum_{k=0}^n S\psi\calT_k\plr{G^T\psi\calT_{n-k}}
		= \sum_{k=0}^n\sum_{r=0}^k S_{r+1}G^k_r 
			\sum_{s=0}^{n-k}(G^TS_s)G^{n-k}_s \\
		&= \sum_{k=0}^n\sum_{r=0}^n S_{r+1}G^k_r 
			\sum_{s=0}^{n-k}(G^TS_s)G^{n-k}_s
		= \sum_{r=0}^n S_{r+1}\sum_{k=0}^n G^k_r 
			\sum_{s=0}^{n-k}(G^TS_s)G^{n-k}_s \\
	\end{split}\end{equation*}
	次の漸化式が得られる。
	\begin{equation}\label{eq:Gの漸化式}\begin{split}
		G^n_0 = \is{n=0},\quad G^{n+1}_{k+1} 
		= \sum_{r=k}^n G^r_k \sum_{s=0}^{n-r} \plr{G^TS_s}G^{n-r}_s
		\quad\text{for all } n,k\in\sizen
	\end{split}\end{equation}

	$S_n$は次のようになっている。
	\begin{equation*}\begin{split}
		S_0 = \begin{pmatrix}
			1 \\ 1 \\ 1 \\ 1 \\ 1 \\ \vdots
		\end{pmatrix},\quad S_1 = \begin{pmatrix}
			0 \\ 1 \\ 2 \\ 3 \\ 4 \\ \vdots
		\end{pmatrix},\quad S_2 = \begin{pmatrix}
			0 \\ 1 \\ 3 \\ 6 \\ 10 \\ \vdots
		\end{pmatrix},\quad S_3 = \begin{pmatrix}
			0 \\ 1 \\ 4 \\ 10 \\ 20 \\ \vdots
		\end{pmatrix}
	\end{split}\end{equation*}
	$S_n=(s^n_0,s^n_1,\dots)^T$とすると、$s^n_k$は次の漸化式を満たし、
	\begin{equation*}\begin{split}
		s^0_k = 1,\quad s^{n+1}_k = \sum_{r=1}^k s^n_r
		\quad\text{for all } n,k\in\sizen
	\end{split}\end{equation*}
	次のように二項係数で与えられる。
	\begin{equation*}\begin{split}
		s^n_{k+1} = \binom{k+n}{n} \quad\text{for all } n,k\in\sizen
	\end{split}\end{equation*}
	\begin{proof} %{
		二項係数のPascalの公式を繰り返し用いると、
		\begin{equation*}\begin{split}
			\binom{n+1}{k+1} = \binom{n}{k} + \binom{n}{k+1}
		\end{split}\end{equation*}
		次のようにして$s^n_k$の漸化式が得られる。
		\begin{equation*}\begin{split}
			\binom{k+n+1}{n+1} &= \binom{k+n}{n} + \binom{k+n}{n+1}
			= \binom{k+n}{n} + \binom{k+n-1}{n} + \binom{k+n-1}{n+1} = \cdots \\
			&= \binom{k+n}{n} + \binom{k+n-1}{n} +\cdots+ \binom{n+1}{n+1} \\
			&= \binom{k+n}{n} + \binom{k+n-1}{n} +\cdots+ \binom{n}{n} \\
			&= s^n_{k+1} + s^n_k +\cdots+ s^n_1 \\
		\end{split}\end{equation*}
	\end{proof} %}
	$S_n$の具体的な数値を表\ref{table:低次のS}に挙げておく。

\begin{table}[ht]
\begin{center}
\begin{tabular}{r|rrrrrrrrrrr}
  \hline
$n\backslash k$ & 0 & 1 & 2 & 3 & 4 & 5 & 6 & 7 & 8 & 9 & 10 \\ 
  \hline
	0 & 1 &  1 &   1 &   1 &   1 &   1 &   1 &   1 &   1 &   1 &   1 \\ 
  1 & 0 &  1 &   2 &   3 &   4 &   5 &   6 &   7 &   8 &   9 &  10 \\ 
  2 & 0 &  1 &   3 &   6 &  10 &  15 &  21 &  28 &  36 &  45 &  55 \\ 
  3 & 0 &  1 &   4 &  10 &  20 &  35 &  56 &  84 & 120 & 165 & 220 \\ 
  4 & 0 &  1 &   5 &  15 &  35 &  70 & 126 & 210 & 330 & 495 & 715 \\ 
  5 & 0 &  1 &   6 &  21 &  56 & 126 & 252 & 462 & 792 & 1287 & 2002 \\ 
  6 & 0 &  1 &   7 &  28 &  84 & 210 & 462 & 924 & 1716 & 3003 & 5005 \\ 
  7 & 0 &  1 &   8 &  36 & 120 & 330 & 792 & 1716 & 3432 & 6435 & 11440 \\ 
  8 & 0 &  1 &   9 &  45 & 165 & 495 & 1287 & 3003 & 6435 & 12870 & 24310 \\ 
  9 & 0 &  1 &  10 &  55 & 220 & 715 & 2002 & 5005 & 11440 & 24310 & 48620 \\ 
   \hline
\end{tabular}
\end{center}\caption{$S_n$}\label{table:低次のS}
\end{table}

	任意の$n\in\sizen$に対して$s^n_1=1$なので、代数式の摂動解は
	次のようになる。
	\begin{equation*}\begin{split}
		x_t = 1 + t(g|x_t)x_t \impliedby x_t = \sum_{n\in\sizen}t^n x_n,\quad
		x_n = \sum_{k=0}^n G^n_k
	\end{split}\end{equation*}
	特に、$N\in\sizen$として、次の代数式を考えよう。
	\begin{equation*}\begin{split}
		x_t = 1 + tx_t^{N+1}
	\end{split}\end{equation*}
	この代数式の摂動解は$(N+1)$-分木の数を与える。
	\begin{itemize}\setlength{\itemsep}{-1mm} %{
		\item $N=0$の時 \\
		この時は、$E_0^TS_n=\is{n=0}$より、$G^n_k$の漸化式は次のようになる。
		\begin{equation*}\begin{split}
			G^{n+1}_{k+1} 
			= \sum_{r=k}^n G^r_k \sum_{s=0}^{n-r} \plr{E_0^TS_s}G^{n-r}_s
			= G^n_k
		\end{split}\end{equation*}
		したがって、$G^n_k=\is{n=k}$となる。
		%
		\item $N=1$の時 \\
		この時は、任意の$n\in\sizen$に対して$E_1^TS_n=1$より、
		$x_n=\sum_{k=0}^nG^n_k$とすると、$G^n_k$の漸化式は次のようになる。
		\begin{equation*}\begin{split}
			G^{n+1}_{k+1} 
			= \sum_{r=k}^n G^r_k \sum_{s=0}^{n-r} \plr{E_1^TS_s}G^{n-r}_s
			= \sum_{r=k}^n G^r_k x_{n-r}
		\end{split}\end{equation*}
		特に、$k$について和をとってしまうと、Catalan数に満たす漸化式
		$x_{n+1} = \sum_{r=0}^n x_rx_{n-r}$が導かれる。
		さらに、表\ref{table:Nが1の場合}から、次の式が成り立っていることが
		予想される。
		\begin{equation*}\begin{split}
			G^{n+1}_{k+1} = \sum_{r=k}^n G^n_r \quad\text{for all } k=0..n
		\end{split}\end{equation*}
		%
		\item $2\le N$の時 \\
		この時は、任意の$N\in\sizen_+$と$n\in\sizen$に対して
		$E_N^TS_n=\binom{n+N-1}{n}$より、$G^n_k$の漸化式は次のようになる。
		\begin{equation*}\begin{split}
			G^{n+1}_{k+1} 
			= \sum_{r=k}^n G^r_k \sum_{s=0}^{n-r} \plr{E_N^TS_s}G^{n-r}_s
			= \sum_{r=k}^n G^r_k \sum_{s=0}^{n-r} \binom{s+N-1}{s}G^{n-r}_s
		\end{split}\end{equation*}
		表\ref{table:Nが2の場合}と\ref{table:Nが3の場合}から、
		次の式が成り立っていることが予想される。
		\begin{equation*}\begin{split}
			G^{n+1}_{k+1} = \sum_{r=0}^{n-k} \binom{N-1+r}{r} G^n_{k+r} 
			\quad\text{for all } k=0..n
		\end{split}\end{equation*}

	\end{itemize} %}

	\begin{table}[htbp] %{
		\begin{center}\begin{tabular}{r|r|rrrrrrrrr} \hline
$n$ & 和 & $G^n_0$ & $G^n_1$ & $G^n_2$ & $G^n_3$ & $G^n_4$ & $G^n_5$ & $G^n_6$ & $G^n_7$ & $G^n_8$ \\\hline
0 & 1 & 1 \\
1 & 1 & 0 & 1 \\
2 & 2 & 0 & 1 & 1 \\
3 & 5 & 0 & 2 & 2 & 1 \\
4 & 14 & 0 & 5 & 5 & 3 & 1 \\
5 & 42 & 0 & 14 & 14 & 9 & 4 & 1 \\
6 & 132 & 0 & 42 & 42 & 28 & 14 & 5 & 1 \\
7 & 429 & 0 & 132 & 132 & 90 & 48 & 20 & 6 & 1 \\
8 & 1430 & 0 & 429 & 429 & 297 & 165 & 75 & 27 & 7 & 1 \\
\hline
		\end{tabular}\end{center}
		\caption{$N=1$}\label{table:Nが1の場合}
	\end{table} %}

	\begin{table}[htbp] %{
		\begin{center}\begin{tabular}{r|r|rrrrrrrrr} \hline
$n$ & 和 & $G^n_0$ & $G^n_1$ & $G^n_2$ & $G^n_3$ & $G^n_4$ & $G^n_5$ & $G^n_6$ & $G^n_7$ & $G^n_8$ \\\hline
0 & 1 & 1 \\
1 & 1 & 0 & 1 \\
2 & 3 & 0 & 2 & 1 \\
3 & 12 & 0 & 7 & 4 & 1 \\
4 & 55 & 0 & 30 & 18 & 6 & 1 \\
5 & 273 & 0 & 143 & 88 & 33 & 8 & 1 \\
6 & 1428 & 0 & 728 & 455 & 182 & 52 & 10 & 1 \\
7 & 7752 & 0 & 3876 & 2448 & 1020 & 320 & 75 & 12 & 1 \\
8 & 43263 & 0 & 21318 & 13566 & 5814 & 1938 & 510 & 102 & 14 & 1 \\
\hline
		\end{tabular}\end{center}
		\caption{$N=2$}\label{table:Nが2の場合}
	\end{table} %}

	\begin{table}[htbp] %{
		\begin{center}\begin{tabular}{r|r|rrrrrrrrr} \hline
$n$ & 和 & $G^n_0$ & $G^n_1$ & $G^n_2$ & $G^n_3$ & $G^n_4$ & $G^n_5$ & $G^n_6$ & $G^n_7$ & $G^n_8$ \\\hline
0 & 1 & 1 \\
1 & 1 & 0 & 1 \\
2 & 4 & 0 & 3 & 1 \\
3 & 22 & 0 & 15 & 6 & 1 \\
4 & 140 & 0 & 91 & 39 & 9 & 1 \\
5 & 969 & 0 & 612 & 272 & 72 & 12 & 1 \\
6 & 7084 & 0 & 4389 & 1995 & 570 & 114 & 15 & 1 \\
7 & 53820 & 0 & 32890 & 15180 & 4554 & 1012 & 165 & 18 & 1 \\
8 & 420732 & 0 & 254475 & 118755 & 36855 & 8775 & 1625 & 225 & 21 & 1 \\
\hline
		\end{tabular}\end{center}
		\caption{$N=3$}\label{table:Nが3の場合}
	\end{table} %}

%s2:多項式の摂動解}
%s1:ベキ等半環上の文法}
}\endgroup %}
