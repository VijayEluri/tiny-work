\section{Programming}
We use the following dot notations to denoet as the image of the map 
$f:A\times B\times C\times\cdots\to X$:
\begin{equation}\begin{split}
	f(a, b, c, \dots) &\Leftrightarrow (a.f)(b, c,\dots) \\
		&\Leftrightarrow a.f(b, c,\dots) \\
		&\Leftrightarrow \kakko{\kakko{a,b}.f}(c,\dots) \\
		&\Leftrightarrow \kakko{a,b}.f(c,\dots) \\
\end{split}\end{equation}
. This notation represents the map which is called curring in the programming
world. The curring is given by the followings:
\begin{equation}\begin{split}
	\myop{curry}: \set{X\times Y\to Z} &\simeq \set{X\to \set{Y\to Z}} \\
		f &\mapsto \myop{curry}f \text{ such that} \\
		& \kakko{\kakko{\myop{curry}f} x}y = f(x, y) \text{ for all }x\in X \text{ and } y\in Y \\
\end{split}\end{equation}
, and we denote $x.f:=(\myop{curry}f)x$.
Furthermore, when the map has unary operand $f:A\to B$,
we omit the braket as follows:
\begin{equation}\begin{split}
	f(a) &\Leftrightarrow fa \\
		&\Leftrightarrow a.f() \\
	f(a, b) &\Leftrightarrow a.f(b) \\
		&\Leftrightarrow a.fb \\
		&\Leftrightarrow (a,b).f()
\end{split}\end{equation}
. For example, the plus operation is written as follows:
\begin{equation}\begin{split}
	+(a, b) &\Leftrightarrow a .+ b \\
	+(a, b, c) &\Leftrightarrow a .+ b .+ c \\
\end{split}\end{equation}
. The dot notation is not standard mathmatical notation, 
but it is convenient to handle product.
For example, associativity of multiplication $m:X\times X\to X$ is written 
in the mid-position notation as the followings:
\begin{equation}\begin{split}
	(a.m b).m c &= a.m (b.m c) \\
\end{split}\end{equation}
. On the otherhand, samething can be written in the pre-position notation 
without referring elements explicitly as the followings:
\begin{equation}\begin{split}
	m(m\times \myid) &= m(m\times \myid) \\
\end{split}\end{equation}
. The dot notation is convention to translate between pre-position notation
and mid-postion notation with using same symbol.

We denote $\mybf{N}$ as a set of natural numbers $\set{0,1,2,\dots}$.

\subsection{Bag}
\begin{definition}[Bag]
Let $X$ is a finite set.
The set of the map $X\to \mybf{N}$ is called bag of $X$.
\end{definition}

Let $X$ is a finite set, $BX$ is a bag of $X$.
Let $+$ is the standard addition of $\mybf{N}$,
, $m$ be the standard multiplication of $\mybf{N}$.
We define the map $+_B$ and $m_B$ by the following commutative diagram:
\begin{equation}\xymatrix{
	X\times X \ar[r]^{(f, g)}
	& \mybf{N}\times \mybf{N} \ar[d]^{\square}
	\\
	X \ar@{.>}[r]^{\square_B(f, g)} \ar[u]^{\delta}
	& \mybf{N} 
	\\
}\end{equation}
, where $\square=\set{+,m}$ and $\delta$ is a co-multiplication $\delta x=(x,x)$. 

$+_B$ and $m_B$ are associative, because the following diagram is commutative
for all $f,g,h\in BX$:
\begin{equation}\xymatrix{
	& X \ar[r]^{\square_B\kakko{f,\square_B\kakko{g, h}}} \ar[d]_{\delta} 
	& \mybf{N}
	\\
	& X\times X \ar[r]^{\kakko{f,\square_B\kakko{g, h}}} \ar[d]_{(\myid,\delta)}
	& \mybf{N}\times \mybf{N} \ar[u]_{\square}
	\\
	X \ar@(u,l)[ruu]^{\myid} \ar@(d,l)[rdd]_{\myid}
	& X\times X\times X \ar[r]^{(f, g, h)}
	& \mybf{N}\times \mybf{N}\times \mybf{N} \ar[d]^{(\square, \myid)} \ar[u]_{(\myid, \square)}
	& \mybf{N} \ar@(u,r)[luu]_{\myid} \ar@(d,r)[ldd]^{\myid}
	\\
	& X\times X \ar[r]^{\kakko{\square_B\kakko{f, g},h}} \ar[u]^{(\delta,\myid)}
	& \mybf{N}\times \mybf{N} \ar[d]^{\square}
	\\
	& X \ar[r]^{\square_B\kakko{\square_B\kakko{f, g},h}} \ar[u]^{\delta}
	& \mybf{N} 
	\\
}\end{equation}
. This commutative diagram says the followings plain formula:
\begin{equation}\begin{split}
	(f.\square_Bg).\square h = f.\square_B(g.\square_Bh) \\
\end{split}\end{equation}
.

$+_B$ and $m_B$ is distributive, because the following diagram is commutative
for all $f,g,h\in BX$:
\begin{equation}\xymatrix@C+24pt{
	& X \ar[r]^{+_B\kakko{m_B\kakko{f,g}, m_B\kakko{f,h}}} \ar[d]_{\delta}
	& \mybf{N}
	\\
	& X\times X \ar[r]^{\kakko{m_B\kakko{f,g}, m_B\kakko{f,h}}} \ar[d]_{(\delta,\delta)}
	& \mybf{N}\times \mybf{N} \ar[u]_{+}
	\\
	& X^{\times4} \ar[r]^{\kakko{f,g,f,h}} \ar[d]_{(\myid,\sigma,\myid)}
	& \mybf{N}^{\times4} \ar[u]_{(m,m)}
	\\
	& X^{\times4} \ar[r]^{\kakko{f,f,g,h}} \ar[d]_{(\pi_1,\myid,\myid)}
	& \mybf{N}^{\times4} \ar[u]_{(\myid,\sigma,\myid)}
	\\
	X \ar@(u,l)[ruuuu]^{\myid} \ar@(d,l)[rdd]_{\myid}
	& X\times X\times X \ar[r]^{\kakko{f,g,h}}
	& \mybf{N}\times \mybf{N}\times \mybf{N} \ar[u]_{(\delta,\myid,\myid)} \ar[d]^{(\myid,+)}
	& \mybf{N} \ar@(u,r)[luuuu]_{\myid} \ar@(d,r)[ldd]^{\myid}
	\\
	& X\times X \ar[r]^{\kakko{f,+_B\kakko{g,h}}} \ar[u]^{(\myid,\delta)}
	& \mybf{N}\times \mybf{N} \ar[d]^{m}
	\\
	& X \ar[r]^{m_B\kakko{f,+_B\kakko{g,h}}} \ar[u]^{\delta}
	& \mybf{N}
	\\
}\end{equation}
. Where $\sigma$ is the swap, $\pi_1$ is the projection:
\begin{equation}\begin{split}
	\sigma: X_1\times X_2 &\to X_2\times X_1 \\
		(x_1,x_2) &\mapsto (x_2,x_1) \\
	\pi_1: X\times X_1 &\to X_2 \\
		(x_1,x_2) &\mapsto x_1 \\
\end{split}\end{equation}
. This commutative diagram says the followings plain formula:
\begin{equation}\begin{split}
	f.m_B(g.+_Bh) = (f.m_Bg)+_B(f.m_Bh) \\
\end{split}\end{equation}
.

The follwing maps $0_B\in BX$ and $1_B\in BX$ are identities of $+_B$ and $m_B$
, respectively:
\begin{equation}\begin{split}
	0_Bx &= 0 \text{ for all }x\in X \\
	1_Bx &= 1 \text{ for all }x\in X \\
\end{split}\end{equation}
.

The above argument can be summerized in the form of proposition.

\begin{proposition}[Bag is a semi-ring]
Let $X$ is a finite set, $BX$ is a bag of $X$.
And let $+_B$ and $m_B$ are induced maps $BX\times BX\to BX$ 
from $+$ and $m$ of $\mybf{N}$, respectively.
$BX=(BX,+_B,0_B,m_B,1_B)$ is a semi-ring.
\end{proposition}

Furthermore, the map $\kappa$:
\begin{equation}\begin{split}
	\kappa: \mybf{N} &\to BX \\
		k &\mapsto \kappa k \text{ such that} 
			(\kappa k)x = k \text{ for all }x\in X \\
\end{split}\end{equation}
is a ring-homeo.
$0_B$ and $1_B$ are given by $\kappa 0$ and $\kappa 1$, respectively.
Therefore $BX$ can be view as semi-module over $\myop{N}$ by the following 
identification:
\begin{equation}\begin{split}
	k.m_Bf := (\kappa k).m_Bf \text{ for all }k\in \myop{N} \text{ and }f\in BX
\end{split}\end{equation}
. We can summarize this identification and the previous proposition.

\begin{proposition}[Bag is a semi-algebra]
Let $X$ is a finite set, $BX$ is a bag of $X$.
And let $+_B$ and $m_B$ are induced maps $BX\times BX\to BX$ 
from the ring structure $+$ and $m$ of $\mybf{N}$, respectively.
$BX=(BX,+_B,0_B,m_B,1_B)$ is a semi-algebra over $\myop{N}$.
\end{proposition}

Sometimes, we only take acount the multiplication $m_B$ as the scalar 
multiplication $m_B:\myop{N}\times BX\to BX$.

\begin{proposition}[Bag is a semi-module]
Let $X$ is a finite set, $BX$ is a bag of $X$.
And let $+_B$ and $m_B$ are induced maps $BX\times BX\to BX$ 
from the ring structure $+$ and $m$ of $\mybf{N}$, respectively.
$BX=(BX,+_B,0_B,m_B)$ is a semi-module over $\myop{N}$, where
$m_B$ is a scalor muliplication.
\end{proposition}

Owing to $BX$ is a semi-module, the tensor product $BX\otimes BX$ can be 
defined uniquely. And owing to $X$ is finite, we can define dual basis
of $X$ in the $BX$. Let the map $\iota_B$ is the followings:
\begin{equation}\begin{split}
	\iota_B: X &\to BX \\
		x &\mapsto \iota_B x \text{ such that }(\iota_B x)y = \begin{cases}
			1 & \text{iff } x = y \\
			0 & \text{otherwise} \\
			\end{cases} \text{ for all } y\in X \\
\end{split}\end{equation}
. $BX$ is spanned by $\iota_B X$ as the followings:
\begin{equation}\begin{split}
	f = \sum_{x\in X}(fx).m_B(\iota_B x) \text{ for all }f\in BX \\
\end{split}\end{equation}
. Where summation is taken by $+_B$. 
$\set{X^{\times2}\to \mybf{N}^{\otimes2}}$ is spanned by $\iota_B X\otimes \iota_B X$ also as the followings:
\begin{equation}\begin{split}
	f &= \sum_{(x,y)\in X^{\times2}}\kakko{f\kakko{x,y}}.m_B\kakko{\kakko{\iota_B x}\otimes\kakko{\iota_B y}} \\
	& \quad\text{for all } f\in \set{X^{\times2}\to \mybf{N}^{\otimes2}}
\end{split}\end{equation}
. 
Then the relation between $BX\otimes BX$ and $\set{X^{\times2}\to\mybf{N}^{\otimes2}}$ 
is not $BX\otimes BX\subset\set{X^{\times2}\to\mybf{N}^{\otimes2}}$
but $BX\otimes BX=\set{X^{\times2}\to\mybf{N}^{\otimes2}}$.

\begin{proposition}[Bag has no more than tensor]
Let $X$ be a finite set, $BX=(BX,+_B,0_B)$ be a semi-module over $\myop{N}$.
The following isomorphism is satisfied:
\begin{equation}\begin{split}
	BX\otimes BX \simeq \set{X^{\times2}\to \mybf{N}^{\otimes2}} \\
\end{split}\end{equation}
. In other word, the following isomorphism is satisfied:
\begin{equation}\begin{split}
	\set{{X}\to \mybf{N}}^{\otimes2} \simeq \set{X^{\times2}\to \mybf{N}^{\otimes2}} \\
\end{split}\end{equation}
.
\end{proposition}

Let $FX=(FX,*,1_F)$ be a free-monoid of $X$, $BX=(BX,+_B,0_B)$ be a semi-module over $\myop{N}$.
Owing to the universality of free-module,
there exists a unique homeo $\varphi$ that satisfies the following commutative diagram:
\begin{equation}\xymatrix{
	X \ar[r]^{\iota_F} \ar[rd]^{\iota_B} & FX \ar@{.>}[d]^{\varphi} \\
	& BX \\
}\end{equation}
. Where $\iota_F$ is a singleton map $\iota_F x=\bakko{x}$.
Since $+_B$ is abelian, $\varphi(w_1w_2)=\varphi(w_2w_1)$ for all $w_1,w_2\in FX$. Thus we might can say that 'a set of symmetrized words is a bag'.

\subsection{Shuffle product}
Let $X$ is a set, $FX=(FX, m, 1_F)$ is a free-monoid of $X$.
We denote a word of $FX$ with brackets as $\bakko{abc}$ when using characters 
$\set{a,b,c}$ explicitly. The identity $1_F$ for $m$ is denoted also
$\bakko{}$. 
We denote $F_RX$ as a module over the ring $R$ with 

Let $R$ be a ring. 
We define the map $+$ as followings:
\begin{equation}\begin{split}
	+: FX\times  &\to FX \\
\end{split}\end{equation}

\subsection{Sorting}
Let $X$ be a set, $PX$ be a power-set of $X$, $FX$ be a free-monoid of $X$.
We define the map $\myop{asSet}$ as folloings:
\begin{equation}\begin{split}
	\myop{asSet}: FX &\to PX \\
		\bakko{x_1x_2\cdots x_n} &\mapsto \set{x_1}\cup\set{x_2}\cup\cdots\cup\set{x_n} \\
\end{split}\end{equation}
. The map $\myop{asSet}$ is a homeo w.r.t. string concatenation $*$ 
and set-union $\cup$:
\begin{equation}\xymatrix@C+20pt{
	FX\times FX \ar[r]^{\myop{asSet}\times \myop{asSet}} \ar[d]^{*} & PX\times PX \ar[d]^{\cup} \\
	FX \ar[r]^{\myop{asSet}} & PX \\
}\end{equation}
.
We denote $\exp(FX,PX)$ as the set of maps $f$ these satisfy the 
following commutative diagram:
\begin{equation}\xymatrix@C+20pt{
	PX\times PX \ar[r]^{f\times f} \ar[d]^{\cup} & FX\times FX \ar[d]^{*} \\
	PX \ar[r]^{f} & FX \\
}\end{equation}
. All $f\in\exp(FX,PX)$ is a homeo, 
because $(f0_P)*(f0_P)=f(0_P\cup0_P)$ impies $(f0_P)*(f0_P)=f0_P$
, then $f0_P=1_F$ owing to $FX$ be a free.

We define the product $P\mybf{N}\times<P\mybf{N}$ as follows:
\begin{equation}\begin{split}
	P\mybf{N}\times<P\mybf{N} &= \set{(a,b)\in P\mybf{N}\times P\mybf{N}\bou \cdots} \\
	\cdots &= x_a<x_b \text{ for all }x_a\in a \text{ and } x_b\in b \\
\end{split}\end{equation}
. Let consider the map $\varphi$ that satisfies the following commutativ diagram:
\begin{equation}\label{eq:order-product}\xymatrix{
	P\mybf{N}\times<P\mybf{N} \ar[r]^{\varphi\times\varphi} \ar[d]^{\cup} 
	& F\mybf{N}\times F\mybf{N} \ar[d]^{*} \\
	P\mybf{N} \ar[r]^{\varphi} & F\mybf{N} \\
}\end{equation}
. This commutative diagram implies $\varphi0_P=1_F$,
because $(\varphi0_P)*(\varphi0_P)=\varphi(0_P\cup0_P)$
impies $(\varphi0_P)*(\varphi0_P)=\varphi0_P$.
Then $\varphi0_P=1_F$, because $F\mybf{N}$ is a free.
Therefore the map $\varphi$ is determined satisfies this commutative diagram

The commutative diagram \eqref{eq:order-product} gives the essence of
quick-sort. Let $\varphi$ be the map satisfies
\begin{itemize}
\item the commutative diagram \eqref{eq:order-product} and
\item $\varphi\set{a}=\bakko{a}$ for all $a\in\mybf{N}$.
\end{itemize}
.
\begin{todo}
$\varphi$ is looked like $1:1$.
\end{todo}
and $\delta$ be the map of the followings:
\begin{equation}\begin{split}
	\delta: P\mybf{N} &\to P\mybf{N}\times<P\mybf{N} \\
	\cup\delta &= \myid \\
\end{split}\end{equation}
.

\subsection{Range representation of power set}
We denote $N_n$ as the set $\set{0, 1, \dots, n-1}$ through this section.
This representation of range is obeying the C-programming language convention.
We denote the following functors for the set $X$:
\begin{itemize}
	\item $PX$ as the power set of $X$
	\item $P_0X=PX-\set{\emptyset}$
	\item $FX$ as the free monoid set of $X$
	\item $F_0X=FX-\set{\text{idenitty element}}$
\end{itemize}
.

The map $\cup_R$ is defined by the following commutative diagram:
\begin{equation}\xymatrix{
	RN_n\times RN_n \ar[r]^{\myop{set}\times\myop{set}} \ar@{.>}[d]^{\cup_R}
		& PN_n\times PN_n \ar[d]^{\cup} \\
	RN_n \ar@{<-}[r]^{\myop{set}^{-1}} & PN_n \\
}\end{equation}
. Assume that $w_1,w_2\in RN_n$ are decomposed as the followings:
\begin{equation}\begin{split}
	w_1 &= w_{11} * w_{12} \\
	w_2 &= w_{21} * w_{22} \\
\end{split}\end{equation}
, and satisfy the following condition:
\begin{equation}\begin{split}
	\max(w_{11} \overline{\cup}_R w_{21}) < \min(w_{12} \overline{\cup}_R w_{22})
\end{split}\end{equation}
. Then the following equations are satisfied:
\begin{equation}\begin{split}
	w_1 \cup_R w_2 &= \kakko{w_{11} * w_{12}} \cup_R \kakko{w_{21} * w_{22}} \\
		&= \kakko{w_{11} \cup_R w_{21}} * \kakko{w_{12} \cup_R w_{22}} \\
\end{split}\end{equation}
. This deformation provides the way to calculate $\cup_R$.

\begin{equation}\begin{split}
\xymatrix{
	PN_{n+1} \ar@<1ex>[d]^{\pi}
		& FN_{n+1} \ar@<1ex>[l]^{\myop{set}} \ar@<1ex>[d]^{\pi} \\
	PN_n \ar@{.>}[r]^{\myop{set}^{-1}} \ar[ur]_{\myop{rep}} \ar@<1ex>[d]^{\pi} \ar[u]^{\kappa}
		& RN_n \ar@<1ex>[l]^{\myop{set}} \ar@<1ex>[d]^{\pi} \ar[u]^{\kappa} \\
	P^cN_n \ar[r]^{\myop{set}^{-1}} \ar[u]^{\kappa} 
		& R^cN_n \ar@<1ex>[l]^{\myop{set}} \ar[u]^{\kappa} \\
} &\quad \xymatrix{
	FN_{n+1}\times FN_{n+1} \ar@{<-}[r]^{\kappa\times \kappa} \ar[d]^{*} 
		& RN_n\times_R RN_n \ar@{.>}[d]^{*} \\
	FN_{n+1} \ar[r]^{\pi} & RN_n \\
}
\end{split}\end{equation}
\begin{equation}\begin{split}
	\pi\kappa &= \myop{id} \\
	\myop{set}\pi\myop{rep} &= \myop{id} \\
\end{split}\end{equation}
$\pi$ is a partial map.

\begin{equation}\begin{split}
	\min: PN_n-\set{\emptyset} &\to N_n \\
		p &\mapsto x \text{ such that } x\in p \text{ and } x\le y \text{ for all } y\in p \\
	\min: FN_n-\bakko{} &\to N_n \\
		w &\mapsto \min\myop{set} w \\
\end{split}\end{equation}

\subsection{backup}
We can define the map $\min/\max$ for the power set not including empty set $P_0N_n$ as usual.
We would like to extend $\min/\max$ to the power set including empty set $PN_n$.
The standard $\min$ might satisfy the following commutative diagram:
\begin{equation}\xymatrix@C+18pt{
	P_0N_n\times P_0N_n \ar[r]^{\min\times\min} \ar[d]^{-\cup-} & N_n\times N_n \ar[d]^{\min} \\
	P_0N_n \ar[r]^{\min} & N_n \\
}\end{equation}
, and same as for $\max$.
We extend $\min$ to the $PN_n$ with keeping the shape of this diagram.

\begin{todo}[Told a lie]
Let $\overline{N}_n = N_n\cup\set{-\infty, \infty}$ and define the following maps:
\begin{equation}\begin{split}
	\min: PN_n &\to \overline{N}_n \\
		p &\mapsto \begin{cases}
			-\infty &\text{iff }p = \emptyset \\
			x &\text{else} \\
		\end{cases} \\
		&\text{where } x\in p \text{ and } x\le y \text { for all } y\in p \\
	\max: PN_n &\to \overline{N}_n \\
		p &\mapsto \begin{cases}
			\infty &\text{iff }p = \emptyset \\
			x &\text{else} \\
		\end{cases} \\
		&\text{where } x\in p \text{ and } y\le x \text { for all } y\in p \\
\end{split}\end{equation}
. The following diagram be commutative with this definition of $\min$:
\begin{equation}\xymatrix@C+18pt{
	PN_n\times PN_n \ar[r]^{\min\times\min} \ar[d]^{-\cup-} & \overline{N}_n\times \overline{N}_n \ar[d]^{\min} \\
	PN_n \ar[r]^{\min} & \overline{N}_n \\
}\end{equation}
, and same as for $\max$.
The above statement is a lie, because
\begin{equation}\begin{split}
	\min(\set{}\cup\set{0}) = 0 \not= -\infty = \min(-\infty, 0)
\end{split}\end{equation}
.
\end{todo}

\subsection{backup}
We use the following symbols through this section:
\begin{equation}\begin{split}
	N_n &= \set{0, 1, \dots, n - 1} \\
	\overline{N}_n &= N_n \cup \set{\infty} \\
	W_n &= \text{free monoid on }\overline{N}_n \\ 
\end{split}\end{equation}
, and denote the power set of set $X$	by $PX$.

We define the following maps:
\begin{equation}\begin{split}
	\min_N: PN_n &\to \overline{N}_n \\
		p &\mapsto \begin{cases}
			\infty & \text{if } p = \emptyset \\
			x & \text{else} \\
			\end{cases} \\
		& \text{where} \\
		& x\in p \text{ and } x\le y \text{ for all } y \in p \\
	\max_N: PN_n &\to \overline{N}_n \\
		p &\mapsto \begin{cases}
			\infty & \text{if } p = \emptyset \\
			x & \text{else} \\
			\end{cases} \\
		& \text{where} \\
		& x\in p \text{ and } y\le x \text{ for all } y \in p \\
\end{split}\end{equation}
.

We shall denote a word $w\in W_n$ with characters surrounding by box bracket, like $\bakko{246\infty}$.

\begin{equation}\begin{split}
	\min: P_n\backslash\set{\emptyset} &\to N_n \\ 
		p &\mapsto x \text{ satisfies }x\le y \text{ for all }y\in p \\ 
	\max: P_n\backslash\set{\emptyset} &\to N_n \\ 
		p &\mapsto x \text{ satisfies }y\le x \text{ for all }y\in p \\ 
\end{split}\end{equation}
.

We want to represent $P_n$ with $W_n$ for data compression.
Bitarray is easest and direct way to represnt power set in program.
But bitarray requires a lot of memory, the length of array is $\zettai{S}/64$
to represent any subset of $S$.
For example, bitarray requires the length $2^{21}/64=2^{6}=2^{15}=32768$ 
of long-array to represent any subset of unicode-2.0.
Thus we shall consider the way to represent power set with range.

Let $R_n$ be the following set:
\begin{equation}\begin{split}
	R_n &= \set{w\in W_n\bou \zettai{w}\text{ is even and ordered from smaller to bigger}}
\end{split}\end{equation}
, and $R_n^\circ\subseteq R_n$ be the following set:
\begin{equation}\begin{split}
	R_n^\circ &= \set{w\in R_n\bou \zettai{w}=2}
\end{split}\end{equation}
. For example, $R_4$ is given by the followings:
\begin{equation}\begin{split}
	R_4 &= \set{\bakko{}
		, \bakko{01}, \bakko{02}, \bakko{03}, \bakko{12}, \bakko{13}, \bakko{23}
		, \bakko{0123}} \\
	R_4^\circ &= \set{\bakko{}
		, \bakko{01}, \bakko{02}, \bakko{03}, \bakko{12}, \bakko{13}, \bakko{23}
		} \\
\end{split}\end{equation}
. Note that is not closed under the 

\subsection{backup}
We define the following maps:
\begin{equation}\begin{split}
	\myop{begin}: R_n &\to N_n \\
		w &\mapsto \begin{cases}
			n & \text{iff } \zettai{w} = 0 \\
			\text{the first character of }w & \text{otherwise} \\
			\end{cases} \\
	\myop{end}: R_n &\to N_n \\
		w &\mapsto \begin{cases}
			n & \text{iff } \zettai{w} = 0 \\
			\text{the last character of }w & \text{otherwise} \\
			\end{cases} \\
	\myop{splitFirst}: R_n &\to R_n \otimes R_n \\
		w &\mapsto \begin{cases}
			\bakko{}\times \bakko{} & \text{iff } \zettai{w} = 0 \\
			w_1 \times w_2 & \text{otherwise} \\
			\end{cases} \\
			&\text{where } \zettai{w_1} = 2 \text{ and } w = w_1 * w_2 \\
	\myop{splitLast}: R_n &\to R_n \otimes R_n \\
		w &\mapsto \begin{cases}
			\bakko{}\times \bakko{} & \text{iff } \zettai{w} = 0 \\
			w_1 \times w_2 & \text{otherwise} \\
			\end{cases} \\
			&\text{where } \zettai{w_2} = 2 \text{ and } w = w_1 * w_2 \\
\end{split}\end{equation}
and define the map $\myop{asSet}$
\begin{equation}\begin{split}
	\myop{asSet}: R_{n+1} &\to P_n \\
		w &\mapsto \begin{cases}
			\emptyset & \text{if }\zettai{w} = 0 \\
			\set{x\in N_n\bou w.\myop{begin}() \le x < w.\myop{end}()} & \text{if }\zettai{w} = 2 \\
			w.\myop{splitFirst}().(\myop{asSet}()\times\myop{asSet}()).(-\cup-) &\text{otherwise} \\
		\end{cases}
\end{split}\end{equation}
. We assume without proof that $\myop{asSet}$ is isomorphism.
We define the map $\cup_R$ by the following commutative diagram:
\begin{equation}\begin{split}
	\xymatrix @C+12pt {
		R_{n+1}\times R_{n+1} \ar@{.>}[d]^{\cup_R}
			\ar[r]^{\myop{asSet}\times\myop{asSet}} 
			& P_n\times P_n \ar[d]^{\cup} \\
		R_{n+1} & P_n \ar[l]^{\myop{asSet}^{-1}} \\
	}
\end{split}\end{equation}
. We can show for all $r_1,r_2\in R_n^\circ$ the followings:
\begin{equation}\begin{split}
	r_1.\cup_Rr_2 &= \begin{cases}	
		r_1*r_2 &\text{if } r_1.\myop{end}() < r_2.\myop{begin}() \\
		r_2*r_1 &\text{if }r_2.\myop{end}() < r_1.\myop{begin}() \\
		r_1.\overline{\cup}_Rr_2 &\text{else} \\
	\end{cases} \\
\end{split}\end{equation}
, where
\begin{equation}\begin{split}
	\overline{\cup}_R: R_n \otimes R_n &\to R_n^\circ \\
		w_1 \times w_2 &\mapsto \bakko{ab} \text{ where} \\
	& a = \min(w_1.\myop{begin}(),w_2.\myop{begin}()) \\
	& b = \max(w_1.\myop{end}(),w_2.\myop{end}()) \\
\end{split}\end{equation}
. We can show for all $w\in R_n$ and $r\in R_n^\circ$ the followings:
\begin{equation}\begin{split}
	w.\cup_R r &= w_1 * (w_2.\overline{\cup}_R r) * w_3 \\
\end{split}\end{equation}
, where 
\begin{equation}\begin{split}
	w &= w_1 * w_2 * w_3 \\
	\text{and } & w_1.\myop{end}() < r.\myop{begin}() \\
	\text{and } & r.\myop{end}() < w_2.\myop{begin}() \text{ and} \\
	\text{and } & r.\myop{begin}() \le w_2.\myop{end}() 
	\text{ and } w_2.\myop{begin}() \le r.\myop{end}() \\
\end{split}\end{equation}
.
\begin{cprog}
w.cup = { [ab] |->
	(w0, w1) = w.splitFirst();
	if (w0.begin() == w0.end()) {
		return [ab];
	} else if (w0.end() < a) {
		return w0.* w1.cup [ab];
	} else if (b < w0.begin()) {
		return [ab] * w;
	} else {
	}
}
\end{cprog}

\subsection{backup}
\begin{equation}\begin{split}
	\bakko{ab}\vee\bakko{cd} &= \begin{cases}
		\bakko{abcd} & b < c \\
		\bakko{cdab} & d < a \\
		\bakko{(a,c).\min(),(b,d).\max()} & \text{else} \equiv a\le d \myop{and} d\le b  \\
		\end{cases}
\end{split}\end{equation}
\begin{equation}\begin{split}
	\bakko{ab}\vee w &= w_1 * u * w_2 \\
\end{split}\end{equation}
. Where
\begin{equation}\begin{split}
	w_1 &= w.\myop{subset}\set{x\bou x.\myop{end}() < a} \\
	w_2 &= w.\myop{subset}\set{x\bou b < x.\myop{begin}()} \\
	u &= \bakko{(a,ww).\min()(b,ww).\max()} \\
	ww &= w.\myop{subset}\set{x\bou a\le x.\myop{end}()\myop{and} x.\myop{begin}()\le b} \\
\end{split}\end{equation}
. Then we define the map $\myop{split}$ as the followings:
\begin{equation}\begin{split}
	\myop{split}: P_n^\circ\times P_n &\to P_n\times P_n\times P_n \\
		r\times p &\mapsto p_1\times q\times p_2 \\
		p_1 &= p.\myop{subset}\set{x\bou x.\myop{end}() < r.\myop{begin}()} \\
		p_2 &= p.\myop{subset}\set{x\bou r.\myop{end}() < x.\myop{begin}()} \\
		q &= p.\myop{subset}\set{x\bou r.\myop{begin}() \le x.\myop{end}() \myop{and} x.\myop{begin}() \le r.\myop{end}()} \\
\end{split}\end{equation}
\begin{equation}\begin{split}
	w_1\vee w_2 &= (r * w_{11})\vee w_2 \\
		&r \text{ is the first range of }w_1 \\
		&= (r * w_{11})\vee (w_{21} * u * w_2) \\
		&w_{21}\times u\times w_2 \text{ is the first range of }w_1 \\
\end{split}\end{equation}

\subsection{backup}

We view power set as semi-ring, 'or' as additive, 'and' as multiplicative
operations. A power set become a idempotent semi-ring in this view.

We shall first define the term 'connected' to represent range.



\begin{definition}[connected subset and disconnectable point]
Let $p$ is an element of $P_n$.
$p$ is called connected iff $p=\emptyset$ or $p=\set{x\in N_n\bou p.\min()\le x\le p.\max()}$.
$x\in N_n$ is called disconnectable point of $p$ 
iff $x$ is $p.\min()<x<p.\max()$ and $x\not\in p$. 
\end{definition}
. 

We shall define the following maps as convention:
\begin{equation}\begin{split}
	\myop{begin}: P_n &\to N_{n+1} \\ 
		p &\mapsto \begin{cases}
			n + 1 & \text{iff }p = \emptyset \\
			p.\min() & \text{else} \\
			\end{cases} \\
	\myop{end}: P_n &\to N_{n+1} \\ 
		p &\mapsto \begin{cases}
			n + 1 & \text{iff }p = \emptyset \\
			p.\max() + 1 & \text{else} \\
			\end{cases}
\end{split}\end{equation}
.

We denote $P_n^\circ\subseteq P_n$ as all connected subset of $N_n$.
$P_n^\circ$ is closed under the binary operation 'and'.

\subsection{backup}


\begin{definition}[pointwise split]
Let $p$ is an element of $P_n$ for some $n>0$.
The map $\myop{split}$ is defined as the followings:
\begin{equation}\begin{split}
	\myop{split}: N_n\times P_n &\to P_n\times P_n \\
		x\times p &\mapsto p_1\times p_2 \text{ such that} \\
			&\quad y < x \text{ for all }y\in p_1 \text{ and } \\
			&\quad x < y \text{ for all }y\in p_2 \\
\end{split}\end{equation}
\end{definition}

We define the map $\varphi$ as follows:
\begin{equation}\begin{split}
	\varphi: P_n &\to W_{n+1} \\
		p &\mapsto \begin{cases}
			1 & \text{if } p = \emptyset \\
			\bakko{\kakko{\myop{beg}p}\kakko{\myop{end}p}} & \text{else if } p \text{ is connected} \\
			m(\varphi\times\varphi)\myop{split}(x\times p) & \text{else} \\
			\end{cases} \\
\end{split}\end{equation}
, where $m$ is string concatenation, $x$ is a some disconnectable point 
of $p$. The map $\varphi$ converges for all element of $P_n$.

\begin{example}[$\varphi$]
\begin{equation}\begin{split}
	\varphi\set{2,3,4} = \bakko{24} \\
	\varphi\set{2,3,0} = \bakko{03} \\
\end{split}\end{equation}
\end{example}

The set $\varphi P_n\subset W_{n+1}$ is consisted with the following 
words.
\begin{itemize}
	\item all characters are ordered from smaller to bigger and
	\item there is not duplicated character and
	\item the number of characters is event.
\end{itemize}
We denote $R_n\subseteq W_n$ as a set of all words that satisfies the
followings:
\begin{itemize}
	\item all characters are ordered from smaller to bigger and
	\item there is not duplicated character.
\end{itemize}
, and define the map $\pi$ as the followings:
\begin{equation}\begin{split}
	\pi: W_n &\to R_n \\
		w &\mapsto \text{sorts and eliminates duplicated characters of }w \\
\end{split}\end{equation}
. 

\begin{example}[$\pi$]
\begin{equation}\begin{split}
	\pi\bakko{312} &= \bakko{123} \\
	\pi\bakko{2312} &= \bakko{123} \\
\end{split}\end{equation}
\end{example}

We denote $EW_n\subseteq W_n$ as the set of all even-length words over
$N_n$, and denote $ER_n\subseteq W_n$ same way.
The set $EW_n$ is closed under string concatenation.

We define the map $\mu$ as the followings:
\begin{equation}\begin{split}
	\mu: ER_{n+1} &\to P_n \\
		w &\mapsto \begin{cases}
			\emptyset & \text{if } \zettai{w} = 0 \\
			\set{x\in N_n\bou a\le x< b} & \text{else if } w \text{ is in the form } \bakko{ab} \\
			(\mu w_1) + (\mu w_2) & \text{else} \\
			\end{cases}
\end{split}\end{equation}
. Where $w_1$ and $w_2$ are defined as the followings:
\begin{equation}\begin{split}
	0 < \bakko{w_1} \text{ and } 0 < \bakko{w_1} \text{ and } w = w_1 * w_2
\end{split}\end{equation}
. We assume without proof that the following diagam is commute:
\begin{equation}\begin{split}
	\xymatrix {
		P_n \ar[r]^{\varphi} & W_{n+1} \ar[r]^{\pi} & R_{n+1} \ar@(d,d)[ll]^{\mu} 
	}
\end{split}\end{equation}
.
