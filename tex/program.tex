\section{Programming}

\subsection{Convension in this section}
We denote $\mybf{N}$ as a set of natural numbers $\set{0,1,2,\dots}$.
We denote $\mybf{B}$ as a set $\set{0,1}$ with boolean algebra.

We use the following dot notations to denoet as the image of the map 
$f:A\times B\times C\times\cdots\to X$:
\begin{equation}\begin{split}
	f(a, b, c, \dots) &\Leftrightarrow (a.f)(b, c,\dots) \\
		&\Leftrightarrow a.f(b, c,\dots) \\
		&\Leftrightarrow \kakko{\kakko{a,b}.f}(c,\dots) \\
		&\Leftrightarrow \kakko{a,b}.f(c,\dots) \\
\end{split}\end{equation}
. This notation represents the map which is called curring in the programming
world. The curring is given by the followings:
\begin{equation}\begin{split}
	\myop{curry}: \set{X\times Y\to Z} &\simeq \set{X\to \set{Y\to Z}} \\
		f &\mapsto \myop{curry}f \text{ such that} \\
		& \kakko{\kakko{\myop{curry}f} x}y = f(x, y) \text{ for all }x\in X \text{ and } y\in Y \\
\end{split}\end{equation}
, and we denote $x.f:=(\myop{curry}f)x$.
Furthermore, when the map has unary operand $f:A\to B$,
we omit the braket as follows:
\begin{equation}\begin{split}
	f(a) &\Leftrightarrow fa \\
		&\Leftrightarrow a.f() \\
	f(a, b) &\Leftrightarrow a.f(b) \\
		&\Leftrightarrow a.fb \\
		&\Leftrightarrow (a,b).f()
\end{split}\end{equation}
. For example, the plus operation is written as follows:
\begin{equation}\begin{split}
	+(a, b) &\Leftrightarrow a .+ b \\
	+(a, b, c) &\Leftrightarrow a .+ b .+ c \\
\end{split}\end{equation}
. The dot notation is not standard mathmatical notation, 
but it is convenient to handle product.
For example, associativity of multiplication $m:X\times X\to X$ is written 
in the mid-position notation as the followings:
\begin{equation}\begin{split}
	(a.m b).m c &= a.m (b.m c) \\
\end{split}\end{equation}
. On the otherhand, associativity can be written in the pre-position notation 
without referring elements explicitly as the followings:
\begin{equation}\begin{split}
	m(m\times \myid) &= m(m\times \myid) \\
\end{split}\end{equation}
. The dot notation is convention to translate between pre-position notation
and mid-postion notation with using same symbol.

\subsection{Convolution}
\newcommand{\rmapr}{M\mybf{R}}
\newcommand{\loner}{L\mybf{R}}
\newcommand{\intallr}[1]{\int_{{#1}\in\mybf{R}}}
Let $\rmapr$ be a set of all maps from $\mybf{R}$ to $\mybf{R}$.
We can define the structure of ring on $\rmapr$ from the target space 
ring structure $(\mybf{R},+,0,*,1)$.
\begin{equation}\begin{split}
	(f.+g)x &= (fx).+ (gx) \\
	(f.*g)x &= (fx).* (gx) \\
\end{split}\end{equation}
We denote $\loner\subset\rmapr$ as the space of the followings:
\begin{equation}\begin{split}
	\loner &= \set{f\in\rmapr\bou \int_{x\in\mybf{R}}\zettai{fx} < \infty}
\end{split}\end{equation}
. 
The $\loner$ is closed under $+$ and $*$.
\begin{equation}\begin{split}
	\intallr{x}(f.+g)x &= \kakko{\intallr{x}fx}.+ \kakko{\intallr{x}gx} \\
		&\le \kakko{\intallr{x}\zettai{fx}}.+ \kakko{\intallr{x}\zettai{gx}} \\
		& < \infty \\
	\intallr{x}(f.*g)x &= \kakko{\intallr{x}\kakko{fx}\kakko{gx}} \\
		&\le \kakko{\intallr{x}\zettai{fx}}.* \kakko{\intallr{x}\zettai{gx}} \\
		& < \infty \\
\end{split}\end{equation}
Where we use the following inequality to derive the second inequality.
\begin{equation}\begin{split}
	(a.*b).+(c.*d) 
		&\le \zettai{a.*b}.+\zettai{c.*d} \\
		&\le \kakko{\zettai{a}.+\zettai{c}}.*\kakko{\zettai{b}.+\zettai{d}} \\
		&\text{for all }a,b,c,d\in\mybf{R} \\
\end{split}\end{equation}
Note that all non-zero constant maps are not elements of $\loner$,
especially $1\not\in\loner$. 
Therefore, $*$ dose not have identity $1$ in the $\loner$.

We define another multiplication beside $*$.
\begin{definition}[Convolution]
We define the map $\sqcap$ as the followings:
\begin{equation}\begin{split}
	\sqcap: \rmapr\times\rmapr &\to \rmapr \\
	(f,g) &\mapsto (f.\sqcap g) \text{ such that} \\
	& (f.\sqcap g)x = \int_{y\in\mybf{R}}\kakko{f\kakko{x.-y}}\kakko{gy} \\
\end{split}\end{equation}
. The map $\sqcap$ is called convolution.
\end{definition}
The convolution satisfies the folllowing properties:
\begin{itemize}
\item distritutive $f.\sqcap(g.+ h) = (f.\sqcap g).+ (f.\sqcap h)$
\item associative $(f.\sqcap g).\sqcap h = f.\sqcap(g.\sqcap h)$
	\begin{proof}
		by changing variable of integration.
		\begin{equation*}\begin{split}
			\intallr{z}\kakko{\kakko{f.\sqcup g}\kakko{x.-z}}.*\kakko{hz}
				&= \intallr{y}\intallr{z}\kakko{f\kakko{x.-y.-z}}.*\kakko{gy}.*\kakko{hz} \\
				&= \intallr{y}\intallr{z}\kakko{f\kakko{x.-y}}.*\kakko{g\kakko{y.-z}}.*\kakko{hz} \\
				&= \intallr{y}\kakko{f\kakko{x.-y}}.*\kakko{\kakko{g\sqcup h}y} \\
		\end{split}\end{equation*}
	\end{proof}
\item commutative $f\sqcap g = g\sqcap f$
	\begin{proof}
		by changing variable of integration.
		\begin{equation*}\begin{split}
			\intallr{y}\kakko{f\kakko{x.-y}}.*\kakko{gy}
			&= \intallr{y}\kakko{fy}.*\kakko{g\kakko{x.-y}} \\
		\end{split}\end{equation*}
	\end{proof}
\end{itemize}
.
The $\loner$ is closed also under $\sqcup$, and $\sqcup$ also dose not
have identity in the $\loner$.

\subsection{Words}
\begin{definition}[Free-monoid]
Let $X$ be a set, $FX$ be set of all the finite sequences of elements of $X$.
Note that $FX$ includes empty sequence. $FX$ is called free-monoid on $X$.
An element of $FX$ is called word of $X$.
The operation of concatenation of 2 words is associative and has the identiy
element empty-word.
\end{definition}

We denote $\bakko{x_1x_2\cdots x_n}$ sorrounding with bracket as an element of
$FX$ that is a sequence of $(x_1,x_2,\dots,x_n)\in X^{\times n}$.

\subsection{Shuffle product}
Let $X$ be a set, $FX=(FX, m, \braket{})$ be a free-monoid on X.
Let $F_{\myop{B}}X$ be a free-module over the boolean $B$ on $FX$.

\subsection{Bag}
\begin{definition}[Bag]
Let $X$ be a finite set.
The set of the map $X\to \mybf{N}$ is called bag of $X$.
\end{definition}

Let $X$ is a finite set, $BX$ is a bag of $X$.
Let $+$ is the standard addition of $\mybf{N}$,
, $m$ be the standard multiplication of $\mybf{N}$.
We define the map $+_B$ and $m_B$ by the following commutative diagram:
\begin{equation}\xymatrix{
	X\times X \ar[r]^{(f, g)}
	& \mybf{N}\times \mybf{N} \ar[d]^{\square}
	\\
	X \ar@{.>}[r]^{\square_B(f, g)} \ar[u]^{\delta}
	& \mybf{N} 
	\\
}\end{equation}
, where $\square=\set{+,m}$ and $\delta$ is a co-multiplication $\delta x=(x,x)$. 

$+_B$ and $m_B$ are associative, because the following diagram is commutative
for all $f,g,h\in BX$:
\begin{equation}\xymatrix{
	& X \ar[r]^{\square_B\kakko{f,\square_B\kakko{g, h}}} \ar[d]_{\delta} 
	& \mybf{N}
	\\
	& X\times X \ar[r]^{\kakko{f,\square_B\kakko{g, h}}} \ar[d]_{(\myid,\delta)}
	& \mybf{N}\times \mybf{N} \ar[u]_{\square}
	\\
	X \ar@(u,l)[ruu]^{\myid} \ar@(d,l)[rdd]_{\myid}
	& X\times X\times X \ar[r]^{(f, g, h)}
	& \mybf{N}\times \mybf{N}\times \mybf{N} \ar[d]^{(\square, \myid)} \ar[u]_{(\myid, \square)}
	& \mybf{N} \ar@(u,r)[luu]_{\myid} \ar@(d,r)[ldd]^{\myid}
	\\
	& X\times X \ar[r]^{\kakko{\square_B\kakko{f, g},h}} \ar[u]^{(\delta,\myid)}
	& \mybf{N}\times \mybf{N} \ar[d]^{\square}
	\\
	& X \ar[r]^{\square_B\kakko{\square_B\kakko{f, g},h}} \ar[u]^{\delta}
	& \mybf{N} 
	\\
}\end{equation}
. This commutative diagram says the followings plain formula:
\begin{equation}\begin{split}
	(f.\square_Bg).\square h = f.\square_B(g.\square_Bh) \\
\end{split}\end{equation}
.

$+_B$ and $m_B$ is distributive, because the following diagram is commutative
for all $f,g,h\in BX$:
\begin{equation}\xymatrix@C+24pt{
	& X \ar[r]^{+_B\kakko{m_B\kakko{f,g}, m_B\kakko{f,h}}} \ar[d]_{\delta}
	& \mybf{N}
	\\
	& X\times X \ar[r]^{\kakko{m_B\kakko{f,g}, m_B\kakko{f,h}}} \ar[d]_{(\delta,\delta)}
	& \mybf{N}\times \mybf{N} \ar[u]_{+}
	\\
	& X^{\times4} \ar[r]^{\kakko{f,g,f,h}} \ar[d]_{(\myid,\sigma,\myid)}
	& \mybf{N}^{\times4} \ar[u]_{(m,m)}
	\\
	& X^{\times4} \ar[r]^{\kakko{f,f,g,h}} \ar[d]_{(\pi_1,\myid,\myid)}
	& \mybf{N}^{\times4} \ar[u]_{(\myid,\sigma,\myid)}
	\\
	X \ar@(u,l)[ruuuu]^{\myid} \ar@(d,l)[rdd]_{\myid}
	& X\times X\times X \ar[r]^{\kakko{f,g,h}}
	& \mybf{N}\times \mybf{N}\times \mybf{N} \ar[u]_{(\delta,\myid,\myid)} \ar[d]^{(\myid,+)}
	& \mybf{N} \ar@(u,r)[luuuu]_{\myid} \ar@(d,r)[ldd]^{\myid}
	\\
	& X\times X \ar[r]^{\kakko{f,+_B\kakko{g,h}}} \ar[u]^{(\myid,\delta)}
	& \mybf{N}\times \mybf{N} \ar[d]^{m}
	\\
	& X \ar[r]^{m_B\kakko{f,+_B\kakko{g,h}}} \ar[u]^{\delta}
	& \mybf{N}
	\\
}\end{equation}
. Where $\sigma$ is the swap, $\pi_1$ is the projection:
\begin{equation}\begin{split}
	\sigma: X_1\times X_2 &\to X_2\times X_1 \\
		(x_1,x_2) &\mapsto (x_2,x_1) \\
	\pi_1: X\times X_1 &\to X_2 \\
		(x_1,x_2) &\mapsto x_1 \\
\end{split}\end{equation}
. This commutative diagram says the followings plain formula:
\begin{equation}\begin{split}
	f.m_B(g.+_Bh) = (f.m_Bg)+_B(f.m_Bh) \\
\end{split}\end{equation}
.

The follwing maps $0_B\in BX$ and $1_B\in BX$ are identities of $+_B$ and $m_B$
, respectively:
\begin{equation}\begin{split}
	0_Bx &= 0 \text{ for all }x\in X \\
	1_Bx &= 1 \text{ for all }x\in X \\
\end{split}\end{equation}
.

The above argument can be summerized in the form of proposition.

\begin{proposition}[Bag is a semi-ring]
Let $X$ is a finite set, $BX$ is a bag of $X$.
And let $+_B$ and $m_B$ are induced maps $BX\times BX\to BX$ 
from $+$ and $m$ of $\mybf{N}$, respectively.
$BX=(BX,+_B,0_B,m_B,1_B)$ is a semi-ring.
\end{proposition}

Furthermore, the map $\kappa$:
\begin{equation}\begin{split}
	\kappa: \mybf{N} &\to BX \\
		k &\mapsto \kappa k \text{ such that} 
			(\kappa k)x = k \text{ for all }x\in X \\
\end{split}\end{equation}
is a ring-homeo.
$0_B$ and $1_B$ are given by $\kappa 0$ and $\kappa 1$, respectively.
Therefore $BX$ can be view as semi-module over $\myop{N}$ by the following 
identification:
\begin{equation}\begin{split}
	k.m_Bf := (\kappa k).m_Bf \text{ for all }k\in \myop{N} \text{ and }f\in BX
\end{split}\end{equation}
. We can summarize this identification and the previous proposition.

\begin{proposition}[Bag is a semi-algebra]
Let $X$ is a finite set, $BX$ is a bag of $X$.
And let $+_B$ and $m_B$ are induced maps $BX\times BX\to BX$ 
from the ring structure $+$ and $m$ of $\mybf{N}$, respectively.
$BX=(BX,+_B,0_B,m_B,1_B)$ is a semi-algebra over $\myop{N}$.
\end{proposition}

Sometimes, we only take acount the multiplication $m_B$ as the scalar 
multiplication $m_B:\myop{N}\times BX\to BX$.

\begin{proposition}[Bag is a semi-module]
Let $X$ is a finite set, $BX$ is a bag of $X$.
And let $+_B$ and $m_B$ are induced maps $BX\times BX\to BX$ 
from the ring structure $+$ and $m$ of $\mybf{N}$, respectively.
$BX=(BX,+_B,0_B,m_B)$ is a semi-module over $\myop{N}$, where
$m_B$ is a scalor muliplication.
\end{proposition}

Owing to $BX$ is a semi-module, the tensor product $BX\otimes BX$ can be 
defined uniquely. And owing to $X$ is finite, we can define dual basis
of $X$ in the $BX$. Let the map $\iota_B$ is the followings:
\begin{equation}\begin{split}
	\iota_B: X &\to BX \\
		x &\mapsto \iota_B x \text{ such that }(\iota_B x)y = \begin{cases}
			1 & \text{iff } x = y \\
			0 & \text{otherwise} \\
			\end{cases} \text{ for all } y\in X \\
\end{split}\end{equation}
. $BX$ is spanned by $\iota_B X$ as the followings:
\begin{equation}\begin{split}
	f = \sum_{x\in X}(fx).m_B(\iota_B x) \text{ for all }f\in BX \\
\end{split}\end{equation}
. Where summation is taken by $+_B$. 
$\set{X^{\times2}\to \mybf{N}^{\otimes2}}$ is spanned by $\iota_B X\otimes \iota_B X$ also as the followings:
\begin{equation}\begin{split}
	f &= \sum_{(x,y)\in X^{\times2}}\kakko{f\kakko{x,y}}.m_B\kakko{\kakko{\iota_B x}\otimes\kakko{\iota_B y}} \\
	& \quad\text{for all } f\in \set{X^{\times2}\to \mybf{N}^{\otimes2}}
\end{split}\end{equation}
. 
Then the relation between $BX\otimes BX$ and $\set{X^{\times2}\to\mybf{N}^{\otimes2}}$ 
is not $BX\otimes BX\subset\set{X^{\times2}\to\mybf{N}^{\otimes2}}$
but $BX\otimes BX=\set{X^{\times2}\to\mybf{N}^{\otimes2}}$.

\begin{proposition}[Bag has no more than tensor]
Let $X$ be a finite set, $BX=(BX,+_B,0_B)$ be a semi-module over $\myop{N}$.
The following isomorphism is satisfied:
\begin{equation}\begin{split}
	BX\otimes BX \simeq \set{X^{\times2}\to \mybf{N}^{\otimes2}} \\
\end{split}\end{equation}
. In other word, the following isomorphism is satisfied:
\begin{equation}\begin{split}
	\set{{X}\to \mybf{N}}^{\otimes2} \simeq \set{X^{\times2}\to \mybf{N}^{\otimes2}} \\
\end{split}\end{equation}
.
\end{proposition}

Let $FX=(FX,*,1_F)$ be a free-monoid of $X$, $BX=(BX,+_B,0_B)$ be a semi-module over $\myop{N}$.
Owing to the universality of free-module,
there exists a unique homeo $\varphi$ that satisfies the following commutative diagram:
\begin{equation}\xymatrix{
	X \ar[r]^{\iota_F} \ar[rd]^{\iota_B} & FX \ar@{.>}[d]^{\varphi} \\
	& BX \\
}\end{equation}
. Where $\iota_F$ is a singleton map $\iota_F x=\bakko{x}$.
Since $+_B$ is abelian, $\varphi(w_1w_2)=\varphi(w_2w_1)$ for all $w_1,w_2\in FX$. Thus we might can say that 'a set of symmetrized words is a bag'.
