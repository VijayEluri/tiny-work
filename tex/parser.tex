\section{余半群からモノイドへの写像}\label{s1:集合からモノイドへの写像} %{ 
	$A=(A,\Delta_A)$を余半群、$B=(B,m_B,1_B)$をモノイド、$B^A$を$A$から$B$への
	写像全体とする。$B^A$に二項演算$m_B^*$を次の可換図が成り立つように定義する。
	\begin{equation}\xymatrix{ %{
		A \times A \ar[d]^{f \times g} & A \ar[l]_{\Delta_A} \ar@{.>}[d]^{m_B^*(f \times g)} \\
		B \times B \ar[r]^{m_B} & B \\
	}\end{equation} %}
	すると、$\Delta_A$と$m_B$の結合性から、任意の$f,g,h\in B^A$に対して、
	次の可換図が成り立ち、二項演算$m_B^*$は結合的になることがわかる。
	\begin{equation}\label{eq:誘導された二項演算の結合性}\xymatrix@C+1ex{
		& A \ar[d]_{\Delta_A} \ar[r]^{m_B^*(m_B^*(f\times g)\times h)}
		& B \ar@(r,u)[rdd]^{\myid} 
		\\
		& A\times A \ar[d]_{\Delta_A\times \myid} \ar[r]^{m_B^*(f\times g)\times h}
		& B\times B \ar[u]^{m_B}
		\\
	A \ar@(u,l)[ruu]^{\myid} \ar@(d,l)[rdd]_{\myid} 
		& A\times A\times A \ar[r]^{f\times g\times h}
		& B\times B\times B \ar[u]_{m_B\times \myid} \ar[d]^{\myid\times m_B}
		& B 
		\\
		& A\times A \ar[u]^{\myid\times \Delta_A} \ar[r]^{f\times m_B^*(g\times h)}
		& B\times B \ar[d]^{m_B}
		\\
		& A \ar[u]^{\Delta_A} \ar[r]^{m_B^*(f\times m_B^*(g\times h))}
		& B \ar@(r,d)[ruu]_{\myid}
		\\
	}\end{equation}
	よって、二項演算$m_B^*$は積となる。
	
	定数写像を埋め込み$i_B$によって定義する。
	\begin{equation}\label{eq:constant-map}\begin{split} %{
		i_B: B &\to B^A \\
			b &\mapsto i_Bb \text{ such that } (i_Bb)a = b \text{ for all }a\in A \\
	\end{split}\end{equation} %}
	埋め込み$i_B$は半群同型になり、$i_B1_B$は積$m_B^*$の単位元となる。
	以上より、$(B^A,m_B^*,i_B1_B)$はモノイドとなる。
%s1:集合からモノイドへの写像}

\section{集合から半環への写像}\label{s1:集合から半環への写像} %{ 
	$A$を集合、$B=(B,+,0_B,*,1_B)$を可換半環、$B^A$を$A$から$B$への写像全体
	とする。二項演算$+$と$*$は中置記法で書くことにする。$B^A$に二項演算
	$+$と$*$を、任意の$f,g\in B^A$と任意の$a\in A$に対して、次の式で定める。
	\begin{equation}\begin{split} %{
		(f+g)a &= (fa)+(ga) \\
		(f*g)a &= (fa)*(ga) \\
	\end{split}\end{equation} %}
	定数写像を埋め込み$i_B$によって定義する。
	\begin{equation}\label{eq:constant-map}\begin{split} %{
		i_B: B &\to B^A \\
			b &\mapsto i_Bb \text{ such that } (i_Bb)a = b \text{ for all }a\in A \\
	\end{split}\end{equation} %}
	$i_B0_B$が$+$の単位元、$i_B1_B$が$*$の単位元となる。
	さらに、分配性
	\begin{equation*}\begin{split} %{
		f*(g+h) &= (f*g)+(f*h) \\
		(f+g)*h &= (f*h)+(g*h) \\
	\end{split}\end{equation*} %}
	とゼロ性
	\begin{equation*}\begin{split} %{
		(i_B0_B)*f = i_B0_B = f*(i_B0_B)
	\end{split}\end{equation*} %}
	が成り立つから$(B^A,+,i_B0_B,*,i_B1_B)$は半環になる。

	前節のように、$A$の任意の余積から$B^A$に積$+,*$を定義した場合は、
	分配性が保障されない。ここでは、群的な余積$a\mapsto a\times a$
	を用いて分配性を保障している。

	埋め込み$i_A$を次のように定義する。
	\begin{equation}\label{eq:constant-map}\begin{split} %{
		i_A: A &\to B^A \\
			a &\mapsto i_Aa \text{ such that } (i_Aa)a_1 = \begin{cases}
				i_B1_B, &\text{ iff }a=a_1 \\
				i_B0_B, &\text{ otherwise } \\
			\end{cases}
	\end{split}\end{equation} %}
	$0_B=1_B$でない限り、$i_A$は集合同型となる。
	任意の$f\in B^A$は次のように書くことができる。
	\begin{equation}\begin{split} %{
		f &= \sum_{a\in A}(fa)*(i_Aa)
	\end{split}\end{equation} %}
	\footnote {
		特に、$i_B1_B=sum_{a\in A}(i_Aa)$となる。
	}
	したがって、$\set{i_Bb}_{b\in B}$と$B$を同一視することで、$B^A$は
	\begin{itemize}
		\item スカラー積が$*$で定義され、
		\item $\set{i_Aa}_{a\in A}$を基底とする
	\end{itemize}
	$B$係数の半モジュールとしてみることができる。さらに、$*$によって積
	\begin{equation}\begin{split} %{
		(i_Aa_1) * (i_Aa_2) = \begin{cases}
			i_Aa_2, &\text{ iff }a_1 = a_2 \\
			i_B0_B, &\text{ otherwise } \\
		\end{cases}
	\end{split}\end{equation} %}
	が定義されるから、$B^A$は$*$を積とする半代数としてみることもできる。
%s1:集合から半環への写像}
