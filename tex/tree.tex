\section{木} %{
	この節で用いる約束を挙げておく。この節では、これらの約束を断りなく使う。
	\section{文字列}\label{s1:文字列} %{
	$R=(R,+,0,\myspace,1)$を可換半環、$A$を有限集合、$WA=(WA,m_*,1_W)$を$A$
	から生成された自由モノイド、$RWA$を$WA$を基底とする$R$係数自由半
	モジュールとする。ここで、積$m_*$は文字列の連結で定義された積とし、
	中置記法で$*$とも書くことにする。また、$1_W$は文字数$0$の単語とする。
	$WA$の元を$A$の元を並べたものを括弧でくくって表すことにする。
	例えば、$a_1,a_2,\dots, a_n\in A$を並べた$WA$の元を$[a_1a_2\cdots a_n]$
	と書く。
	
	任意の$n\in\mybf{N}$に対して$W_nA\subseteq WA$を文字数$n$の単語の集合と
	する。つまり、$WA=\oplus_{n\in N}W_nA$となる。$RW_nA$を$W_nA$を基底とする
	$R$係数自由半モジュールとする。やはり、$RWA=\oplus_{n\in N}RW_nA$となる。
	積$m_*$は$R$双線形写像$m_*:RW_mA\otimes RW_nA\to RW_{m+n}A$として
	みることもできる。

	ケース文を'簡潔に書くために、デルタ関数'$\jump{\mathchar`-}$を定義して
	おく。論理値$\mybf{B}=\set{0_{\mybf{B}},1_{\mybf{B}}}$から半環
	$R=(R,+,0_R,\myspace,1_R)$への写像$\jump{\mathchar`-}$を次のように
	定義する。
	\begin{equation*}\begin{split} %{
		\jump{\mathchar`-}: \mybf{B} &\mapsto R \\
		0_{\mybf{B}} &\mapsto 0_R \\
		1_{\mybf{B}} &\mapsto 1_R \\
	\end{split}\end{equation*} %}

	$RWA$を$WA$を基底とする$R$係数半モジュールとする。$WA$の積$m_*$を$R$線形
	に拡張して$RWA$の積としたものを同じ記号$m_*$で書き、中置記法で$*$とも
	書く。さらに、中置記法$*$をテンソル積に対して次のように定義する。
	\begin{equation}\begin{split} %{
		&(w_{11}\otimes w_{12}\otimes\cdots\otimes w_{1m})
		*(w_{21}\otimes w_{22}\otimes\cdots\otimes w_{2m}) \\
		&\quad= (w_{11}*w_{21})\otimes (w_{12}*w_{22})\otimes\cdots\otimes (w_{1m}*w_{2m}) \\
		&\quad\text{for all }w_{11},w_{12},\dots,w_{1m},w_{21},w_{22},\dots,w_{2m}\in WA
	\end{split}\end{equation} %}
	積$m_*$に双対で、任意の$a\in A$に対して
	$\Delta_*[a]=[a]\otimes 1_W+1_W\otimes [a]$となる余積$\Delta_*$を求める。
	任意の$a_1,a_2,\dots,a_m\in A$に対して次の式が成り立つ必要がある。
	\begin{equation}\begin{split} %{
		\Delta_*[a_1a_2\cdots a_m] &= (\Delta_*[a_1])*(\Delta_*[a_2])*\cdots*(\Delta_*[a_m]) \\
		&= [a_1a_2\cdots a_m]\otimes 1_W \\
		&\; + \sum_{1\le i\le n}[a_1a_2\cdots a_m]_{\neg{\set{i}}}\otimes [a_i] \\
		&\; + \sum_{1\le i<j\le n}[a_1a_2\cdots a_m]_{\neg{\set{i,j}}}\otimes [a_ia_j] \\
		&\; + \cdots \\
		&\; + 1_W\otimes [a_1a_2\cdots a_m] \\
	\end{split}\end{equation} %}
	ここで、任意の$1\le i_1<i_2<i_n\le m$に対して
	$[a_1a_2\cdots a_m]_{\neg\set{i_1,i_2,\dots,i_n}}$を$[a_1a_2\cdots a_m]$
	から$i_1$番目と$i_2$番目と...と$i_n$番目の文字を取り除いた文字列とした。
	例えば、$[abc]_{\neg\set{1}}=[bc]$、$[abc]_{\neg\set{2}}=[ac]$、
	$[abc]_{\neg\set{1,3}}=[b]$となる。更に、余単位射を
	$\epsilon_*:w\mapsto \jump{w=1_W}$で定めると、単位元$1_W$に対する余積が
	$\Delta_*1_W=1_W\otimes 1_W+\cdots$という形になる必要がある。
	一方、双対性$\Delta_*[a]=(\Delta_*1_W)*(\Delta_*[a])$を満たすためには、
	$\Delta_*1_W=1_W\otimes 1_W$となる必要があることがわかる。まとめると、
	次のようになる。

	\begin{definition}[文字列の連結に双対な余積]\label{def:文字列の連結に双対な余積} %{
		$R$線形写像$\Delta_*$を次のように定義する。
		\begin{equation}\begin{split} %{
			\Delta_*: RWA\otimes RWA &\to RWA \\
			1_W &\mapsto 1_W\otimes 1_W \\
			[a_1a_2\cdots a_m] &\mapsto (\Delta_*[a_1])*(\Delta_*[a_2])*\cdots*(\Delta_*[a_m]) \\
			&= [a_1a_2\cdots a_m]\otimes 1_W \\
			&\; + \sum_{1\le i\le n}[a_1a_2\cdots a_m]_{\neg{\set{i}}}\otimes [a_i] \\
			&\; + \sum_{1\le i<j\le n}[a_1a_2\cdots a_m]_{\neg{\set{i,j}}}\otimes [a_ia_j] \\
			&\; + \cdots \\
			&\; + 1_W\otimes [a_1a_2\cdots a_m] \\
			&\quad\text{for all }a_1,a_2,\dots,a_m \in A
		\end{split}\end{equation} %}
		$\Delta_*$は積$m_*$に双対になる。
		また、次の$R$線形写像$\epsilon_*$は余積$\Delta_*$の余単位射となる。
		\begin{equation}\begin{split} %{
			\epsilon_*: RWA &\to R \\
				w &\mapsto \jump{w=1_W}\quad\text{for all }w\in WA
		\end{split}\end{equation} %}
	\end{definition} %def:文字列の連結に双対な余積}

	\begin{proposition}[$\Delta_*$は余可換]\label{prop:Delta_*は余可換} %{
		$\Delta_*$は余可換である。
	\end{proposition} %prop:Delta_*は余可換}
	\begin{proof} %{
		文字数についての帰納法で証明する。
		$\Delta_*1_W=1_W\otimes 1_W$だから、文字数が$0$の場合は余可換となる
		ことがわかる。
		任意の$a\in A$に対して$\Delta_*1[a]=[a]\otimes 1_W+[a]\otimes 1_W$
		だから、文字数が$1$の場合も余可換となることがわかる。
		文字数が$n\in\mybf{N}\bou 1\le n$以下の任意の単語に対して$\Delta\*$が
		余可換だとする。
		\begin{equation*}\begin{split} %{
			\Delta_*^{(1)}w\otimes\Delta_*^{(2)}w
			=\Delta_*^{(2)}w\otimes\Delta_*^{(1)}w
			\quad\text{for all }w\in WA\bou \zettai{w}\le n
		\end{split}\end{equation*} %}
		$w_1,w_2$を文字数が$n$以下の単語とする。次の式から単語$w_1*w_2$は
		余可換になることがわかる。
		\begin{equation*}\begin{split} %{
			\Delta_*(w_1*w_2) &= (\Delta_*w_1)*(\Delta_*w_2) \\
			&= \left((\Delta_*^{(1)}w_1)*(\Delta_*^{(1)}w_2)\right)
			\otimes \left((\Delta_*^{(2)}w_1)*(\Delta_*^{(2)}w_2)\right) \\
			&= \left((\Delta_*^{(2)}w_1)*(\Delta_*^{(2)}w_2)\right)
			\otimes \left((\Delta_*^{(1)}w_1)*(\Delta_*^{(1)}w_2)\right) \\
			& = \left(\Delta_*^{(2)}(w_1*w_2)\right)
			\otimes \left(\Delta_*^{(1)}(w_1*w_2)\right) \\
		\end{split}\end{equation*} %}
		任意の$n+1$文字の単語$w$は、ある$1$文字の単語$x$とある$n$文字の単語$y$
		の積$w=x*y$で書くことができるので、任意の$n+1$文字の単語に対する
		余積$\Delta_*$は余可換となることがわかる。
	\end{proof} %}

	ここで、余可換な余積について調べる。$\Delta$を一般の余積とすると、
	余結合性より、$\Delta^2$を次のように定義することができる。
	\begin{equation*}\begin{split} %{
		\Delta^2w=(\Delta\otimes\myid)\Delta w=(\myid\otimes\Delta)\Delta w
	\end{split}\end{equation*} %}
	同様にして、$\Delta^3$を次のように定義することができる。
	\begin{equation*}\begin{split} %{
		\Delta^3w
		&=(\Delta\otimes\myid\otimes\myid)(\Delta\otimes\myid)\Delta w \\
		&=(\myid\otimes\Delta\otimes\myid)(\Delta\otimes\myid)\Delta w \\
		&=(\myid\otimes\myid\otimes\Delta)(\Delta\otimes\myid)\Delta w \\
		&=(\Delta\otimes\myid\otimes\myid)(\myid\otimes\Delta)\Delta w \\
		&=(\myid\otimes\Delta\otimes\myid)(\myid\otimes\Delta)\Delta w \\
		&=(\myid\otimes\myid\otimes\Delta)(\myid\otimes\Delta)\Delta w \\
	\end{split}\end{equation*} %}
	任意の$n\in\mybf{N}_+$に対して、$\Delta^n$を次のように定義する。
	\begin{equation*}\begin{split} %{
		\Delta^nw = \Delta^{(1)}w\otimes \Delta^{(1)}\Delta^{(2)}w\otimes 
		\cdots\otimes \Delta^{(1)}\Delta^{(2)(n-1)}w\otimes \Delta^{(2)n}w
	\end{split}\end{equation*} %}
	この式を次の二分木で表すことにする。
	\begin{equation}\label{eq:余積の二分木}
		\Delta^nw = \xymatrix@R=1pc@C=1pc{
			w \ar[r]\ar[d] & \Delta^{(2)}w \ar[r]\ar[d] 
			& \cdots \ar[r] & \Delta^{(2)(n-1)} \ar[r]\ar[d] & \Delta^{(2)n}w \\
			\Delta^{(1)}w & \Delta^{(1)}\Delta^{(2)}w 
			& \cdots & \Delta^{(1)}\Delta^{(2)(n-1)}w \\
		}
	\end{equation}
	この木のどの葉に対して余積$\Delta$をとっても、余結合性により次の図のよう
	になり、最終的に図\eqref{eq:余積の二分木}の二分木の形にすることができる。
	\begin{equation*}\begin{split} %{
		&\xymatrix@R=1pc@C=1pc{
			\cdots \ar[r] & \Delta^{(2)k}w \ar[r]\ar[d] & \Delta^{(2)(k+1)} \ar[r]\ar[d] & \cdots \\
			& \Delta^{(1)}\Delta^{(2)k}w \ar[ld]\ar[rd] & \Delta^{(1)}\Delta^{(2)(k+1)}w \\
			\Delta^{(1)}\Delta^{(1)}\Delta^{(2)(k+1)}w && \Delta^{(2)}\Delta^{(1)}\Delta^{(2)(k+1)}w \\
		} \\
		\\
		&= \xymatrix@R=1pc@C=1pc{
			\cdots \ar[r] & \Delta^{(2)k}w \ar[r]\ar[d] & \Delta^{(2)(k+1)} \ar[r]\ar[d] & \cdots \\
			& \Delta^{(1)}\Delta^{(2)k}w & \Delta^{(1)}\Delta^{(2)(k+1)}w \ar[ld]\ar[rd] \\
			& \Delta^{(1)}\Delta^{(1)}\Delta^{(2)k}w && \Delta^{(2)}\Delta^{(1)}\Delta^{(2)k}w \\
		} \\
	\end{split}\end{equation*} %}
	余表現の面から余積の余可換性を見てみる。
	$WA$の基底を$\set{e_0,e_1,\cdots}$とおき、$R$値行列$\Delta_i^{jk}$を
	用いて$\Delta e_i=\Delta_i^{jk}e_j\otimes e_k$とする。
	余積$\Delta$の余結合性は、余表現に対する
	$\Delta_i^{ja}\Delta_a^{kl}=\Delta_a^{jk}\Delta_i^{al}$という条件になる。
	図\eqref{eq:余積の二分木}による$\Delta^n$の表し方は、
	$(\Delta^3)_i^{jklm}=\Delta_i^{ja}\Delta_a^{kb}\Delta_b^{lm}$
	という縮約のとり方を指定していることに対応する。
	余積$\Delta$が余可換であった場合、$\Delta_i^{jk}=\Delta_i^{kj}$となる。
	$\Delta^2$についてみてみる。
	$(\Delta^2)_i^{jkl}=\Delta_i^{ja}\Delta_a^{kl}$となるが、
	$(\Delta^2)_i^{jkl}$の添え字$(kl)$について対称なことは、
	余積$\Delta$が余可換であることからわかる。また、余結合性
	$\Delta_i^{ja}\Delta_a^{kl}=\Delta_a^{jk}\Delta_i^{al}$を使うと、
	$(\Delta^2)_i^{jkl}$の添え字$(jk)$について対称なこともわかる。
	したがって、$(\Delta^2)_i^{jkl}$は添え字$(jkl)$の任意の置換で不変なこと
	がわかる。同様にして、$(\Delta^n)_i^{i_1i_2\cdots i_{n+1}}$は添え字
	$(i_1i_2\cdots i_{n+1})$の任意の置換で不変なこともわかる。
	したがって、Sweedler記法を用いて
	$\Delta^nw=w_{(1)}\otimes w_{(2)}\otimes\cdots\otimes w_{(n+1)}$
	と書くと、$\Delta$が余可換であれば、$n+1$次の任意の置換$\sigma$に対して
	$\Delta^nw=w_{(\sigma1)}\otimes w_{(\sigma2)}\otimes\cdots\otimes w_{\left(\sigma(n+1)\right)}$
	となる。
%s1:文字列}

\section{文字列とその双対空間}\label{s1:文字列とその双対空間} %{
	前節\ref{s1:文字列}と同じ記号を用いて、さらに、
	$WA$から$\mybf{2}\subseteq R$への写像全体の作る空間を$WA^t$、
	$RWA$から$R$への$R$線形写像全体の作る空間を$RWA^t$とする。
	単語$w\in WA$に双対な元を$w^t$と書く。
	\begin{equation*}\begin{split} %{
		w_1^tw_2 = \jump{w_1=w_2} \quad\text{for all }w_1,w_2\in WA
	\end{split}\end{equation*} %}
	$WA$の元を$A$の元を用いて書いた場合にも同様に書く。
	例えば、$a_1,a_2,\dots, a_n\in A$として、$[a_1a_2\cdots a_n]\in WA$
	に双対な元を$[a_1a_2\cdots a_n]^t\in WA^t$と書く。

	次の畳み込みによって$RWA$に余積$\Delta_\sqcup$を定義する。
	\begin{equation*}\begin{split} %{
		\xymatrix{
			RWA\otimes RWA \ar[r]^{m_*} \ar@{.>}[d]^{\Delta_\sqcup f} 
			& RWA \ar[d]^{f} \\
			R\otimes R & R \ar[l]_{\Delta_\myspace} \\
		} \\
		\text{where } \Delta_\myspace r = r\otimes 1 = 1\otimes r
		\quad\text{for all }r\in R
	\end{split}\end{equation*} %}
	$R$値行列$D$を用いて、任意の$w\in WA$に対して
	$\Delta_\sqcup w^t=\sum_{w_1,w_2\in WA}D_{w_1w_2}^ww_1^t\otimes w_2^t$
	とすると、可換図式により$D_{w_1w_2}^w=w^t(w_1*w_2)$となる。したがって、
	余積$\Delta_\sqcup$は具体的に求まって次のようになる。
	\begin{equation*}\begin{split} %{
		\Delta_\sqcup w^t 
		= \sum_{w_1,w_2\in WA}\jump{w=w_1*w_2}w_1^t\otimes w_2^t
		\quad\text{for all }w\in WA
	\end{split}\end{equation*} %}
	$A$の元を用いて書くと、次のようになる。
	\begin{equation*}\begin{split} %{
		\Delta_\sqcup [a_1a_2\cdots a_m]^t
		&= 1_W^t\otimes [a_1a_2\cdots a_m]^t + [a_1]^t\otimes [a_2\cdots a_m]^t  \\
		&\;+ [a_1a_2]^t\otimes [\cdots a_m]^t + \cdots + [a_1a_2\cdots a_m]^t\otimes 1_W^t \\
		&\quad\text{for all }a_1,a_2,\dots,a_m\in A
	\end{split}\end{equation*} %}
	さらに、$m_\myspace\Delta_\myspace=\myid$となるから、余積$\Delta_\sqcup$
	は次の可換図を満たす。
	\begin{equation*}\xymatrix{
		RWA\otimes RWA \ar[r]^{m_*} \ar[d]^{\Delta_\sqcup f} 
		& RWA \ar[d]^{f} \\
		R\otimes R \ar[r]^{m_\myspace} & R \\
	}\end{equation*}
	また、余積$\Delta_\sqcup$に対する余単位射$\epsilon_\sqcup$は、
	余積$\Delta_\sqcup$に対する単位射$\epsilon_*$と同じ形になる。
	\begin{equation*}\begin{split} %{
		\epsilon_\sqcup w^t = \jump{w=1_W} \quad\text{for all }w\in WA
	\end{split}\end{equation*} %}

	次の畳み込みによって$RWA$に余積$m_\sqcup$を定義する。
	\begin{equation*}\begin{split} %{
		\xymatrix{
			RWA\otimes RWA \ar[d]^{f\otimes g} 
			& RWA \ar[l]_{\Delta_*} \ar@{.>}[d]^{f\sqcup g}\\
			R\otimes R \ar[r]^{m_\myspace} & R  \\
		} 
	\end{split}\end{equation*} %}
	$m_*$と$\Delta_*$、$m_\myspace$と$\Delta_\myspace$が共に双対な積と余積
	になっているから、積$m_\sqcup$と余積$\Delta_\sqcup$は双対になる。
	さらに、$\Delta_\myspace m_\myspace=\myid\otimes \myid$となるから、
	積$m_\sqcup$は次の可換図を満たす。
	\begin{equation*}\begin{split} %{
		\xymatrix{
			RWA\otimes RWA \ar[d]^{f\otimes g} 
			& RWA \ar[l]_{\Delta_*} \ar[d]^{f\sqcup g}\\
			R\otimes R & R \ar[l]_{\Delta_\myspace} \\
		} 
	\end{split}\end{equation*} %}
	$R$値行列$D$を用いて、任意の$w_1,w_2\in WA$に対して
	$w_1^t\sqcup w_2^t=\sum_{w\in WA}M_{w_1w_2}^ww^t$とすると、
	可換図式により$M_{w_1w_2}^w=(w_1^t\Delta_{*(1)}w)(w_2^t\Delta_{*(2)}w)$
	となる。余積$\Delta_\sqcup$の場合と異なり、$m_\sqcup$の具体的な形を
	求めることは難しい。しかし、次の事柄はすぐわかる。
	\begin{itemize}\setlength{\itemsep}{-1mm} %{
		\item $\Delta_*$が余可換だから、$m_\sqcup$は可換になる。つまり、
		任意の$w_1,w_2\in WA$に対して$w_1^t\sqcup w_2^t=w_2^t\sqcup w_1^t$
		となる。
		\item $\Delta_*$の余単位射$\epsilon_*$が$\epsilon_*=1_W^t$となるから、
		任意の$w,w_2\in WA$に対して
		\begin{equation*}\begin{split} %{
			M_{1_Ww_2}^w &= (1_W^t\Delta_{*(1)}w)(w_2^t\Delta_{*(2)}w) \\
			&= (\epsilon_*\Delta_{*(1)}w)(w_2^t\Delta_{*(2)}w) \\
			&= w_2^tw \\
		\end{split}\end{equation*} %}
		となり、$1_W$が単位元になる。つまり、
		任意の$w\in WA$に対して$w^t\sqcup 1_W^t=w=1_W^t\sqcup w^t$となる。
	\end{itemize} %}
	余積$\Delta_\sqcup$の具体的な形を求めることを考える。
	まず、任意の$a\in A$と$w,x\in WA$に対して
	$M_{[a]x}^w=([a]^t\Delta_{*(1)}w)*(x^t\Delta_{*(2)}w)$
	となり、任意の$a,b_1,b_2,\dots,b_m\in A$に対して
	\begin{equation*}\begin{split} %{
		[a]^t\sqcup [b_1b_2\cdots b_m]^t
		&= \sum_{w\in WA}M_{[a][b_1b_2\cdots b_m]}^ww^t \\
		&= \sum_{w\in WA}\jump{\Delta_*w=[a]\otimes [b_1b_2\cdots b_m]+\cdots}w^t \\
		&= [ab_1b_2\cdots b_m]^t + [b_1ab_2\cdots b_m]^t
			+ \cdots + [b_1b_2\cdots b_ma]^t \\
	\end{split}\end{equation*} %}
	となることがわかる。つまり、次のようになる。
	\begin{equation*}\begin{split} %{
		[a]^t\sqcup w^t 
		&= (\Delta_{\sqcup(1)}w^t)*[a]^t*(\Delta_{\sqcup(2)}w^t) \\
		&\quad\text{for all }a\in A,\;w\in WA
	\end{split}\end{equation*} %}
	\begin{todo}[双対空間に対する積$m_*$]\label{todo:双対空間に対する積} %{
		双対空間$WA^t$に対する積$m_*$の定義をしないまま使っている。
		転置${\mathchar`-}^t:WA\to WA^t$を同型で定義
		\begin{equation*}\begin{split} %{
			(w_1*w_2)^t &= w_1^t * w_2^t \\
			(w_1^t\otimes w_2^t)(w_3\otimes w_4) &= (w_1^tw_3)(w_2^tw_4) \\
		\end{split}\end{equation*} %}
		するか、反同型で定義
		\begin{equation*}\begin{split} %{
			(w_1*w_2)^t &= w_2^t * w_1^t \\
			(w_1^t\otimes w_2^t)(w_3\otimes w_4) &= (w_2^tw_3)(w_1^tw_4) \\
		\end{split}\end{equation*} %}
		するかで双対空間$WA^t$の文字$A^t$を並べる順序が異なってくる。
	\end{todo} %todo:双対空間に対する積}
	次に、任意の$a_1,a_2\in A$と$w,x\in WA$に対して
	$M_{[a_1a_2]x}^w=([a_1a_2]^t\Delta_{*(1)}w)*(x^t\Delta_{*(2)}w)$
	となり、任意の$a_1,a_2,b_1,b_2,\dots,b_m\in A$に対して
	\begin{equation*}\begin{split} %{
		[a_1a_2]^t\sqcup [b_1b_2\cdots b_m]^t
		&= \sum_{w\in WA}M_{[a_1a_2][b_1b_2\cdots b_m]}^ww^t \\
		&= \sum_{w\in WA}\jump{\Delta_*w=[a_1a_2]\otimes [b_1b_2\cdots b_m]+\cdots}w^t \\
		&= [a_1a_2b_1b_2\cdots b_m]^t + [a_1b_1a_2b_2\cdots b_m]^t
			+ \cdots + [a_1b_1b_2\cdots b_ma_2]^t \\
		&\;+ [b_1a_1a_2b_2\cdots b_m]^t + \cdots + [b_1a_1b_2\cdots b_ma_2]^t \\
		&\;+ \cdots \\
		&\;+ [b_1b_2\cdots a_1a_2b_m]^t + [b_1b_2\cdots a_1b_ma_2]^t \\
		&\;+ [b_1b_2\cdots b_ma_1a_2]^t \\
	\end{split}\end{equation*} %}
	となることがわかる。つまり、次のようになる。
	\begin{equation*}\begin{split} %{
		[a_1a_2]^t\sqcup w^t
		&= (\Delta_{\sqcup(1)}w^t)*[a_1]^t
			*(\Delta_{\sqcup(1)}\Delta_{\sqcup(2)}w^t)*[a_2]^t
			*(\Delta_{\sqcup(2)}\Delta_{\sqcup(2)}w^t) \\
		&= (\Delta_{\sqcup(1)}w^t)*[a_1]^t
			*\bigl([a_2]^t\sqcup(\Delta_{\sqcup(2)}w^t)\bigr) \\
		&\quad\text{for all }a_1,a_2\in A,\;w\in WA
	\end{split}\end{equation*} %}
	余積$\Delta_\sqcup$の$W_1A\otimes WA\to RWA$と$W_2A\otimes WA\to RWA$の
	場合の計算から、次の式が成り立つことが予想される。
	\begin{equation*}\begin{split} %{
		([a]*w_1)^t\sqcup w_2^t 
		&= (\Delta_{\sqcup(1)}w_2^t)*[a]^t
			*\bigl(w_1^t\sqcup(\Delta_{\sqcup(2)}w_2^t)\bigr) \\
		&\quad\text{for all }a\in A,\;w_1,w_2\in WA
	\end{split}\end{equation*} %}

	\begin{todo}[余積の計算]\label{todo:余積の計算} %{
		次の式が成り立つことを証明するためには、$\Delta_*$に対して成り立つ式
		がいくつか必要になると思われる。
		\begin{equation*}\begin{split} %{
			([a]*w_1)^t\sqcup w_2^t 
			&= (\Delta_{\sqcup(1)}w_2^t)*[a]^t
				*\bigl(w_1^t\sqcup(\Delta_{\sqcup(2)}w_2^t)\bigr) \\
			&\quad\text{for all }a\in A,\;w_1,w_2\in WA
		\end{split}\end{equation*} %}
		積$m_\sqcup$は余積$\Delta_*$から次のように定義される。
		\begin{equation*}\begin{split} %{
			w_1^t\sqcup w_2^t &= \sum_{w\in WA}M_{w_1w_2}^ww^t \\
			&= \sum_{w\in WA}(w_1^t\Delta_{*(1)}w)(w_2^t\Delta_{*(2)}w)w^t \\
			&= \sum_{w\in WA}\jump{\Delta_{*}w\propto w_1\otimes w_2+\cdots}w^t \\
		\end{split}\end{equation*} %}
	\end{todo} %todo:余積の計算}
%s1:文字列とその双対空間}

	\begin{todo}[ここまで]\label{todo:ここまで} %{
	\end{todo} %todo:ここまで}

	\begin{todo}[シャッフル積]\label{todo:シャッフル積} %{
		次の可換図で定義された$R$双線形二項演算$\beta_\sqcup$は積になるか?
		\begin{equation}\xymatrix{
			RWA\otimes RWA \ar[r]^{m_*} \ar@{.>}[d]^{\beta_\sqcup} 
			& RWA \ar[d]^{\Delta_*} \\
			RWA & RWA\otimes RWA \ar[l]_{m_*} \\
		}\end{equation}
	\end{todo} %todo:シャッフル積}

	次の$R$線形写像$\Delta_\amalg$は余積になる。
	\begin{equation}\begin{split} %{
		\Delta_\amalg: RWA &\to RWA\otimes RWA \\
			[a_1a_2\cdots a_{m-1}a_m] 
				&\mapsto [a_1a_2\cdots a_{m-1}a_m]\otimes 1_W \\
				&\quad + [a_1a_2\cdots a_{m-1}]\otimes [a_m] \\
				&\quad + \cdots \\
				&\quad + [a_1]\otimes [a_2\cdots a_{m-1}a_m] \\
				&\quad + 1_W\otimes [a_1a_2\cdots a_{m-1}a_m] \\
	\end{split}\end{equation} %}
	余積$\Delta_\amalg$に対する余単位射は$\epsilon_*$となる。

	\begin{todo}[余積から積の導出]\label{todo:余積から積の導出} %{
		与えられた余積と双対になる積を導出する方法を考える。
		逆の場合の、与えられた積に双対になる余積の導出は、文字数の小さいもの
		から大きなものを順の求めていけばよい。
		一般に、積$m_\odot$に双対な余積$\Delta_\odot$は次のようになる。
		\begin{equation}\begin{split} %{
			\Delta_\odot(w_1\odot w_2) &= (\Delta_\odot w_1)\odot(\Delta_\odot w_2) \\
		\end{split}\end{equation} %}
		したがって、積$m_\odot$が文字数を保存する場合には、文字数の小さいもの
		から大きなものへと余積$\Delta_\odot$が順に求まる。
	\end{todo} %todo:余積から積の導出}


	\subsection{作用から積の導出}\label{s2:作用から積の導出} %{
		$A,B$を集合、$R$を半環とする。
		$R$双線形作用$\beta_\rhd:RA\otimes RB\to RB$
		と$R$双線形二項演算$\beta_\perp:RA\otimes RA\to RA$が定義されていて、
		次の性質を満たすとする。
		\begin{equation}\label{eq:結合的な作用}\begin{split} %{
			a_2\rhd a_1\rhd b &= (a_2\perp a_1)\rhd b
			\quad\text{for all }a_1,a_2\in A,\;b\in B
		\end{split}\end{equation} %}
		すると、任意の$a_1,a_2,a_3\in A,\;b\in B$に対して、内側から
		二項演算$\perp$に書き直していくと
		\begin{equation*}\begin{split} %{
			a_3\rhd a_2\rhd a_1\rhd b &= a_3\rhd(a_2\perp a_1)\rhd b \\
			&= \Bigl(a_3\perp(a_2\perp a_1)\Bigr)\rhd b
		\end{split}\end{equation*} %}
		となり、外側から二項演算$\perp$に書き直していくと
		\begin{equation*}\begin{split} %{
			a_3\rhd a_2\rhd a_1\rhd b &= (a_3\perp a_2)\rhd a_1\rhd b \\
			&= \Bigl((a_3\perp a_2)\perp a_1\Bigr)\rhd b
		\end{split}\end{equation*} %}
		となる。したがって、$A$の同値関係$\sim_\rhd$
		\begin{equation}\begin{split} %{
			a_1 \sim_\rhd a_2 \implies
			a_1\rhd b = a_2\rhd b \quad\text{for all }b\in B
		\end{split}\end{equation} %}
		を用いて、$\perp$の結合性が導かれる。
		\begin{equation*}\begin{split} %{
			a_3\perp(a_2\perp a_1) \sim_\rhd (a_3\perp a_2)\perp a_1
		\end{split}\end{equation*} %}

		$\rhd$と$\perp$の関係式\eqref{eq:結合的な作用}を可換図で書くと
		次のようになる。
		\begin{equation*}\xymatrix@C+1pc{
			RA\otimes RA\otimes RB \ar[d]^{\myid\otimes \beta_\rhd} \ar[r]^{\beta_\perp\otimes \myid} 
			& RA\otimes RB \ar[d]^{\beta_\rhd} \\
			RA\otimes RB \ar[r]^{\beta_\rhd} & RB \\
		}\end{equation*}
		余作用$\nabla:RB\to RA\otimes RB$と
		余二項演算$\gamma:RA\to RA\otimes RA$を用いて、
		この可換図の矢印の向きを反転させると次の可換図になる。
		\begin{equation*}\xymatrix@C+1pc{
			RA\otimes RA\otimes RB  
			&  RA\otimes RB \ar[l]_{\gamma\otimes \myid} \\
			RA\otimes RB \ar[u]_{\myid\otimes \nabla}
			& RB \ar[l]_{\nabla} \ar[u]_{\nabla} \\
		}\end{equation*}
		式で書くと$(\myid\otimes\nabla)\nabla=(\gamma\otimes\myid)\nabla$
		となる。この関係を$
		(\myid\otimes\myid\otimes\nabla)(\myid\otimes\nabla)\nabla
		$に対して左から適用していくと
		\begin{equation*}\begin{split} %{
			(\myid\otimes\myid\otimes\nabla)(\myid\otimes\nabla)\nabla
			&= (\myid\otimes\gamma\otimes\myid)(\myid\otimes\nabla)\nabla \\
			&= (\myid\otimes\gamma\otimes\myid)(\gamma\otimes\myid)\nabla \\
		\end{split}\end{equation*} %}
		となり、右から適用していくと
		\begin{equation*}\begin{split} %{
			(\myid\otimes\myid\otimes\nabla)(\myid\otimes\nabla)\nabla
			&= (\myid\otimes\myid\otimes\nabla)(\gamma\otimes\myid)\nabla \\
			&= (\gamma\otimes\nabla)\nabla \\
			&= (\gamma\otimes\myid\otimes\myid)(\myid\otimes\nabla)\nabla \\
			&= (\gamma\otimes\myid\otimes\myid)(\gamma\otimes\myid)\nabla \\
		\end{split}\end{equation*} %}
		となる。したがって、$\set{\nabla^{(1)}b}_{b\in B}$で張られる
		部分空間$\nabla^{(1)}A\subseteq A$内で、$\gamma$の余結合性が導かれる。
		\begin{equation*}\begin{split} %{
			(\myid\otimes\gamma)\gamma = (\gamma\otimes\myid)\gamma
			\quad\text{in }\nabla^{(1)}A
		\end{split}\end{equation*} %}

		一般には、与えられた作用$\rhd$に対して、式\eqref{eq:結合的な作用}
		を満たすような二項演算$\perp$は定義できないが、
		$\beta_\rhd:RA\otimes RB\to RB$から空間$RA$を二項演算$\perp$が定義
		できるところまで拡大していけることがある。
		\begin{equation*}\begin{split} %{
			\xymatrix{
				RA \ar[dr]_{\mybiop{\rhd}} \ar[r]^{i}
				& R\widetilde{A} \ar[d]^{\mybiop{\widetilde{\rhd}}} \\
				& \myop{end}RB \\
			} \quad
			\mybiop{\widetilde{\rhd}}(\myid\otimes\mybiop{\widetilde{\rhd}})
			=\mybiop{\widetilde{\rhd}}(\mybiop{\perp}\otimes\myid)
		\end{split}\end{equation*} %}
		空間の拡大の仕方が、文字から単語、単語から木、木からグラフといった
		データ構造を拡大していく方法が使えることがある。
	%s2:作用から積の導出}

	\subsection{木の定義}\label{s2:木の定義} %{
		木構造は様々な場面で使われ、使う場面ごとに様々な定義がある。
		例えば、通常のプログラミングで使われる木いうのは、数学では
		平面上のラベル付き根付き木\cite{arxiv:hoffman:0710.3739}という長い
		修飾子がついた木になる。この節で使ういくつかの木を定義しておく。

		\begin{todo}[定義すべきもの]\label{todo:定義すべきもの} %{
			\begin{description} %{
				\item[根付きの木]木の同値関係の観点から
				\item[平面上の木]木の同値関係の観点から
				\item[ラベル付きの木]木の同値関係の観点から
				\item[頂点の指定の仕方]三種類の頂点の指定の仕方
				\begin{itemize} %{
					\item 行きがけ順
					\item 帰りがけ順
				\end{itemize} %}
				頂点に順序をつけて表す。木の操作ごとに適切な順序を用いる。
				例えば、次のような場合は、順序関係$i\le j$は行きがけ順と帰りがけ順
				で異なる。
				\begin{equation*}\begin{split} %{
					\sum_{i\le j\in\myop{pre}t}t\lhd_ix\lhd_jy
				\end{split}\end{equation*} %}
				また、木を単語に直す際、行きがけ順と帰りがけ順では、文字の並び以外
				にも単語としての同値性も違いがでる。例えば、次の五つの木は、
				行きがけ順で単語にするとすべて同じ単語になってしまう。
				\begin{equation*}\begin{array}{rlllll} %{
					& \mytree{
						& a_0 \ar@{-}[dl]\ar@{-}[d]\ar@{-}[dr] \\
						a_1 & a_2 & a_3 \\
					},& \mytree{
						& a_0 \ar@{-}[dl]\ar@{-}[dr] \\
						a_1 \ar@{-}[d] && a_3 \\
						a_2 \\
					},& \mytree{
						& a_0 \ar@{-}[dl]\ar@{-}[dr] \\
						a_1 && a_2 \ar@{-}[d]\\
						&& a_3
					},& \mytree{
						& a_0 \ar@{-}[d] \\
						& a_1 \ar@{-}[dl]\ar@{-}[dr] \\
						a_2 && a_3
					},& \mytree{
						a_0 \ar@{-}[d] \\
						a_1 \ar@{-}[d] \\
						a_2 \ar@{-}[d] \\
						a_3
					} \\
					\text{pre} & [0123] & [0123] & [0123] & [0123] & [0123] \\
					\text{post} & [1230] & [2130] & [1320] & [2310] & [3210] \\
				\end{array}\end{equation*} %}
				逆に、次の五つの木は、帰りがけ順で単語にするとすべて同じ単語に
				なってしまう。
				\begin{equation*}\begin{array}{rlllll} %{
					& \mytree{
						& a_3 \ar@{-}[dl]\ar@{-}[d]\ar@{-}[dr] \\
						a_0 & a_1 & a_2 \\
					},& \mytree{
						& a_3 \ar@{-}[dl]\ar@{-}[dr] \\
						a_1 \ar@{-}[d] && a_2 \\
						a_0 \\
					},& \mytree{
						& a_3 \ar@{-}[dl]\ar@{-}[dr] \\
						a_0 && a_2 \ar@{-}[d]\\
						&& a_1
					},& \mytree{
						& a_3 \ar@{-}[d] \\
						& a_2 \ar@{-}[dl]\ar@{-}[dr] \\
						a_0 && a_1
					},& \mytree{
						a_3 \ar@{-}[d] \\
						a_2 \ar@{-}[d] \\
						a_1 \ar@{-}[d] \\
						a_0
					} \\
					\text{pre} & [3012] & [3102] & [3021] & [3201] & [3210] \\
					\text{post} & [0123] & [0123] & [0123] & [0123] & [0123] \\
				\end{array}\end{equation*} %}
				\item[木の書き方]子供を括弧$[]$でくくって書く。また、文献
				\cite{arxiv:hoffman:0710.3739}のBalanced Bracket Arrangement
				という書き方は、この書き方のラベルなしの場合になっている。
				\item[空の木の取り扱い]空の木をどのように取り扱うかを明記すること。
				\item[森]森を定義すること。
			\end{description} %}
		\end{todo} %todo:定義すべきもの}
	%s2:木の定義}

	\subsection{この節で扱う木}\label{s2:この節で扱う木} %{
		$A$を集合、$TA$を$A$を頂点に持つ木の集合とする。
		$TA$には空の木を含めるとする。空の木を含めない場合は$T_+A$と書く。

		$TA$の元は根が固定されているものとする(根付き木)。
		例えば、次のような同値関係とする。
		\begin{equation*}\begin{split} %{
			\mytree{
				& a_0 \ar@{-}[dl] \ar@{-}[dr] \\
				a_1 && a_2 \\
			} \neq \mytree{
				a_1 \ar@{-}[d] \\
				a_0 \ar@{-}[d] \\
				a_2 \\
			} 
		\end{split}\end{equation*} %}
		また、$TA$の同値関係は子供の頂点の並びまで含めるものする(ラベル付き木)。
		例えば、次のような同値関係とする。
		\begin{equation*}\begin{split} %{
			\mytree{
				& a_0 \ar@{-}[dl] \ar@{-}[dr] \\
				a_1 && a_2 \\
			} = \mytree{
				& a_0 \ar@{-}[dl] \ar@{-}[dr] \\
				a_2 && a_1 \\
			} \iff a_1 = a_2
		\end{split}\end{equation*} %}

		木を次の図のように子供の頂点を括弧でくくって表すことにする。
		\begin{equation*}\begin{split} %{
			a_0[a_1[a_3]a_2[]] = \mytree{
				& a_0 \ar@{-}[dl] \ar@{-}[dr] \\
				a_1 \ar@{-}[d] && a_2 \\
				a_3 \\
			}
		\end{split}\end{equation*} %}
		子供を持たない頂点に関しては、$a_0[a_1[a_3]a_2]=a_0[a_1[a_3]a_2[]]$の
		ように括弧を省略して書くこともある。
		頂点を一つも持たない木を$1_T$と書く。
		木$t$の頂点の個数を$\zettai{t}$と書く。
		例えば、$\zettai{a_0[a_1[a_3]a_2]}=3$、$\zettai{1_T}=0$となる。

		$A$から$TA$への写像$i_T$を次のように定める。
		\begin{equation*}\begin{split} %{
			i_Ta = a[]
		\end{split}\end{equation*} %}
		$i_TA=\set{i_Ta}_{a\in A}\subseteq TA$は根だけからなる木の集合となり、
		頂点数が一つの木はすべて$i_TA$に含まれる。

		$TA$から生成される自由モノイド$WTA$の元を森という。森の文字が木になる。
		$a\in A$を根とし、根の子供の部分木が$t_1,t_2,\dots,t_m$となる木
		を$a[t_1t_2\cdots t_m]$と書く。また、文字列の連結$m_*$を用いて
		$a([t_1]*[t_2\cdots t_m])=a[t_1t_2\cdots t_m]$とも書く。

		\begin{todo}[空の木と空の森]\label{todo:空の木と空の森} %{
			空の木と空の森は同一視されるべきものではないだろうか?
			単語の場合と異なり、空の木を含めない$T_+A$で考えた方がきれいに
			まとまるように思える。
		\end{todo} %todo:空の木と空の森}
	%s2:この節で扱う木}

	\subsection{根への接木}\label{s2:根への接木} %{
		木の頂点を行きがけ順に並べることで、木から文字列への写像が定義できる。
		例えば、$a_0[a_1[a_3]a_2]\mapsto[a_0a_1a_3a_2]$という写像になる。
		この写像を$\pi_W$と書く。次の可換図を満たす、$T_+A$の二項演算$\land_0$
		を定義する。
		\begin{equation}\label{eq:行きがけ順を保つ二項演算}\xymatrix{
			T_+A\times T_+A \ar@{.>}[d]^{\beta_{\land_0}} \ar[r]^{\pi_W\times\pi_W}
			& WA\times WA \ar[d]^{m_*} \\
			T_+A \ar[r]^{\pi_W} & WA \\
		}\end{equation}
		$\land_0$は一意には決まらないが、次のように定義すれば可換図を満たす。

		\begin{definition}[根への接木]\label{def:根への接木} %{
			$R$双線形二項演算$\beta_{\land_0}: RT_+A\otimes RT_+A \to RT_+A$を、
			任意の$t\in T_+A$に対して次のように、
			\begin{equation*}\begin{split} %{
				t\land_0 1_T = t \\
			\end{split}\end{equation*} %}
			任意の$t,t_1,t_2,\dots,t_m\in T_+A$に対して次のように定義する。
			\begin{equation*}\begin{split} %{
				a[t_1t_2\cdots t_m]\land_0 t &= a[t_1t_2\cdots t_mt] 
			\end{split}\end{equation*} %}
			二項演算$\land_0$を根への接木ということにする。
		\end{definition} %def:根への接木}

		$t_1\land_0 t_2$は木$t_1$の根の最右の子供に木$t_2$を付け足す操作である。
		$\land_0$は結合性を満たさない。例えば、任意の$a_1,a_2,a_3\in A$に
		対して次のようになり、
		\begin{equation*}\begin{array}{rll} %{
			a_1\land_0(a_2\land_0 a_3) &= a_1\land_0 a_2[a_3] &= a_1[a_2[a_3]] \\
			(a_1\land_0 a_2)\land_0 a_3 &= a_1[a_2]\land_0 a_3 &= a_1[a_2a_3] \\
		\end{array}\end{equation*} %}
		次のように結合性が満たされないことがわかる。
		\begin{equation*}\begin{split} %{
			a_1\land_0(a_2\land_0 a_3) \neq (a_1\land_0 a_2)\land_0 a_3
		\end{split}\end{equation*} %}
		根への接木$\land_0$は結合性を満たさないが、根だけから木
		$\set{a[]}_{a\in A}$から$T_+A$を生成することができる。
		例えば、次のようになる。
		\begin{equation*}\begin{split} %{
			&a_1\land_0 a_2 = \mytree{
				a_1 \ar@{-}[d] \\
				a_2
			} \\
			&(a_1\land_0 a_2)\land_0 a_3 = \mytree{
				& a_1 \ar@{-}[ld] \ar@{-}[rd] \\
				a_2 && a_3
			},\quad a_1\land_0(a_2\land_0 a_3) = \mytree{
				a_1 \ar@{-}[d] \\
				a_2 \ar@{-}[d] \\
				a_3
			} \\
			&\bigl((a_1\land_0 a_2)\land_0 a_3\bigr)\land_0 a_4 = \mytree{
				& a_1 \ar@{-}[ld] \ar@{-}[d] \ar@{-}[rd] \\
				a_2 & a_3 & a_4
			},\quad \bigl(a_1\land_0 (a_2\land_0 a_3)\bigr)\land_0 a_4 = \mytree{
				& a_1 \ar@{-}[ld] \ar@{-}[rd] \\
				a_2 \ar@{-}[d] && a_4 \\
				a_3
			}\\
			&(a_1\land_0 a_2)\land_0 (a_3\land_0 a_4) = \mytree{
				& a_1 \ar@{-}[ld] \ar@{-}[rd] \\
				a_2 && a_3 \ar@{-}[d] \\
				&& a_4
			} \\
			&a_1\land_0\bigl((a_2\land_0 a_3)\land_0 a_4\bigr) = \mytree{
				& a_1 \ar@{-}[d] \\
				& a_2 \ar@{-}[ld] \ar@{-}[rd] \\
				a_3 && a_4
			},\quad a_1\land_0\bigl(a_2\land_0 (a_3\land_0 a_4)\bigr) = \mytree{
				a_1 \ar@{-}[d] \\
				a_2 \ar@{-}[d] \\
				a_3 \ar@{-}[d] \\
				a_4
			} \\
			&\cdots
		\end{split}\end{equation*} %}
		このことを命題の形で証明しておく。

		\begin{proposition}[木の生成]\label{prop:木の生成} %{
			$T_+A$の任意の元$t$は、$\zettai{t}$個の$i_TA$の元を根への
			接木$\land_0$で掛け合わせたもので書くことができる。
		\end{proposition} %prop:木の生成}
		\begin{proof} %{
			木の頂点数に関する帰納法によって証明する。
			頂点数が二つの木は、$a_1,a_2\in A$に対して$a_1\land_0 a_2=a_1[a_2]$
			となるから命題が成り立つ。$n$を$2$以上の自然数とし、頂点数が$n$個の木
			に対して命題が成り立つとする。頂点数が$n+1$の任意の木$t\in T_+A$は
			ある自然数$p$があって、空でない木$t_1,t_2,\dots,t_p,t_{p+1}\in T_+X$
			と根$a\in A$によって$t=a[t_1t_2\cdots t_pt_{p+1}]$と書ける。
			木$t_p$の頂点数は$n$以下で、木$a[t_1t_2\cdots t_p]$の頂点数も$n$以下
			である。したがって、帰納法の仮定によって、木$t_p$も
			木$a[t_1t_2\cdots t_p]$も共に$T_1X$からに$\land_0$よって生成される。
			また、$t=a[t_1t_2\cdots t_p]\land_0 t_{p+1}$だから、$t$もまた$\land_0$に
			よって生成される。
			したがって、頂点数が$n+1$に場合にも命題が成り立つ。
		\end{proof} %}

		根への接木$\land_0$に前節\ref{s2:作用から積の導出}の議論を適用してみる。
		任意の$a\in A,\;w\in WT_+A$と任意の木$t_1,t_2\in WT_+A$に対して、
		$((aw)\land_0 t_1)\land_0 t_2=a(w*[t_1t_2])$となるから、
		任意の森$w_1\in WT_+A$に対して$(aw)\land_0 w_1=a(w*w_1)$と定義すると、
		$((aw)\land_0 t_1)\land_0 t_2=(aw)\land_0([t_1]*[t_2])$と書ける。
		このことを、定義と命題の形にまとめておく。

		\begin{definition}[森による根への接木]\label{def:森による根への接木} %{
			$R$双線形作用$\beta_{\land_0}: RT_+A\otimes RWT_+A \to RT_+A$を、
			任意の$t\in T_+A$に対して次のように、
			\begin{equation*}\begin{split} %{
				t\land_0 1_W = t
			\end{split}\end{equation*} %}
			任意の$a\in A,\;w_1,w_2\in WT_+A$に対して次のように定義する。
			\begin{equation*}\begin{split} %{
				(aw_1)\land_0 w_2 &= a(w_1*w_2)
			\end{split}\end{equation*} %}
			作用$\land_0$を森による根への接木または、単に、根への接木ということに
			する。
		\end{definition} %def:森による根への接木}

		\begin{proposition}[根への接木と単語の連結の関係]\label{prop:根への接木と単語の連結の関係} %{
			任意の$t\in T_+A,\;w_1,w_2\in WT_+A$に対して次の式が成り立つ。
			\begin{equation*}\begin{split} %{
				(t\land_0 w_1)\land_0 w_2 = t\land_0(w_1*w_2)
			\end{split}\end{equation*} %}
		\end{proposition} %prop:根への接木と単語の連結の関係}
		\begin{proof} %{
			任意の木$t$はある$a\in A,\;w\in WT_+A$で$t=aw$と書くことができる。
			森による根への接木$\land_0$の定義\ref{def:森による根への接木}より、
			任意の$w_1,w_2\in WT_+A$に対して$
			((aw)\land_0 w_1)\land_0 w_2=a(w*w_1*w_2)=(aw)\land_0(w_1*w_2)
			$となり命題が成り立つ。
		\end{proof} %}

		根への接木に双対となる操作を定義する。

		\begin{definition}[根での枝刈り]\label{def:根での枝刈り} %{
			$R$線形余作用$\widebar{\land}_0:RT_+A\to RT_+A\otimes RWT_+A$を
			任意の$a\in A,\;w\in WT_+A$に対して次のように定義する。
			\begin{equation*}\begin{split} %{
				\widebar{\land}_0(aw) &= (aw_{(1)})\otimes(w_{(2)})
				\quad\text{where } \Delta_*w = w_{(1)}\otimes w_{(2)} \\
			\end{split}\end{equation*} %}
		\end{definition} %def:根での枝刈り}

		\begin{proposition}[根での枝刈りと単語の連結の関係]\label{prop:根での枝刈りと単語の連結の関係} %{
			根での枝刈り$\widebar{\land}_0$と余積$\Delta_*$は次の関係が成り立つ。
			\begin{equation*}\begin{split} %{
				(\widebar{\land}_0\otimes\myid)\widebar{\land}_0
				= (\myid\otimes\Delta_*)\widebar{\land}_0
			\end{split}\end{equation*} %}
		\end{proposition} %prop:根での枝刈りと単語の連結の関係}
		\begin{proof} %{
			$\Delta_*$に対してSweedler記法を使ってテンソルの成分を計算する。
			任意の$w\in WT_+A$に対して次の式が成り立つ。
			\begin{equation*}\begin{split} %{
				(\widebar{\land}_0\otimes\myid)\widebar{\land}_0(aw) 
				&= (aw_{(11)})\otimes w_{(21)}\otimes w_{(2)}
				\quad\lcomment{$\widebar{\land}_0$の定義} \\
				&= (aw_{(1)})\otimes w_{(12)}\otimes w_{(22)}
				\quad\lcomment{$\Delta_*$の余結合性} \\
				&= (\myid\otimes \Delta_*)\Bigl((aw_{(1)})\otimes w_{(2)}\Bigr) \\
				&= (\myid\otimes \Delta_*)\widebar{\land}_0(aw)
				\quad\lcomment{$\widebar{\land}_0$の定義} \\
			\end{split}\end{equation*} %}
		\end{proof} %}

		命題\ref{prop:根への接木と単語の連結の関係}と
		命題\ref{prop:根での枝刈りと単語の連結の関係}を作用素の形で書くと
		次のようになる。
		\begin{equation*}\begin{split} %{
			\land_0(\land_0\otimes \myid) &= \land_0(\myid\otimes m) \\
			(\widebar{\land}_0\otimes \myid)\widebar{\land}_0 
			&= (\myid\otimes \Delta_*)\widebar{\land}_0 \\
		\end{split}\end{equation*} %}
		結合性と余結合性の定義によく似た形をしている。

		根に森を付け足して木にする$R$双線形な操作
		$\myop{tree}:RA\otimes RWT_+A\to RT_+A$は
		$\myop{tree}= \beta_{\land_0}(i_T\otimes\myid)$と書くことができる。

		次の例は、可換図\eqref{eq:行きがけ順を保つ二項演算}が
		二項演算$\land_0$を定めない例である。
		\begin{equation*}\begin{split} %{
			\mytree{
				& a_1 \ar@{-}[dl] \ar@{-}[d] \ar@{-}[dr] \\
				a_2 & a_3 & *+[F]{t} \\
			} \xrightarrow{\pi_W} [a_1a_2a_3] * (\pi_W t) \xleftarrow{\pi_W}
			\mytree{
				& a_1 \ar@{-}[dl] \ar@{-}[dr] \\
				a_2 && a_3 \ar@{-}[d] \\
				&& *+[F]{t}
			}
		\end{split}\end{equation*} %}
		頂点が$2$個以上の木では、行きがけ順で最後になるように頂点を付け加える
		方法は$2$通りある。一つの方法が、根の最右の子供として付け加える方法、
		もう一つの方法が、現在の行きがけ順で最後の頂点の子供として付け加える
		方法である。その両方の方法を足し合わせてしまうことを考える。

		\begin{todo}[足し合わせた場合]\label{todo:足し合わせた場合} %{
			どのような森の二項演算が導かれるか?
		\end{todo} %todo:足し合わせた場合}
	%s2:根への接木}

	\begin{todo}[ここまで]\label{todo:ここまで} %{
	\end{todo} %todo:ここまで}

	木$t\in TA$の頂点の集合を$\set{t}$と書く。
	$WTA=(TA,m_*,1_*)$を$TA$から生成された自由モノイドとする。
	$WTA$の元を$TA$の元を並べて括弧でくくって表すことにする。
	例えば、$t_1,t_2,\dots,t_m\in TA$に対して$[t_1t_2\cdots t_m]$と書く。
	$m_*$は文字列の連結で、$1_*$は空文字である。
	$WTA$の元を森と言うことにする。
	森$w\in WTA$の頂点の集合を$\set{w}$と書く。例えば、
	木$t_1,t_2,\dots,t_m\in TA$に対して$
	\set{[t_1t_2\cdots t_m]}=\set{t_1}\cup\set{t_2}\cup\cdots\cup\set{t_m}
	$となる。木から頂点を取り除いてできる森を木の森ということにする。
	例えば、$t_1,t_2,\dots,t_m\in TA$に対して、森$[t_1t_2\cdots t_m]$は、
	任意の$s\in S$を頂点とする木$s[t_1t_2\cdots t_m]$の森となる。

	写像$\myop{tree}$を次のように定義する。
	\begin{equation}\begin{split} %{
		\myop{tree}: S\times WTA &\to TA \\
		s\times w &\mapsto sw \\
	\end{split}\end{equation} %}
	写像$\myop{tree}$は集合同型となる。$\myop{tree}$の逆写像を
	$\myop{tree}^{^1}$書く。
	\begin{equation}\begin{split} %{
		\myop{tree}^{-1}: TA &\to S\times WTA \\
		sw &\mapsto s\times w \\
	\end{split}\end{equation} %}
	木から根を取り出す操作を$\myop{root}=\pi_1\myop{tree}^{-1}$、
	木から根の子供達を$\myop{forest}=\pi_2\myop{tree}^{-1}$と書く。

	$R=(R,+,0,\myspace,1)$を半環とする。
	$RS$を$S$を基底とする$R$係数半モジュール、
	$RTA$を$TA$を基底とする$R$係数半モジュール、
	$RWTA$を$WTA$を基底とする$R$係数半モジュールとする。
	文字列の連結$m_*:WTA\times WTA\to WTA$を$R$線形に$RWTA$に拡張したものを
	同一の記号$m_*$で書くことにする。積$m_*$に双対になる余積$\Delta_*$を次の
	ように定義する。
	\begin{equation}\begin{split} %{
		\Delta_*: RWTA\otimes RWTA &\to RWTA \\
			[t_1t_2\cdots t_m] &\mapsto [t_1t_2\cdots t_m]\otimes 1_* \\
				&\quad + \sum_{1\le i\le m}[t_1t_2\cdots t_m]\neg\set{i}\otimes [t_i] \\
				&\quad + \sum_{1\le i<j\le m}[t_1t_2\cdots t_m]\neg\set{i,j}\otimes [t_it_j] \\
				&\quad + \cdots \\
				&\quad + \sum_{1\le i<j\le m}[t_i]\otimes [t_1t_2\cdots t_m]\neg\set{i} \\
				&\quad + \sum_{1\le i<j\le m}1_*\otimes [t_1t_2\cdots t_m] \\
	\end{split}\end{equation} %}
	ここで、$[t_1t_2\cdots t_m]\neg{i_1,i_2,\dots,i_p}$は$[t_1t_2\cdots t_m]$
	から$i_1,i_2,\dots,i_p$番目の文字を除いた単語とする。
	例えば、次のようになる。
	\begin{equation*}\begin{split} %{
		[t_1t_2t_3]\neg\set{1} &= [t_2t_3] \\
		[t_1t_2t_3]\neg\set{1,3} &= [t_2] \\
		[t_1t_2t_3]\neg\set{1,2,3} &= [] \\
	\end{split}\end{equation*} %}
	余積$\Delta_*$に対する余単位射$\epsilon_*$は次のようになる。
	\begin{equation}\begin{split} %{
		\epsilon_*: RTW &\to R \\
		w &\mapsto \begin{cases} %{
			1, &\text{ iff }w=1_* \\
			0, &\text{ otherwise } \\
		\end{cases} %}
	\end{split}\end{equation} %}

	$RTA$に積を定義するための準備をする。

	\begin{definition}[頂点を指定した接木]\label{def:頂点を指定した接木} %{
		木$t\in TA$の頂点$i\in \set{t}$の最後の子供に木$t_1\in TA$を付け加える
		操作を$t_1$の頂点$i$への接木ということにする。
	\end{definition} %def:頂点を指定した接木}

	ラベルによらずに頂点の位置を表すために頂点を'頂点の位置:ラベル'
	という形で書いて、接木を図示すると次のようになる。
	\begin{equation*}\begin{split} %{
		\mytree{
			& 0:a_0 \ar@{-}[dl] \ar@{-}[dr] \\
			1:a_1 \ar@{-}[d] && 3:a_3 \\
			2:a_2 \\
		} \lhd_{1} \mytree{
			& r_0 \ar@{-}[dl] \ar@{-}[dr] \\
			r_1 && r_2 \\
		} &= \mytree{
			&& 0:a_0 \ar@{-}[dl] \ar@{-}[dr] \\
			& 1:a_1 \ar@{-}[dl] \ar@{-}[dr] && 3:a_3 \\
			2:a_2 && r_0 \ar@{-}[dl] \ar@{-}[dr] \\
			& r_1 && r_2 \\
		}
	\end{split}\end{equation*} %}

	すべての頂点にわたって接木をする操作を定義する。

	\begin{definition}[木への接木]\label{def:木への接木} %{
		木$t\in TA$の頂点$i\in \set{t}$への接木を$\lhd_i$と書く。
		次の$R$双線形写像$\lhd$を木への接木ということにする。
		\begin{equation}\begin{split} %{
			\lhd: RTA\otimes RTA &\to RTA \\
				t\otimes u &\mapsto \sum_{i\in \set{t}}t\lhd_{i}u
				\quad\text{for all }t,u\in TA
		\end{split}\end{equation} %}
	\end{definition} %def:木への接木}

	木への接木を図示すると次のようになる。
	\begin{equation*}\begin{split} %{
		\mytree{
			& a_0 \ar@{-}[dl] \ar@{-}[dr] \\
			a_1 \ar@{-}[d] && a_3 \\
			a_2 \\
		} \lhd \mytree{
			& r_0 \ar@{-}[dl] \ar@{-}[dr] \\
			r_1 && r_2 \\
		} &= \mytree{
			& a_0 \ar@{-}[dl] \ar@{-}[d] \ar@{-}[dr] \\
			a_1 \ar@{-}[d] & a_3 & r_0 \ar@{-}[dl] \ar@{-}[dr] \\
			a_2 & r_1 && r_2 \\
		} + \mytree{
			&& a_0 \ar@{-}[dl] \ar@{-}[dr] \\
			& a_1 \ar@{-}[dl] \ar@{-}[dr] && a_3 \\
			a_2 && r_0 \ar@{-}[dl] \ar@{-}[dr] \\
			& r_1 && r_2 \\
		} \\
		&+ \mytree{
			&& a_0 \ar@{-}[dl] \ar@{-}[dr] \\
			& a_1 \ar@{-}[d] && a_3 \\
			& a_2 \ar@{-}[d] \\
			& r_0 \ar@{-}[dl] \ar@{-}[dr] \\
			r_1 && r_2 \\
		} + \mytree{
			& a_0 \ar@{-}[dl] \ar@{-}[dr] \\
			a_1 \ar@{-}[d] && a_3 \ar@{-}[d] \\
			a_2 && r_0 \ar@{-}[dl] \ar@{-}[dr] \\
			& r_1 && r_2 \\
		}
	\end{split}\end{equation*} %}

	接木はの$R$双線形二項演算であるが、結合的ではないことに注意する。
	例えば次のようになって、結合性は満たさない。
	\begin{equation*}\begin{split} %{
		\bigl([a_1]\lhd[a_2]\bigr)\lhd[a_3] &= [a_1]\lhd\bigl([a_2]\lhd[a_3]\bigr)+[a_1[a_2][a_3]] \\
		[a_1]\lhd\bigl([a_2]\lhd[a_3]\bigr) &= [a_1[a_2[a_3]]] \\
	\end{split}\end{equation*} %}

	木への接木を森による接木に拡張する。

	\begin{definition}[森による木への接木]\label{def:森による木への接木} %{
		木$t\in TA$の頂点$i\in \set{t}$への接木を$\lhd_i$と書く。
		次の$R$双線形写像$\lhd$を森による木への接木または単に木への接木
		ということにする。
		\begin{equation}\begin{split} %{
			\lhd: RTA\otimes RWTA &\to RTA \\
				t\otimes 1_* &\mapsto t \quad\text{for all }t\in TA \\
				t\otimes [t_1t_2\cdots t_m] 
				&\mapsto \sum_{i_1,i_2,\dots,i_m\in\set{t}}t\lhd_{i_1}t_1\lhd_{i_2}t_2\cdots\lhd_{i_m}t_m \\
				&\quad\text{for all }t,t_1,t_2,\dots,t_m\in TA \\
		\end{split}\end{equation} %}
	\end{definition} %def:森による木への接木}

	木への接木の結合則からのずれが、森による木への接木によって表される。

	\begin{proposition}[接木の結合則もどき]\label{prop:接木の結合則もどき} %{
		任意の木$t_1,t_2,t_3\in TA$に対して次の式が成り立つ。
		\begin{equation}\begin{split} %{
			(t_1\lhd t_2)\lhd t_3 = t_1\lhd(t_2\lhd t_3) + t_1\lhd[t_2t_3]
		\end{split}\end{equation} %}
		ここで、左辺と右辺の第一項目の$\lhd$は木による接木で、右辺の第二項目は
		森による接木である。
	\end{proposition} %prop:接木の結合則もどき}
	\begin{proof} %{
		\begin{equation*}\begin{split} %{
			(t_1\lhd t_2)\lhd t_3 
			&= \sum_{i_3\in\set{t_1}\cup\set{t_2}}(t_1\lhd t_2)\lhd_{i_3} t_3 \\
			&= \left(\sum_{i_3\in\set{t_1}}+\sum_{i_3\in\set{t_2}}\right)(t_1\lhd t_2)\lhd_{i_3} t_3 \\
			&= \left(\sum_{i_2,i_3\in\set{t_1}}t_1\lhd_{i_2} t_2\lhd_{i_3} t_3\right)
			+ \bigl(t_1\lhd(t_2\lhd t_3)\bigr) \\
		\end{split}\end{equation*} %}
		右辺の第一項目は森による接木の定義により$t_1\lhd[t_2t_3]$となり、
		命題が成り立つ。
	\end{proof} %}

	森による木への接木は根を不変に保つので、次の式により$R$双線形二項演算
	$\perp$を定義することができる。
	\begin{equation}\label{eq:接木による森の二項演算の定義}\begin{split} %{
		(aw_1)\lhd w_2 = a(w_1\perp w_2)\quad\text{for all }a\in A,\;w_1,w_2\in WTA
	\end{split}\end{equation} %}
	$\perp$は根のラベル$a$には依存しない。つまり、
	$(aw_1)\lhd w_2=a(w_1\perp_aw_2)$、$(bw_1)\lhd w_2=b(w_1\perp_bw_2)$、
	とおいたとき、任意の$w_1,w_2\in WTA$に対して
	$(w_1\perp_aw_2)=(w_1\perp_bw_2)$となる。

	\begin{definition}[接木による森の二項演算]\label{def:接木による森の二項演算} %{
		$R$双線形写像$\perp:RTWA\otimes RTWA\to RTWA$を次のように定義する。
		\begin{equation}\begin{split} %{
			(aw_1)\lhd w_2 = a(w_1\perp w_2)\quad\text{for all }a\in A,\;w_1,w_2\in WTA
		\end{split}\end{equation} %}
	\end{definition} %def:接木による森の二項演算}

	前置記号を用いると$
	\mybiop{\lhd}=\myop{tree}(\myid\otimes\mybiop{\perp})
	(\myop{tree}^{-1}\otimes\myid)
	$となる。

	$\perp$を計算してみる。まず、任意の$a\in A,\;w\in WTA$に対して
	次の式が成り立つ。
	\begin{equation}\label{eq:接木による積の定義その一}\begin{split} %{
		(a[])\lhd w=aw &\implies 1_*\perp w=w \\
		(aw)\lhd 1_*=aw &\implies w\perp 1_*=w \\
	\end{split}\end{equation} %}
	また、任意の$a\in A,\;t_1,t_2,\dots,t_m\in TA$と任意の$1_*$でない
	$w\in WTA$に対しては次の式が成り立つ。
	\begin{equation}\begin{split} %{
		(aw)\lhd[t_1t_2\cdots t_m]
			&= \sum_{i_1,i_2,\dots,i_m\in\set{aw}}(aw)
			\lhd_{i_1}t_1\lhd_{i_2}t_2\cdots\lhd_{i_m}t_m \\
			%
			&= (aw)\lhd_{\set{a}}[t_1t_2\cdots t_m] \\
			&\;+ \sum_{1\le i\le m}(aw)
			\lhd_{\set{a}}[t_1t_2\cdots t_m]_{\neg\set{i}}\lhd_{\set{w}}[t_i] \\
			&\;+ \sum_{1\le i<j\le m}(aw)
			\lhd_{\set{a}}[t_1t_2\cdots t_m]_{\neg\set{i,j}}\lhd_{\set{w}}[t_it_j] \\
			&\;+ \cdots \\
			&\;+ (aw)\lhd_{\set{w}}[t_1t_2\cdots t_m] \\
	\end{split}\end{equation} %}
	ここで、木$t\in TA$の頂点の部分集合$v\in\set{t}$に対して
	$t\lhd_v[t_1t_2\cdots t_m]$を次のようにおいた。
	\begin{equation}\begin{split} %{
		t\lhd_v[t_1t_2\cdots t_m] 
		&= \sum_{i_1,i_2,\dots,i_m\in v}t\lhd_{i_1}t_1\lhd_{i_2}t_2\cdots\lhd_{i_m}t_m
	\end{split}\end{equation} %}
	したがって、$\perp$は任意の$t,t_1,t_2,\dots,t_m\in TA$と
	任意の$1_*$でない$w\in WTA$に対して次のようになることがわかる。
	\begin{equation}\label{eq:接木による積の定義その二}\begin{split} %{
		w\perp[t_1t_2\cdots t_m]
		&= w\lhd[t_1t_2\cdots t_m] \\
		&\;+ \sum_{1\le i\le m}(w\lhd[t_1t_2\cdots t_m]_{\neg\set{i}})*[t_i] \\
		&\;+ \sum_{1\le i<j\le m}(w\lhd[t_1t_2\cdots t_m]_{\neg\set{i,j}})*[t_it_j] \\
		&\;+ \cdots \\
		&\;+ w*[t_1t_2\cdots t_m] \\
	\end{split}\end{equation} %}
	ここで、$w_1,w_2\in WTA$に対して$w_1\lhd w_2$を$w_1$のすべての頂点に
	接木をした和とした。任意の$t\in TA$、$w\in WTA$に対して
	$[t]\lhd w=[t\lhd w]$となり、
	任意の$t,u_1,u_2,\dots,u_m\in TA$、任意の$1_*$でない$w\in WTA$に対して
	次の再帰式を満たす。
	\begin{equation*}\begin{split} %{
		&([t]*w)\lhd [u_1u_2\cdots u_m] \\
		&= ([t]\lhd 1_*)*(w\lhd [u_1u_2\cdots u_m]) \\
		&\;+ \sum_{1\le i\le m}([t]\lhd[u_i])*(w\lhd [u_1u_2\cdots u_m]_{\neg\set{i}}) \\
		&\; + \sum_{1\le i<j\le m}([t]\lhd[u_iu_j])*(w\lhd [u_1u_2\cdots u_m]_{\neg\set{i,j}}) \\
		&\; + \cdots \\
		&\; + ([t]\lhd [u_1u_2\cdots u_m])*(w\lhd 1_*) \\
	\end{split}\end{equation*} %}
	この式は、任意の$1_*$でない$w_1,w_2\in WTA$と任意の$w_3\in WTA$に対する
	次の再帰式にまとまる。
	\begin{equation}\label{eq:森同士の接木}\begin{split} %{
		(w_1*w_2)\lhd w_3 = \left(w_1\lhd (\Delta_*^{(1)}w_3)\right)*\left(w_1\lhd (\Delta_*^{(2)}w_3)\right) \\
	\end{split}\end{equation} %}
	さらに、任意の$w\in WTA$に対して$1_*\lhd w=\jump{w=1_*}1_*$と定義すると、
	式\eqref{eq:森同士の接木}は、任意の$t\in TA,\;w\in WTA$に対して
	$[t]\lhd w=[t\lhd w]$を満たす。改めて、森への接木$\lhd$を定義しておく。

	\begin{definition}[森への接木]\label{def:森への接木} %{
		木$t\in TA$の頂点$i\in \set{t}$への接木を$\lhd_i$と書く。
		次の$R$双線形写像$\lhd:RWTA\otimes RWTA\to RWTA$を森による森への接木
		または単に森への接木ということにする。
		\begin{itemize} %{
			\item 任意の$w\in WTA$に対して
			\begin{equation*}\begin{split} %{
				1_*\lhd w = \begin{cases} %{
					1_*, &\text{ iff }w=1_* \\
					0, &\text{ otherwise } \\
				\end{cases} %}
			\end{split}\end{equation*} %}
			\item 任意の$t\in TA,\;w\in WTA$に対して
			\begin{equation*}\begin{split} %{
				[t]\lhd w = [t\lhd w] 
			\end{split}\end{equation*} %}
			\item 任意の$w_1,w_2\in WTA$に対して
			\begin{equation*}\begin{split} %{
				(w_1*w_2)\lhd w 
				&= \left(w_1\lhd(\Delta_*^{(1)}w)\right) * \left(w_2\lhd(\Delta_*^{(2)}w)\right) \\
				&= m_*\bigl((w_1\otimes w_2)\lhd(\Delta_*w)\bigr) \\
			\end{split}\end{equation*} %}
		\end{itemize} %}
	\end{definition} %def:森への接木}

	$\perp$の定義\eqref{eq:接木による積の定義その一}と
	\eqref{eq:接木による積の定義その二}は、任意の森$w_1,w_2\in WTA$に対して、$
	w_1\perp w_2=\left(w_1\lhd(\Delta_*^{(2)}w_2)\right)*(\Delta_*^{(1)}w_2)
	$と書かれることがわかる。
	森への接木\ref{def:森への接木}では、空文字への接木以外は
	木への接木\ref{def:森による木への接木}を自然に拡張したものになっている。
	空文字への接木に自然な拡張がないので、式$
	w_1\perp w_2=\left(w_1\lhd(\Delta_*^{(2)}w_2)\right)*(\Delta_*^{(1)}w_2)
	$が成り立つように空文字への接木$1_*\lhd1_*=\jump{w=1_*}1_*$を定義している。
	木への接木\ref{def:木への接木}が結合的でないのと同様に、森への接木
	\ref{def:森への接木}も結合的でない。
	その一方で、森への接木は右単位元$1_*$をもつことに注意する。

	\begin{proposition}[森の二項演算と森への接木の関係]\label{prop:森の二項演算と森への接木の関係} %{
		任意の$w_1,w_2\in WTS$に対して次の式が成り立つ。
		\begin{equation}\begin{split} %{
			w_1\perp w_2 
			= \left(w_1\lhd(\Delta_*^{(2)}w_2)\right)*(\Delta_*^{(1)}w_2)
		\end{split}\end{equation} %}
	\end{proposition} %prop:森の二項演算と森への接木の関係}
	\begin{proof} %{
		命題の式が成り立つように森への接木$\lhd$を定義した。
	\end{proof} %}

	$1_*$が接木による森の二項演算$\perp$の単位元になることはすぐわかる。

	\begin{proposition}[接木による森の二項演算の単位元]\label{prop:接木による森の二項演算の単位元} %{
		空文字$1_*$は接木による森の二項演算$\perp$の単位元となる。
	\end{proposition} %prop:接木による森の二項演算の単位元}
	\begin{proof} %{
		式\eqref{eq:接木による積の定義その一}によって証明は完了するが、
		ここでは、命題\ref{prop:森の二項演算と森への接木の関係}を用いて
		証明してみる。
		$\Delta_*1_*=1_*\otimes 1_*$より、任意の$w\in WTA$に対して
		$w\perp1_*=(w\lhd1_*)*1_*$となり、森への接木の定義\ref{def:森への接木}
		より、$w\lhd1_*=w$となるから$w\perp1_*=w$となる。
		ここで、$i,j=1,2$に対して$w^{(i)}=\Delta_*^{(j)}w$とする。
		$1_*\perp w=(1_*\lhd w^{(2)})*w^{(1)}$となり、
		森への接木の定義\ref{def:森への接木}より、$1_*w=(\epsilon_*w)1_*$
		となるから、$1_*\perp w=(\epsilon_*w^{(2)})w^{(1)}$となる。
	\end{proof} %}

	接木の結合則もどきの命題\ref{prop:接木の結合則もどき}を拡張することを
	考える。森の二項演算$\perp$の定義\ref{def:接木による森の二項演算}は、
	任意の$a\in A,\;w_1,w_2\in WTA$に対して次のように書き換えることができる。
	\begin{equation*}\begin{split} %{
		([a]\lhd w_1)\lhd w_2 = [a]\lhd(w_1\perp w_2)
	\end{split}\end{equation*} %}
	根だけの木$[a]$から一般の森に変更することで次の命題が得られる。

	\begin{proposition}[森への接木の結合則もどき]\label{prop:森への接木の結合則もどき} %{
		任意の森$w_1,w_2,w_3\in TA$に対して次の式が成り立つ。
		\begin{equation}\begin{split} %{
			(w_1\lhd w_2)\lhd w_3 = w_1\lhd(w_2\perp w_3)
		\end{split}\end{equation} %}
	\end{proposition} %prop:森への接木の結合則もどき}
	\begin{proof} %{
		$\perp$の定義\ref{eq:接木による森の二項演算の定義}から
		命題\ref{prop:森の二項演算と森への接木の関係}を導いた手順を繰り返す
		ことで証明が得られる。まず、任意の$w_1,w_2\in WTA$に対して、$1_*$が
		$\perp$の単位元だから次の式が成り立つ。
		\begin{equation*}\begin{split} %{
			(w_1\lhd w_2)\lhd 1_*=w_1\lhd w_2=w_1\lhd(w_2\perp 1_*)
		\end{split}\end{equation*} %}
		次に、任意の$w_1,w_2\in WTA$と$t_1,t_2,\dots,t_m\in TA$に対して、
		次の式が成り立つ。
		\begin{equation*}\begin{split} %{
			&(w_1\lhd w_2)\lhd[t_1t_2\cdots t_m] \\
			&= \sum_{i_1,i_2,\dots,i_m\in\set{w_1}\cup\set{w_1}}(w_1\lhd w_2)
			\lhd_{i_1}t_1\lhd_{i_2}t_2\cdots\lhd_{i_m}t_m \\
			&= (w_1\lhd w_2)\lhd_{\set{w_1}}[t_1t_2\cdots t_m] \\
			&\;+ \sum_{1\le i\le m}(w_1\lhd w_2)
			\lhd_{\set{w_1}}[t_1t_2\cdots t_m]_{\neg\set{i}}\lhd_{\set{w_2}}[t_i] \\
			&\;+ \sum_{1\le i<j\le m}(w_1\lhd w_2)
			\lhd_{\set{w_1}}[t_1t_2\cdots t_m]_{\neg\set{i,j}}\lhd_{\set{w_2}}[t_it_j] \\
			&\;+ \cdots \\
			&\;+ (w_1\lhd w_2)\lhd_{\set{w_2}}[t_1t_2\cdots t_m] \\
			&= w_1\lhd(w_2*[t_1t_2\cdots t_m]) \\
			&\;+ \sum_{1\le i\le m}w_1
			\lhd\bigl((w_2\lhd[t_i])*[t_1t_2\cdots t_m]_{\neg\set{i}}\bigr) \\
			&\;+ \sum_{1\le i<j\le m}w_1
			\lhd\bigl((w_2\lhd[t_it_j])*[t_1t_2\cdots t_m]_{\neg\set{i,j}}\bigr) \\
			&\;+ \cdots \\
			&\;+ w_1\lhd(w_2\lhd[t_1t_2\cdots t_m]) \\
			&= w_1\lhd\Bigl(\bigl(w_1\lhd(\Delta_*^{(2)}[t_1t_2\cdots t_m])\bigr)*(\Delta_*^{(1)}[t_1t_2\cdots t_m])\Bigr) \\
			&= w_1\lhd(w_1\perp[t_1t_2\cdots t_m]) \\
		\end{split}\end{equation*} %}
		したがって、命題が証明された。
	\end{proof} %}

	$\Delta_*w=w\otimes 1_*+1_*\otimes w+r$とおくと、
	命題\ref{prop:森への接木の結合則もどき}から、$r$に関する$0$次近似が
	木への接木の場合\ref{prop:接木の結合則もどき}と同じ形になる次の式
	が得られる。
	\begin{equation*}\begin{split} %{
		(w_1\lhd w_2)\lhd w_3 = w_1\lhd(w_2\lhd w_3) + w_1\lhd(w_2*w_3) + (\text{order}\;r)
	\end{split}\end{equation*} %}

	\begin{proposition}[接木による森の二項演算は積]\label{prop:接木による森の二項演算は積} %{
		接木による森の二項演算$\perp$は積である。
		\begin{equation}\begin{split} %{
			(w_1\perp w_2)\perp w_3 = w_1\perp(w_2\perp w_3) \quad\text{for all }w_1,w_2,w_3\in WTA
		\end{split}\end{equation} %}
	\end{proposition} %prop:接木による森の二項演算は積}
	\begin{proof} %{
		任意の$a\in A,w,w_1,w_2\in WTA$に対して次の式が成り立つ。
		\begin{equation*}\begin{split} %{
			([aw]\lhd w_2)\lhd w_3 
			&= ((aw)\lhd w_2)\lhd w_3 \quad \lcomment{$\lhd$の定義} \\
			&= (a(w\perp w_2))\lhd w_3 \quad \lcomment{$\perp$の定義} \\
			&= a((w\perp w_2)\perp w_3) \quad \lcomment{$\perp$の定義} \\
			%
			([aw])\lhd(w_2\perp w_3)
			&= (aw)\lhd(w_2\perp w_3) \quad \lcomment{$\lhd$の定義} \\
			&= a(w\perp(w_2\perp w_3)) \quad \lcomment{$\perp$の定義} \\
		\end{split}\end{equation*} %}
		この結果を結合律もどき(命題\ref{prop:森への接木の結合則もどき})
		\begin{equation*}\begin{split} %{
			(w_1\lhd w_2)\lhd w_3 = w_1\lhd(w_2\perp w_3) \quad\text{for all }w_1,w_2,w_3\in WTA
		\end{split}\end{equation*} %}
		に当てはめると、$\perp$の結合律が導かれる。
		\begin{equation*}\begin{split} %{
			(w\perp w_2)\perp w_3 = w\perp(w_2\perp w_3) \quad\text{for all }w,w_1,w_2\in WTA
		\end{split}\end{equation*} %}
	\end{proof} %}

	\subsection{考察}\label{s2:考察} %{
		ここまでの議論を振り返ってみる。木ということを忘れて文字列で考えてみる。
		$A$を集合、$R$を半環とする。
		$WA$を自由モノイドから生成された自由モノイド、$RWA$を
		$WA$を基底とする$R$係数自由半モジュールとする。
		$RWA$の$n$文字だけからなる部分空間を$RW_nA$とする。$RWA$は直和分解できて
		$RWA=\oplus_{n=0}^\infty RW_nA$となる。$WA$の文字列の連結を$m_*$、
		その双対で$W_1A$の元をプリミティブとする余積を$\Delta_*$とする。

		$X$を集合、$RX$を$X$を基底とする$R$係数自由半モジュールとする。
		$RWA$の$RX$への$R$双線形作用$\lhd:RA\otimes RWX\to RX$と
		$RWA$の$R$双線形二項演算$\perp$が定義されていて、次の性質を持つとする。
		\begin{description} %{
			\item[単位性]ある$1_\perp$が存在して、任意の$x\in X$に対して
			次の式が成り立つ。
			\begin{equation*}\begin{split} %{
				x\lhd 1_\perp x = x 
			\end{split}\end{equation*} %}
			\item[結合性]任意の$w_1,w_2\in WA,\; x\in X$に対して次の式が成り立つ。
			\begin{equation*}\begin{split} %{
				x\lhd w_1\lhd w_2 = x\lhd (w_1\perp w_2)
			\end{split}\end{equation*} %}
			\item[キャンセル可能性]任意の$w_1,w_2\in WA,\; x\in X$に対して
			次の式が成り立つ。
			\begin{equation*}\begin{split} %{
				x\lhd w_1 = x\lhd w_2 \implies w_1=w_2
			\end{split}\end{equation*} %}
		\end{description} %}
		
		任意の$w_1,w_2,w_3\in WA,\; x\in X$に対して、内側から'結合性'を適用
		していくことによって、次の式が得られる。
		\begin{equation*}\begin{split} %{
			x\lhd w_1\lhd w_2\lhd w_3
			&= x\lhd (w_1\perp w_2)\lhd w_3 \\
			&= x\lhd \bigl((w_1\perp w_2)\perp w_3)\bigr) \\
		\end{split}\end{equation*} %}
		外側から'結合性'を適用していくことによって、次の式が得られる。
		\begin{equation*}\begin{split} %{
			x\lhd w_1\lhd w_2\lhd w_3
			&= x\lhd w_1\lhd (w_2\perp w_3) \\
			&= \bigl(w_1\perp (w_2\perp w_3)\bigr) \\
		\end{split}\end{equation*} %}
		'キャンセル可能性'を適用すると、$\perp$の結合性が導かれる。
		\begin{equation*}\begin{split} %{
			(w_1\perp w_2)\perp w_3 = w_1\perp (w_2\perp w_3) 
		\end{split}\end{equation*} %}
		また、任意の$w\in WA,\; x\in X$に対して、'単位性'と'結合性'を適用
		すると、次の式が導かれ、
		\begin{equation*}\begin{split} %{
			x\lhd w\lhd 1_\perp
			&= x\lhd w \quad\lcomment{'単位性'} \\
			&= x\lhd (w\perp 1_\perp) x \quad\lcomment{'結合性'} \\
		\end{split}\end{equation*} %}
		'キャンセル可能性'を適用すると、$1_\perp$が$\perp$の右単位元になることが
		導かれる。同様に、次の式が導かれ、
		\begin{equation*}\begin{split} %{
			x\lhd 1_\perp\lhd w
			&= x\lhd w \quad\lcomment{'単位性'} \\
			&= x\lhd (1_\perp\perp w) \quad\lcomment{'結合性'} \\
		\end{split}\end{equation*} %}
		'キャンセル可能性'を適用すると、$1_\perp$が$\perp$の左単位元になることが
		導かれる。以上より、作用$\rhd$から半代数$(RWA,\perp,1_\perp)$の$RX$への
		表現が得られることがわかる。

		\subsubsection{木の場合の考察}\label{s3:木の場合の考察} %{
			$X$を集合、$R$を半環とする。
			$TX$を$X$を頂点とする木の集合、$T_{\bullet}X$を$\bullet$を根、
			$X$を根以外の頂点とする木の集合とする。
			$X$の$RT_{\bullet}X$への作用$
			\beta_\lhd:RT_\bullet X\otimes RX\to RT_\bullet X
			$を次のように定義する。
			\begin{equation}\label{eq:頂点の作用の定義}\begin{split} %{
				t\lhd x &= \sum_{i\in\set{t}}t\lhd_i x \\
				t\lhd_i x &= \text{$t$の頂点$i$の最右の子供として$x$を付け足す}
			\end{split}\end{equation} %}
			例えば、$x_1,x_2,\dots\in X$として、次のようになる。
			\begin{equation}\label{eq:頂点による接木の例}\begin{split} %{
				\bullet\lhd x_1 &= \bullet[x_1] \\
				\bullet[x_1]\lhd x_2 &= \bullet[x_1x_2] + \bullet[x_1[x_2]] \\
				\bullet[x_1x_2]\lhd x_3 &= \bullet[x_1x_2x_3] + \bullet[x_1[x_3]x_2] + \bullet[x_1x_2[x_3]] \\
				\bullet[x_1[x_2]]\lhd x_3 &= \bullet[x_1[x_2]x_3] + \bullet[x_1[x_2x_3]] + \bullet[x_1[x_2[x_3]]] \\
				\dots \\
			\end{split}\end{equation} %}

			'結合性'\;$t\lhd x_1\lhd x_2=t\lhd(x_1\perp x_2)$から積$\perp$を
			定義することを考える。
			\begin{equation}\label{eq:接木のフェチその一}\begin{split} %{
				\bullet\lhd x_1 &= \bullet[x_1] \\
				\bullet\lhd(x_1\perp x_2) 
				&\sim \bullet\lhd\Bigl([x_1x_2] + [x_1[x_2]]\Bigr) \\
				\bullet(x_1\perp x_2\perp x_3) 
				&\sim \bullet\lhd\Bigl([x_1x_2x_3]+[x_1[x_3]x_2]+[x_1x_2[x_3]] \\
				&\;+ [x_1[x_2]x_3]+[x_1[x_2x_3]]+[x_1[x_2[x_3]]]\Bigr) \\
				\dots \\
			\end{split}\end{equation} %}
			$RX$では二項演算$\perp$は定義できないが、
			式\eqref{eq:接木のフェチその一}の右辺が木の単語の和になっているので、
			$RWTX$なら$\perp$が定義できそうである。したがって、
			$X$からへ$TX$の入射$i_T:x\mapsto[x]$と$TX$からへ$WTX$の入射
			$i_W:t\mapsto[t]$を用いて、次の可換図によりそれぞれの$RT_\bullet X$
			への作用を定義することを考える。
			\begin{equation*}\xymatrix{
				RX \ar[rd]_{\beta_\lhd} \ar[r]^{i_T} 
				& RTX \ar@{.>}[d]^{\beta_\lhd} \ar[r]^{i_W}
				& RWTX \ar@{.>}[ld]^{\beta_\lhd} \\
				& \myop{end}RT_\bullet X
			}\end{equation*}

			まず、作用$\beta_\lhd:RT_\bullet X\otimes RTX\to RT_\bullet X$を
			定義することを考える。
			$X$の$RT_{\bullet}X$への作用の定義\eqref{eq:頂点の作用の定義}で
			付け足す頂点を付け足す木に拡張して、$TX$の$RT_{\bullet}X$への作用$
			\beta_\lhd:RT_\bullet X\otimes RTX\to RT_\bullet X$を次のように
			定義する。
			\begin{equation}\label{eq:木の作用の定義}\begin{split} %{
				t\lhd u &= \sum_{i\in\set{t}}t\lhd_i u 
				\quad\text{for all }t\in T_\bullet X,\; u\in TX \\
				t\lhd_i u &= \text{$t$の頂点$i$の最右の子供として$u$を付け足す}
			\end{split}\end{equation} %}
			例えば、$t_1,t_2,\dots\in TX$として、次のようになる。
			\begin{equation}\label{eq:木による接木の例}\begin{split} %{
				\bullet\lhd t_1 &= \bullet[t_1] \\
				\bullet[t_1]\lhd t_2 &= \bullet[t_1t_2] + \bullet[t_1\unlhd t_2] \\
				\bullet[t_1t_2]\lhd t_3 &= \bullet[t_1t_2t_3] + \bullet[(t_1\unlhd t_3)t_2] + \bullet[t_1(t_2\unlhd t_3)] \\
				\bullet[t_1\unlhd t_2]\lhd t_3 &= \bullet[(t_1\unlhd t_2)t_3] + \bullet[(t_1\unlhd t_2)\unlhd t_3] \\
				\dots \\
			\end{split}\end{equation} %}
			ここで、$R$双線形写像$\beta_\unlhd:RTX\otimes RTX\to RTX$を次のように
			定義した。
			\begin{equation}\label{eq:木の二項演算の定義}\begin{split} %{
				t_1\unlhd t_2 &= \sum_{i\in\set{t_1}}t_1\unlhd_i t_2
				\quad\text{for all }t_1,t_2\in TX \\
				t_1\unlhd_i t_2 &= \text{$t_1$の頂点$i$の最右の子供として$t_2$を付け足す}
			\end{split}\end{equation} %}
			$\lhd$の定義\ref{eq:木の作用の定義}と$\unlhd$の定義\ref{eq:木の二項演算の定義}
			は作用か二項演算かの違いだけで、行う操作はまったく同じである。
			したがって、ここで定義した$\unlhd$も同じ記号$\lhd$で書くことにする。
			ただし、$\lhd$を$RTX\otimes RTX\to RTX$の意味で使う場合は演算の順序
			が問題となる。演算順序の指定をしないで書いた場合は、左から演算する
			ものとする。例えば、$t_1\lhd t_2\lhd t_3:=(t_1\lhd t_2)\lhd t_3$
			である。それ以外の場合は、$t_1\lhd(t_2\lhd t_3)$のように括弧に
			よって演算順序を明示的に指定する。

			次に、作用$\beta_\lhd:RT_\bullet X\otimes RWTX\to RT_\bullet X$を
			定義することを考える。$WTX$の元を森ということにする。
			木の場合の例\eqref{eq:木による接木の例}を森の作用に拡張すると
			次のようになる。
			\begin{equation}\label{eq:森による接木の例その一}\begin{split} %{
				\bullet\lhd[t_1] &= \bullet[t_1] \\
				\bullet[t_1]\lhd[t_2] &= \bullet[t_1t_2] + \bullet[t_1\lhd t_2] \\
				\bullet[t_1t_2]\lhd[t_3] &= \bullet[t_1t_2t_3] + \bullet[(t_1\lhd t_3)t_2] + \bullet[t_1(t_2\lhd t_3)] \\
				\bullet[t_1\lhd t_2]\lhd[t_3] &= \bullet[(t_1\lhd t_2)t_3] + \bullet[t_1\lhd t_2\lhd t_3] \\
				\dots \\
			\end{split}\end{equation} %}
			さらに、次のようにを定義すると
			\begin{equation*}\begin{split} %{
				\bullet\lhd1_* &= \bullet \\
				\bullet\lhd[t_1t_2\cdots t_m] &= \bullet[t_1t_2\cdots t_m] \\
			\end{split}\end{equation*} %}
			式\eqref{eq:森による接木の例その一}は次のようになる。
			\begin{equation*}\begin{split} %{
				\bullet\lhd[t_1]\lhd[t_2] 
				&= \bullet\lhd\Bigl([t_1t_2]+[t_1\lhd t_2]\Bigr) \\
				\bullet\lhd[t_1t_2]\lhd[t_3] 
				&= \bullet\lhd\Bigl([t_1t_2t_3]+[(t_1\lhd t_3)t_2]+[t_1(t_2\lhd t_3)]\Bigr) \\
				\bullet\lhd[t_1\lhd t_2]\lhd[t_3] 
				&= \bullet\lhd\Bigl([(t_1\lhd t_2)t_3]+[t_1\lhd t_2\lhd t_3]\Bigr) \\
				\dots \\
			\end{split}\end{equation*} %}
			任意の$w_1,w_2\in WTX$に対して$
			\bullet\lhd w_1=\bullet\lhd w_2 \implies w_1=w_2
			$が成り立つから、次のように$R$双線形二項演算$
			\beta_\perp:RWTX\otimes RWTX\to RWTX
			$が定義される。
			\begin{equation*}\begin{split} %{
				[t_1]\perp[t_2] &= [t_1t_2]+[t_1\lhd t_2] \\
				[t_1t_2]\perp[t_3] 
				&= [t_1t_2t_3]+[(t_1\lhd t_3)t_2]+[t_1(t_2\lhd t_3)] \\
				[t_1\lhd t_2]\perp[t_3]
				&= [(t_1\lhd t_2)t_3]+[t_1\lhd t_2\lhd t_3] \\
				\dots \\
			\end{split}\end{equation*} %}
			さらに、$\lhd$の定義\eqref{eq:木の二項演算の定義}より
			\begin{equation*}\begin{split} %{
				t_1\lhd t_2\lhd t_3 
				&= \sum_{i\in\set{t_1}\cup\set{t_2}}t_1\lhd t_2\lhd_{i}t_3 \\
				&= \Bigl(\sum_{i\in\set{t_1}}+\sum_{i\in\set{t_2}}\Bigr)t_1\lhd t_2\lhd_{i}t_3 \\
				&= \Bigl(\sum_{i\in\set{t_1}}t_1\lhd t_2\lhd_{i}t_3\Bigr) + t_1\lhd(t_2\lhd t_3) \\
			\end{split}\end{equation*} %}
			となるから、$
			t_1\lhd[t_2t_3] = \sum_{i\in\set{t_1}}t_1\lhd t_2\lhd_{i}t_3
			$と定義すれば、
			\begin{equation*}\begin{split} %{
				t_1\lhd t_2\lhd t_3 &= t_1\lhd[t_2t_3] + t_1\lhd(t_2\lhd t_3)
			\end{split}\end{equation*} %}
			となる。


			here

			until

			を用いて$\lhd$を$
			\lhd:RT_\bullet X\otimes RWTX \to RT_\bullet X
			$に拡張して、$
			t\lhd([x_1]\perp[x_2])=t\lhd[x_1]\lhd[x_2]
			$により$RTX$の積$m_\perp$を定義する。例えば次のようになる。
			\begin{equation*}\begin{split} %{
				\bullet\lhd([x_1]\perp[x_2])
				&= \bullet[x_1x_2] + \bullet[x_1[x_2]] \\
				%
				\bullet\lhd([x_1]\perp[x_2]\perp[x_3]) 
				&= \bullet[x_1x_2x_3] + \bullet[x_1[x_3]x_2] + \bullet[x_1x_2[x_3]] \\
				&\;+ \bullet[x_1[x_2]x_3] + \bullet[x_1[x_2x_3]] + \bullet[x_1[x_2[x_3]]] \\
				\dots \\
			\end{split}\end{equation*} %}
			可換図で書くと、次の図を可換にする$\beta_\lhd$を求めることになる。

			$\bullet\in T_\bullet X$に対する$TX$の作用$\lhd$を次のように定義すれば、
			\begin{equation*}\begin{split} %{
				\bullet\lhd1_* &= \bullet \\
				\bullet\lhd[t_1t_2\cdots t_m]&= \bullet[t_1t_2\cdots t_m] \\
			\end{split}\end{equation*} %}
			任意の$w\in WTX$に対して$\bullet\lhd w\neq0$だから、次のようになる。
			\begin{equation*}\begin{split} %{
				[x_1]\perp[x_2] &= [x_1x_2] + [x_1[x_2]] \\
				[x_1]\perp[x_2]\perp[x_3]
				&= [x_1x_2x_3]+[x_1[x_3]x_2]+[x_1x_2[x_3]] \\
				&\;+ [x_1[x_2]x_3]+[x_1[x_2x_3]]+[x_1[x_2[x_3]]] \\
				\dots \\
			\end{split}\end{equation*} %}

until

			$T_{\bullet}X$へ$WX$の作用$\lhd$を次のように定義する。
			\begin{equation*}\begin{split} %{
				\bullet\lhd1_* &= \bullet \\
				\bullet\lhd[x_1] &= \bullet[x_1] \\
				\bullet[x_1]\lhd[x_2] &= \bullet[x_1x_2] + \bullet[x_1[x_2]] \\
				\bullet[x_1x_2]\lhd[x_3] &= \bullet[x_1x_2x_3] + \bullet[x_1[x_3]x_2] + \bullet[x_1x_2[x_3]] \\
				\bullet[x_1[x_2]]\lhd[x_3] &= \bullet[x_1[x_2]x_3] + \bullet[x_1[x_2x_3]] + \bullet[x_1[x_2[x_3]]] \\
				\dots \\
				\bullet\lhd[x_1x_2\cdots x_m] &= \bullet[x_1x_2\cdots x_m] \\
			\end{split}\end{equation*} %}
			'結合性'$t\lhd w_1\lhd w_2=t\lhd(w_1\perp w_2)$は$RWTX$に対して
			次の積を導く。
			\begin{equation*}\begin{split} %{
				[x_1]\perp[x_2] &= [x_1x_2] + [x_1[x_2]] \\
				[x_1x_2]\perp[x_3] &= [x_1x_2x_3] + [x_1[x_3]x_2] + [x_1x_2[x_3]] \\
				[x_1[x_2]]\perp[x_3] &= [x_1[x_2]x_3] + [x_1[x_2x_3]] + [x_1[x_2[x_3]]] \\
				\dots \\
			\end{split}\end{equation*} %}
			また、$\perp$の結合性から次の関係が得られる。
			\begin{equation*}\begin{split} %{
				[x_1]\perp[x_2]\perp[x_3] &= [x_1x_2]\perp[x_3] + [x_1[x_2]]\perp[x_3] \\
				&\Downarrow \quad\lcomment{十分条件} \\
				[x_1]\perp[x_2x_3] &= [x_1x_2x_3] + [x_1[x_3]x_2] + [x_1[x_2]x_3] + [x_1[x_2x_3]] \\
				[x_1]\perp[x_2[x_3]] &= [x_1x_2[x_3]] + [x_1[x_2[x_3]]] \\
				\dots \\
			\end{split}\end{equation*} %}
			$[x]\mapsto[x]\otimes1_*+1_*\otimes[x]$となる$\perp$に双対な余積
			$\Delta$は次のように与えられる。
			\begin{equation*}\begin{split} %{
				\Delta([x_1]\otimes[x_2]) 
				&= (\Delta[x_1])\perp(\Delta[x_2]) \\
				&\Downarrow \quad\lcomment{十分条件} \\
				\Delta[x_1x_2] &= \begin{pmatrix}
				[x_1x_2] \\
				1_*
				\end{pmatrix} + \begin{pmatrix}
				[x_1] \\
				[x_2]
				\end{pmatrix} + \begin{pmatrix}
				[x_2] \\
				[x_1]
				\end{pmatrix} + \begin{pmatrix}
				[x_1x_2] \\
				1_*
				\end{pmatrix} \\
				\Delta[x_1[x_2]] &= \begin{pmatrix}
				[x_1[x_2]] \\
				1_*
				\end{pmatrix} + \begin{pmatrix}
				1_* \\
				[x_1[x_2]]
				\end{pmatrix} \\
			\end{split}\end{equation*} %}
			余積$\Delta$は余積$\Delta_*$に他ならない。

			積$m_\perp$は積$m_*+\cdots$となっている。$m_*$の残りの部分を
			$\beta_\dashv$とする。
			\begin{equation*}\begin{split} %{
				m_\perp &= m_* + \beta_\dashv \\
			\end{split}\end{equation*} %}
			$\beta_\dashv$は次のようになる。
			\begin{equation*}\begin{split} %{
				1_*\dashv1_* &= 0 \\
				1_*\dashv[x_1] &= 0 \\
				[x_1]\dashv1_* &= 0 \\
				[x_1]\dashv[x_2] &= [x_1[x_2]] \\
				[x_1x_2]\dashv[x_3] &= [x_1[x_3]x_2] + [x_1x_2[x_3]] \\
				[x_1[x_2]]\dashv[x_3] &= [x_1[x_2x_3]] + [x_1[x_2[x_3]]] \\
				[x_1]\dashv[x_2x_3] &= [x_1[x_3]x_2] + [x_1[x_2]x_3] + [x_1[x_2x_3]] \\
				[x_1]\dashv[x_2[x_3]] &= [x_1[x_2[x_3]]] \\
				\dots \\
			\end{split}\end{equation*} %}
			結合性$m_\perp(m_\perp\otimes\myid)=m_\perp(\myid\otimes m_\perp)$
			から次の式が導かれる。
			\begin{equation*}\begin{split} %{
				\beta_\dashv(\beta_\dashv\otimes\myid) 
				+ m_*(\beta_\dashv\otimes\myid) 
				+ \beta_\dashv(m_*\otimes\myid) \\
				= \beta_\dashv(\myid\otimes\beta_\dashv)
				+ m_*(\myid\otimes\beta_\dashv)
				+ \beta_\dashv(\myid\otimes m_*)
			\end{split}\end{equation*} %}
			双対性
			\begin{equation*}\begin{split} %{
				\Delta_*m_\perp 
				&= (m_\perp\otimes m_\perp)\sigma_{23}(\Delta_*\otimes \Delta_*) \\
				\Delta_*m_*
				&= (m_*\otimes m_*)\sigma_{23}(\Delta_*\otimes \Delta_*) \\
			\end{split}\end{equation*} %}
			から次の式が導かれる。
			\begin{equation*}\begin{split} %{
				\Delta_*\beta_\dashv 
				= (\beta_\dashv\otimes \beta_\dashv 
				+ m_*\otimes \beta_\dashv
				+ \beta_\dashv\otimes m_*)\sigma_{23}(\Delta_*\otimes\Delta_*)
			\end{split}\end{equation*} %}
		%s3:木の場合の考察}
	%s2:考察}
%}
