\section{Examples of coalgebra and bialgebra}
\begin{example}[polynomial]\label{eg:coalgebra.polynomial}
Let $R[x]$ be a polynomial ring over the commutative ring $R$. 
The linear map $\Delta$
\begin{equation}\begin{split}
	\Delta: R[x] &\to R[x] \otimes R[x] \\
		x^n &\mapsto \sum_{\substack{0\le p,q \le n \\p+q=n}}x^p \otimes x^q \\
\end{split}\end{equation}
satisfies the co-associative low:
\begin{equation}\begin{split}
	(\myid \otimes \Delta) \Delta x^n
		= \sum_{\substack{0\le p,q,r \le n \\p+q+r=n}}x^p \otimes x^q \otimes x^r
		&= (\Delta \otimes \myid) \Delta x^n \\
		&\text{ for all } 0 \le n
\end{split}\end{equation}
. The linear map $\epsilon$
\begin{equation}\begin{split}
	\epsilon: R[x] &\to R \\
		x^n &\mapsto 	\begin{cases}
			1, &\text{ if } n = 0 \\
			0, &\text{ otherwise } \\
		\end{cases}
\end{split}\end{equation}
satisfies the co-unit low:
\begin{equation}\begin{split}
	\pi_2(\epsilon \otimes \myid)\Delta
		= \myid
		= \pi_1(\myid \otimes \epsilon)\Delta
\end{split}\end{equation}
. Therefore $(R[x], \Delta, \epsilon)$ is a coalgebra.
And $(R[x], \cdot, 1, \Delta, \epsilon)$ is a bialgebra.
\begin{equation}\begin{split}
	\Delta(x^m \cdot x^n)
		= \sum_{\substack{0\le p,q \le m+n \\p+q=m+n}}x^p \otimes x^q 
		= (\Delta x^m) \cdot (\Delta x^n)
\end{split}\end{equation}
Where we denote the multiplication $\cdot$ as midpoint operator,
and multiplication of tensor is defined as the followings:
\begin{equation}\begin{split}
	(a \otimes b) \cdot (c \otimes d) = (a \cdot c) \otimes (b \cdot d)
\end{split}\end{equation}
.
\end{example}

\begin{example}[polynomial, another coalgebra]
Let $R[x]$ be a polynomial ring over the commutative ring $R$. 
The linear map $\Delta$
\begin{equation}\begin{split}
	\Delta: R[x] &\to R[x] \otimes R[x] \\
		x^n &\mapsto x^n \otimes x^n \\
\end{split}\end{equation}
satisfies the co-associative low. The linear map $\epsilon$
\begin{equation}\begin{split}
	\epsilon: R[x] &\to R \\
		x^n &\mapsto 1 \\
\end{split}\end{equation}
satisfies the co-unit low. Therefore $(R[x], \Delta, \epsilon)$ is 
a coalgebra. And $(R[x], \cdot, 1, \Delta, \epsilon)$ is a bialgebra.
\begin{equation}\begin{split}
	\Delta(x^m \cdot x^n) = x^{m+n} \otimes x^{m+n}
		= (\Delta x^m) \cdot (\Delta x^n)
\end{split}\end{equation}
Where we use the same convention of \ref{eg:coalgebra.polynomial}.
\end{example}
