\section{Framework of parser} % (fold) {
\label{sec:framework.of.paresr}
We define semirings that we use in this section. Some kind of semirings
make situation being complicated. For example, $2\times 2$ matrices over
$\myop{N}$ of the form:
\begin{equation*}\begin{split}
	\begin{pmatrix}
		a & b \\
		0 & d \\
	\end{pmatrix}
\end{split}\end{equation*}
are semiring with the standard matrix addition and multiplication.
The matrices of the form:
\begin{equation*}\begin{split}
	\begin{pmatrix}
		0 & b \\
		0 & 0 \\
	\end{pmatrix}
\end{split}\end{equation*}
are non-zero but nilpotent ($-^2=0$). We would like to exculude such types
of semiring from the argument of this section so to make easy construction
of semi-module.

\begin{define}[non-degenerate semiring]
Let $A=(A,+,0,\cdot,1)$ be a semiring:
\begin{itemize}
	\item the associative law for $\square=\set{+,\cdot}$ $$
		a_1 \square (a_2 \square a_3) = (a_1 \square a_2) \square a_3
	$$
	\item the unit law for $u_\square=\set{0,1}$ $$
		u_{\square} \square a = a = a \square u_{\square}
	$$
	\item and $+$ is commutative
	\item the distribution law $$
		a_1 \cdot (a_2 + a_3) = (a_1 \cdot a_2) + (a_1 \cdot a_3)
		(a_1 + a_2) \cdot a_3 = (a_1 \cdot a_3) + (a_2 \cdot a_3)
	$$
	\item and $0 \cdot a = 0 = a \cdot 0$.
\end{itemize}
. If the semiring $A$ satisfies the following condition for $a_1,a_2\in A$:
\begin{equation}\begin{split}
	a_1\cdot a_2 = 0 \implies (a_1 = 0) \text{ or } (a_2 = 0)
\end{split}\end{equation}
, we call that $A$ is non-degenerated.
\end{define}

\subsection{Polynomial} % (fold) {
\label{sec:Polynomial}
Let $\mybf{N}$ be the natual $\set{0,1,2,\cdots}$ with the standard
addition and multiplication, $\mybf{N}[x]$ be the polynomial
over $\mybf{N}$ of indeterminant $x$. We define the multiplication 
$\cdot:\mybf{N}[x]\otimes\mybf{N}[x]\to\mybf{N}[x]$
and the unital map $u:\mybf{N}\to\mybf{N}[x]$ as the followings.
\begin{equation*}\begin{split}
	x^m \cdot x^n &= x^{m+n} \\
	um &= \begin{cases}
			x^0, &\text{ if } m \neq 0 \\
			0, &\text{ otherwise } \\
		\end{cases}
\end{split}\end{equation*}
We define the linear map $\Delta$ for the generator as the followings.
\begin{equation*}\begin{split}
	\Delta: x &\mapsto x \otimes x^0 + x^0 \otimes x
\end{split}\end{equation*}
We would like to extend $\Delta$ to the whole $\mybf{N}[x]$ with keeping 
so that $\Delta$ is homeomorphism of $\cdot$.
\begin{equation*}\begin{split}
	\Delta x^{m+n} = (\Delta x^m) \cdot_2 (\Delta x^n)
\end{split}\end{equation*}
Where we denote $\cdot_2$ as the multiplcation of tensor product.
\begin{equation*}\begin{split}
	\cdot_2: (\mybf{N} \otimes \mybf{N}) \otimes (\mybf{N} \otimes \mybf{N}) &\to \mybf{N} \otimes \mybf{N} \\
		(x^p \otimes x^q) \otimes (x^r \otimes x^s) &\mapsto x^{p+r} \otimes x^{q+s}
\end{split}\end{equation*}
The homeomorphism condition implies the following results.
\begin{equation*}\begin{array}{lll}
	\Delta x &= (\Delta x) \cdot_2 (\Delta x^0)
		&= (x \otimes x^0 + x^0 \otimes x) \cdot_2 (\Delta x^0) \\
	\Delta x^2 &= (\Delta x) \cdot_2 (\Delta x)
		&= x^2 \otimes x^0 + 2 x \otimes x + x^0 \otimes x^2 \\
	\Delta x^3 &= (\Delta x) \cdot_2 (\Delta x^2)
		&= x^3 \otimes x^0 + 3 x^2 \otimes x + 3 x \otimes x^2 + x^0 \otimes x^2 \\
	\vdots \\
	\Delta x^m &= (\Delta x) \cdot_2 (\Delta x^{m-1})
		&= \sum_{0\le p\le m} \binom{m}{p} x^{m-p} \otimes x^p \\
\end{array}\end{equation*}
Though it seems that $\Delta x^0$ is not uniquely determined, but
imposing $\Delta x^0 = x^0 \otimes x^0$ satisfies homeomorphism condition.
This $\Delta$ satisfies the co-associativity.
\begin{equation}\begin{split}
	(\Delta \otimes \myid)\Delta x^m
		= \sum_{\substack{0\le p,q,r\le m\\p+q+r=m}}\frac{m!}{p!q!r!} x^p \otimes x^q \otimes x^r
		= (\myid \otimes \Delta)\Delta x^m
\end{split}\end{equation}
The co-unital map $\epsilon$ is defined as the followings.
\begin{equation}\begin{split}
	\epsilon: x^m \mapsto \begin{cases}
		1, &\text{ if } m = 0 \\
		0, &\text{ otherwise } \\
	\end{cases}
\end{split}\end{equation}
% section Polynomial (end) }

\subsection{Polynomial yet another multiplication} % (fold) {
Let $\mybf{N}$ be the natual $\set{0,1,2,\cdots}$ with the standard
addition and multiplication, $\mybf{N}[x]$ be the polynomial
over $\mybf{N}$ of indeterminant $x$. We define the multiplication 
$\star:\mybf{N}[x]\otimes\mybf{N}[x]\to\mybf{N}[x]$
and the unital map $u_{\star}:\mybf{N}\to\mybf{N}[x]$ as the followings.
\begin{equation*}\begin{split}
	x^m \star x^n &= \begin{cases}
		x^n, &\text{ if } m = n \\
		0, &\text{ otherwise } \\
	\end{cases} \\
	u_{\star}m &= \begin{cases}
			x^0 + x + x^2 + \cdots, &\text{ if } m \neq 0 \\
			0, &\text{ otherwise } \\
		\end{cases}
\end{split}\end{equation*}
We define the linear map $\Delta_{\star}$ as the followings.
\begin{equation*}\begin{split}
	\Delta_{\star}: x^m &\mapsto \sum_{0\le p\le m} x^{m-p} \otimes x^p
\end{split}\end{equation*}
This $\Delta_{\star}$ satisfies the co-associativity.
\begin{equation}\begin{split}
	(\Delta_{\star} \otimes \myid)\Delta_{\star} x^m
		= \sum_{\substack{0\le p,q,r\le m\\p+q+r=m}} x^p \otimes x^q \otimes x^r
		= (\myid \otimes \Delta_{\star})\Delta_{\star} x^m
\end{split}\end{equation}
The co-unital map $\epsilon_{\star}$ is defined as the followings.
\begin{equation}\begin{split}
	\epsilon_{\star}: x^m \mapsto \begin{cases}
		1, &\text{ if } m = 0 \\
		0, &\text{ otherwise } \\
	\end{cases}
\end{split}\end{equation}
The homeomorphism $\Delta_{\star}$ with $\star$ is satisfied.
\begin{equation}\begin{split}
	\Delta_{\star}(x^m \star x^n) = \delta(m-n)\Delta_{\star}x^n
		= (\Delta_\star x^m) \star_2 (\Delta_\star x^n)
\end{split}\end{equation}
Where $\delta$ is the Dirac's delta function, and we denote $
	\star_2:(a\otimes b)\otimes(c\otimes d)\mapsto (a\star c)\otimes(b\star d)
$.
% section Polynomial yet another multiplication (end) }

\subsection{Tensor product of semi-module} % (fold) {
\label{sec:tensor.product.of.semi-module}
Let $K=(K,+_K,0_K,m_K,1_K)$ be a non-degenerated semiring, 
$V$ be a semi-module over $K$. We see frequently the following diagrams 
to define unital or co-unital map on $V$.
\begin{equation}\xymatrix{
	K \otimes V \ar[r]^{u} \ar[dr]_{\simeq} & V \otimes V \ar[d]^{m} \\
	& V
}\xymatrix{
	K \otimes V \ar@{<-}[r]^{\epsilon} \ar@{<-}[dr]_{\simeq} & V \otimes V \ar@{<-}[d]^{\Delta} \\
	& V
}\quad\xymatrix{
}\end{equation}
The tensor product $K\otimes V=(K\times V)/\sim$ is defined by 
the formal summation $+$ and the equivalent relation $\sim$:
\begin{equation}\begin{split}
	k \times v &\sim 1_K \times kv \\
	k_1 \times v + k_2 \times v &\sim (k_1 + k_2) \times v \\
	k \times v_1 + k \times v_2 &\sim k \times (v_1 + v_2) \\
\end{split}\end{equation}
. Therefore any element of $K\otimes V$ can be written with $0$ or the form
$1_K\otimes v$, because $k\otimes v=1_K\otimes kv$ for all $k\in K$ and 
$v\in V$, and $K\otimes V$ and $V$ are isomorphic.

Let be $\mybf{B}=(\mybf{B},+_B,0_B,m_B,1_B)$ the idempotent boolean:
\begin{equation}\begin{split}
	+_B &: \text{logical or} \\
	m_B &: \text{logical and} \\
\end{split}\end{equation}
. We define the map $\varphi$:
\begin{equation}\begin{split}
	\varphi: K &\to \mybf{B} \\
		k &\mapsto \begin{cases}
			1_B, &\text{ iff } k \neq 0_K \\
			0_B, &\text{ otherwise } \\
		\end{cases}
\end{split}\end{equation}
. The map $\varphi$ is homeomorhpism:
\begin{equation*}\xymatrix@C+50pt{
	k_1\times k_2 \ar@{|->}[r]^{\varphi\times \varphi} \ar@{|->}[d]^{+_K} 
		& (\varphi k_1)\times (\varphi k_2) \ar@{|->}[d]^{+_B} \\
	k_1 +_K k_2 \ar@{|->}[r]^{\varphi} 
		& \varphi(k_1 +_K k_2) = (\varphi k_1) +_B (\varphi k_2) \\
}\end{equation*}
\begin{equation*}\xymatrix@C+50pt{
	k_1\times k_2 \ar@{|->}[r]^{\varphi\times \varphi} \ar@{|->}[d]^{m_K} 
		& (\varphi k_1)\times (\varphi k_2) \ar@{|->}[d]^{m_B} \\
	k_1 \cdot_K k_2 \ar@{|->}[r]^{\varphi} 
		& \varphi(k_1 \cdot_K k_2) = (\varphi k_1) \cdot_B (\varphi k_2) \\
}\end{equation*}
\begin{equation*}\begin{split}
	\varphi(k_1 +_K k_2) = (\varphi k_1) +_B (\varphi k_2)
		= \begin{cases}
			1_B, &\text{ iff } (k_1 \neq 0_K) \text{ or } (k_2 \neq 0_K) \\
			0_B, &\text{ otherwise } \\
		\end{cases} \\
	\varphi(k_1 \cdot_K k_2) = (\varphi k_1) \cdot_B (\varphi k_2)
		= \begin{cases}
			1_B, &\text{ iff } (k_1 \neq 0_K) \text{ and } (k_2 \neq 0_K) \\
			0_B, &\text{ otherwise } \\
		\end{cases}
\end{split}\end{equation*}
, where we denote $m_X(x_1\times x_2)=x_1\cdot_X x_2$ 
for $X=\set{K,\mybf{B}}$. It is seemed that there is isomorphism
$(K\otimes V)\simeq (\mybf{B}\otimes V)$.
%}

\section{Backup} % (fold) {
\label{sec:Backup}

We shall sketch the framework of parser with respect to algebraic aspect.
Let $A=(A,\Delta_A)$ be a co-semigroup, $B=(B, +_B, 0_B, m_B, 1_B)$
be a semiring. We can define the semiring structure with convolutions:
\begin{equation}\begin{split}
	\square: (A \to B) \times (A \to B) &\to (A \to B) \\
		f \times g &\mapsto \square_B(f \times g)\Delta_A
\end{split}\end{equation}
, where we denote $\square=\set{+, m}$ and $\square_B=\set{+_B, m_B}$
respectively. The constant map $\beta$
\begin{equation}\begin{split}
	\beta: B &\to (A \to B) \\
		b &\mapsto \beta_b \text{ such that } \beta_b a = b \text{ for all } a \in A
\end{split}\end{equation}
satisfies the following properites:
\begin{itemize}
	\item $\beta$ being semiring homeomorphism,
	\item $\beta$ being $1:1$,
	\item $\beta_{0_B}$ being the unital element of $+$,
	\item $\beta_{1_B}$ being the unital element of $m$,
\end{itemize}
. Then the structure $(A\to B, +, \beta_{0_B}, m, \beta_{1_B})$ is a 
semiring. We can see this structure as semi-module over $B$ like that:
\begin{itemize}
	\item scalar product $m^\circ$
	\begin{equation}\begin{split}
		m^\circ: B \times (A \to B) &\to (A \to B) \\
			b \times f &\mapsto m_B(\beta_b \times f) \\
		m^\circ: (A \to B) \times B &\to (A \to B) \\
			f \times b &\mapsto m_B(f \times \beta_b) \\
	\end{split}\end{equation}
	\item unital map $u$
	\begin{equation}\begin{split}
		u: B &\to (A \to B) \\
			b &\mapsto \begin{cases}
				\beta_{1_B}, &\text{ iff } b \neq 0_B \\
				\beta_{0_B}, &\text{ otherwise } \\
			\end{cases} \\
			\text{need proof}
	\end{split}\end{equation}
\end{itemize}
. The map $\alpha$
\begin{equation}\begin{split}
	\alpha: A &\to (A \to B) \\
		a &\mapsto a_1 \mapsto \begin{cases}
			1_B, &\text{ iff } a = a_1 \\
			0_B, &\text{ otherwise } \\
		\end{cases}
\end{split}\end{equation}
gives the basis of $A\to B$, because the following equation is satisfied:
\begin{equation}\begin{split}
	fa = \sum_{x\in A} m_B((fx) \times (\alpha_x a)) \text{ for all } \begin{cases}
		a\in A \\
		f\in (A \to B)
	\end{cases}
\end{split}\end{equation}
. The map $\alpha$ has the constraint that the sum of all basis equals the
unit element:
\begin{equation}\begin{split}
	\sum_{x\in A} \alpha_x = \beta_{1_B}
\end{split}\end{equation}
.

Let $A=(A,m_A,\Delta_A)$ be a bi-semigroup, $B=(B,+_B,0_B,m_B,\Delta_B,1_B)$
be a bi-semiring. We can define the semiring structure 
$(A\to B,+,\beta_{0_B},m,\beta_{1_B})$ or semi-module structure 
$(A\to B,+,\beta_{0_B},m^\circ,u)$. To simplify notation, we restrict 
the bi-semigroup $B$ is commutative for multiplication $m_B$.
We can define the co-multiplication $\Delta$ with convolution:
\begin{equation}\begin{split}
	\Delta: (A \to B) &\to ((A \to B) \otimes (A \to B)) \\
		f &\mapsto \Delta_B f m_A
\end{split}\end{equation}
, and can define the co-unital map $\epsilon$:
\begin{equation}\begin{split}
	\epsilon: (A \to B) &\to B \\
		f &\mapsto \begin{cases}
			1_B, &\text{ iff } f \neq \alpha_{1_B} \\
			0_B, &\text{ otherwise } \\
		\end{cases} \\
		\text{need proof}
\end{split}\end{equation}
. Then the structure $(A\to B,+,m^\circ,u,\Delta,\epsilon)$ is a bialgebra
over $B$.

\begin{observe}[commutative diagrams]
The multiplications $\square=\set{+,m}$ on $A\to B$ are define by
the following commutativ diagram:
\begin{equation}\xymatrix{
	A \ar@{.>}[d]_{\square(f \times g)} \ar[r]^{\Delta_A} & A\times A \ar[d]^{f \times g} \\
	B & B\otimes B \ar[l]_{\square_B} \\
}\end{equation}
. The unital map $u$ and basis $\alpha$ need futher consideration,
especially the set-isomorphism $(X\times Y)\to Z \simeq X\to (Y\to Z)$
and the semiring-homeomorphism $\myop{not-zero}: B\to B^\circ=\set{0_B,1_B}$.
\end{observe}

\begin{example}[words to boolean]\label{eg:words.boolean}
Let $\mybf{B}$ be the boolean $\set{0,1}$ with the following operations:
\begin{equation}\begin{split}
	+ &: \text{logical or} \\
	\cdot &: \text{logical and} \\
\end{split}\end{equation}
. $(\mybf{B},+,0,\cdot,1)$ becomes an idempotent semiring.
We define co-multiplication $\delta$ as the followings:
\begin{equation}\begin{split}
	\delta: \mybf{B} &\to \mybf{B} \otimes \mybf{B} \\
		b &\mapsto b \otimes b \\
\end{split}\end{equation}
. Then $(\mybf{},+,0,\cdot,\delta,1)$ becomes a bi-semiring.
Let $WS$ be the free semigroup of a finite set $S$,
We define the multiplication $\cdot$ and the co-multiplication 
$\delta$ as the followings:
\begin{equation}\begin{split}
	\cdot: WS \times WS &\to WS \\
		[s_1s_2\cdots s_m] \times [t_1t_2\cdots t_n] &\mapsto [s_1s_2\cdots s_mt_1t_2\cdots t_n] \\
	\delta: WS &\to WS \times WS \\
		w &\mapsto w \times w \\
\end{split}\end{equation}
. The multiplication $\cdot$ and co-multiplication $\Delta$ on 
$WS\to \mybf{B}$ are defined as the followings:
\begin{equation}\begin{split}
	(f \cdot g)w &= (fw)\cdot(gw) \\
	(\Delta f)(w_1 \times w_2) &= f(w_1\cdot w_2) \otimes f(w_1\cdot w_2) \\
\end{split}\end{equation}
. Let $\set{w^t}_{w\in WS}$ be the basis of $WS\to \mybf{B}$:
\begin{equation}\begin{split}
	-^t: WS &\to (WS \to \mybf{B}) \\
		w &\mapsto w_1 \mapsto \begin{cases}
			1, &\text{ iff } w = w_1 \\
			0, &\text{ otherwise } \\
		\end{cases}
\end{split}\end{equation}
. The multiplication $\cdot$ of basis implies the followings:
\begin{equation}\begin{split}
	(w_1^t \cdot w_2^t)w &= \begin{cases}
		1, &\text{ iff } w_1 = w_2 = w \\
		0, &\text{ otherwise} \\
	\end{cases} \\
	&\Downarrow \text{sufficent but not neccesary} \\
	w_1^t \cdot w_2^t &= \begin{cases}
		w_1, &\text{ iff } w_1 = w_2 \\
		0, &\text{ otherwise } \\
	\end{cases}
\end{split}\end{equation}
, and co-multiplication $\Delta$ for the basis implies the followings:
\begin{equation}\begin{split}
	(\Delta w^t)(w_1 \times w_2) &= \begin{cases}
		1 \otimes 1, &\text{ iff } w = w_1 \cdot w_2 \\
		0, &\text{ otherwise } \\
	\end{cases} \\
	&\Downarrow \text{sufficent but not neccesary} \\
	\Delta w^t &= \sum_{w_1 \cdot w_2=w} w_1^t \otimes w_2^t \\
\end{split}\end{equation}
. The unital and co-unital map are defined as the followings:
\begin{equation}\begin{split}
	u: \mybf{B} &\to (WS \to \mybf{B}) \\
		b &\mapsto \begin{cases}
			1, &\text{ iff } b \neq 0 \\
			0, &\text{ otherwise } \\
		\end{cases} \\
	\epsilon: (WS \to \mybf{B}) &\to \mybf{B} \\
		w^t &\mapsto \begin{cases}
			1, &\text{ iff } w = 1_A \\
			0, &\text{ otherwise } \\
		\end{cases}
\end{split}\end{equation}
. Note that $1_A^t \neq 1$ in generally.

Let examine the $\Delta$ is homeomorphism with respect to 
$\cdot$ of $WS\to\mybf{B}$.
\begin{proof}
For any $w_1,w_2\in (WS \to \mybf{B})$,
\begin{equation}\begin{split}
	w_1^t \otimes w_2^t \xrightarrow{\cdot} \begin{cases}
			w_1^t \\
			0 \\
		\end{cases} \xrightarrow{\Delta} \begin{cases}
			\sum_{x\cdot y = w_1} x^t \otimes y^t, &\text{ iff } w_1 = w_2 \\
			0, &\text{ otherwise } \\
		\end{cases}
\end{split}\end{equation}
and
\begin{equation}\begin{split}
	w_1^t \otimes w_2^t \xrightarrow{\Delta \otimes \Delta} 
		\sum_{\substack{x_1\cdot x_2=w_1\\y_1\cdot y_2=w_2}} 
			(x_1^t \otimes x_2^t) \otimes (y_1^t \otimes y_2^t) 
		\xrightarrow{\cdot} 
		\sum_{\substack{x_1\cdot x_2=w_1\\y_1\cdot y_2=w_2\\x_1=y_1\\x_2=y_2}} 
			x_1^t \otimes x_2^t
\end{split}\end{equation}
and
\begin{equation}\begin{split}
	\sum_{\substack{x_1\cdot x_2=w_1\\y_1\cdot y_2=w_2\\x_1=y_1\\x_2=y_2}} x_1^t \otimes x_2^t
		= \begin{cases}
			\sum_{x\cdot y = w_1} x^t \otimes y^t, &\text{ iff } w_1 = w_2 \\
			0, &\text{ otherwise } \\
		\end{cases}
\end{split}\end{equation}
implies
\begin{equation}\begin{split}
	\Delta \cdot (w_1^t \otimes w_2^t) = \cdot (\Delta \otimes \Delta) (w_1^t \otimes w_2^t)
\end{split}\end{equation}
\end{proof}
\end{example}

\begin{observe}[shuffle product]
The co-multiplication in the example above \ref{eg:words.boolean} is
called shuffle co-product. The corresponding shuffle product $\amalg$
is defined as the followings:
\begin{equation}\begin{split}
	w_1^t \amalg w_2^t &= ((\Delta^{(1)}w_1^t) \amalg (\Delta^{(1)}w_2^t)) * ((\Delta^{(2)}w_1^t) \amalg (\Delta^{(2)}w_2^t)) \\
	[s_1]^t \amalg [s_2]^t &= [s_1s_2]^t + [s_2s_1]^t \\
	[s_1 \cdots s_p]^t * [t_1 \cdots t_q]^t &= [s_1 \cdots s_p t_1 \cdots t_q]^t \\
\end{split}\end{equation}
. The relation of the shuffle product and the multiplication in the
example \ref{eg:words.boolean} is not clear yet.
\end{observe}

\begin{todo}[representation and automata]
Let $A=(A,m_A,\Delta_A,1_A)$ be a bi-monoid,
$B=(B,+_B,0_B,m_B,\Delta_B,1_B)$ be a bi-semiring.
The map $\rho$
\begin{equation}\begin{split}
	\rho: A \times (A \to B) &\to (A \to B) \\
		a \times f &\mapsto m^\circ((\Delta^{(1)}f)a \otimes \Delta^{(2)}f) \\
\end{split}\end{equation}
satisfies the associativity and the unit low.
Therefore the map $\rho$ gives the representation of $A$. 
When we fix the some map $f\in (A\to B)$, the co-domain $\rho(A, f)$
has the following properties:
\begin{itemize}
	\item $f\in \rho(A, f)$
	\item $\rho(A, \rho(A, f)) \subseteq \rho(A, f)$
\end{itemize}
?
This representation is the automata in the programing world.
\end{todo}

% section Backup (end) }
