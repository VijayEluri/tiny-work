\begingroup %{
	\newcommand{\Mod}[1]{{#1}\myhere\mybf{Mod}}
	\newcommand{\Vect}[1]{\mybf{Vec}_{#1}}
	\newcommand{\Hom}{\myop{Hom}}
	\newcommand{\T}{\mycal{T}}
	\newcommand{\W}{\mycal{W}}
	\newcommand{\id}{\myop{id}}
	\newcommand{\dup}{\myop{du}}
	\newcommand{\onto}{\myop{onto}}
	\newcommand{\Wedge}{{\bigwedge}}
	%
	\newcommand{\tran}{\mathbf{t}}
	%
	\newcommand{\from}{\xfrom{}}
	\newcommand{\toto}{\rightrightarrows}
	\newcommand{\fromfrom}{\leftleftarrows}
	\newcommand{\tofrom}{\rightleftarrows}
	\newcommand{\fromto}{\leftrightarrows}
	%
	{\setlength\arraycolsep{2pt}
	%
\section{可換環}\label{s1:可換環} %{
	この節を通して、$R=(R,+,0,\myspace,1)$を可換環とする。

	\begin{definition}[可換環のイデアル]\label{def:可換環のイデアル} %{
		部分環$A\subseteq R$が次の性質を満たすとき、$A$を$R$のイデアルという。
		\begin{equation*}\begin{split}
			ra\in A \quad\text{for all } a\in A,\; r\in R
			\quad\bigl(\iff RA\subseteq A \bigr)
		\end{split}\end{equation*}
	\end{definition} %def:可換環のイデアル}

	\begin{definition}[単項イデアル]\label{def:単項イデアル} %{
		任意の$r\in R$に対して$R$の部分集合$rR:=\set{rs\in R\bou s\in R}$は
		はイデアルとなる。
		このイデアル$rR$を$r$から生成された単項イデアルという。
	\end{definition} %def:単項イデアル}

	\begin{definition}[部分集合から生成されたイデアル]
	\label{def:部分集合から生成されたイデアル} %{
		任意の空でない部分集合$S\subseteq R$に対して$R$の部分集合
		$SR:=\set{sr\in R\bou s\in S,\; r\in R}$はイデアルとなる。
		このイデアル$SR$を$S$から生成されたイデアルという。
	\end{definition} %def:部分集合から生成されたイデアル}

	自明なイデアルといった場合は、自明な環や自明な群などとは'自明'の意味する
	ものが異なることに注意する。

	\begin{definition}[自明なイデアル]\label{def:自明なイデアル} %{
		加法の単位元のみからなるイデアル$\set{0}$と$R$自身を自明なイデアル
		という。
	\end{definition} %def:自明なイデアル}

	\begin{proposition}[体とイデアル]\label{prop:体とイデアル} %{
		$R$が体になる必要十分条件は、$R$のイデアルが自明なイデアルのみになること
		である。
	\end{proposition} %prop:体とイデアル}
	\begin{proof} 
	\begin{description}\setlength{\itemsep}{-1mm} %{
		\item[必要] 体$F$のイデアルが自明なイデアルに限ることを示せばよい。
		$A$を$F=(F,+,0,\myspace,1)$のイデアルとする。
		$S$が$0$以外の元を含むとすると、イデアルの定義より、
		任意の$a\neq 0\in A$に対して$a^{-1}a=1\in A$となり、$1\in A$となる。
		したがって、イデアルの定義より、任意の$f\in F$に対して$f=f1\in A$
		となり、$A=F$となることが示される。
		%
		\item[十分] 環$R$のイデアルが$\set{0}$と$R$のみだとする。
		仮定から、任意の$r\neq0\in R$に対して単項イデアル$rR$は$rR=R$となる。
		したがって、$rR=R$から$1\in R$となり、$rs=1$となる$s\in R$が存在する
		ことがわかる。
		\begin{itemize}\setlength{\itemsep}{-1mm} %{
			\item $r\neq 1$の場合は、$rR=R$だから$rs=1$となる$s\in R$が存在する
			ことがわかる。
			\item $r=1$の場合は、$rr=1$となる。
		\end{itemize} %}
		よって、任意の$r\neq0\in R$に対して乗法の逆元が存在することがわかる。
	\end{description} %}
	\end{proof}
%s1:可換環}
\section{加群}\label{s1:加群} %{
	この節では、任意の圏$\mycal{C}$に対して、$\mycal{C}$の対象全体を
	$\mycal{C}_O$、$\mycal{C}$の射全体を$\mycal{C}_A$と書く。

	この節を通して、$R=(R,+,0,\myspace,1)$を可換環、$\Mod{R}$を小さな
	$R$-加群の圏とする。任意の$A,B\in\Mod{R}_O$に対して、$A$から$B$への
	$R$-準同型全体を$\Hom_R(A,B)$と書き、混乱の恐ればないときは、
	$A$から$B$への$R$-準同型$f$を単に射$f:A\to B$とか$A\xto{f}B\in\Mod{R}$
	と書く。

	任意の$A,B\in\Mod{A}_R$に対して$\Hom_R(A,B)$は次の畳み込みにより
	$R$-加群となる。
	\begin{equation*}\begin{array}{rcll}
		(f + g)a &:=&  (fa) + (ga)
			& \quad\text{for all } f,g\in\Hom_R(A,B),\; a\in A \\
		(rf)a &:=& rfa
			& \quad\text{for all } f\in\Hom_R(A,B),\; r\in R,\; a\in A \\
	\end{array}\end{equation*}
	また、$0$への恒等射を$0:a\mapsto 0\text{ for all }a\in A$と書くと、
	$0$は加法$+$の単位元となる。この畳み込みによる$\Hom_R(A,B)$の$R$-加群
	は、断りなしに使うことにする。

	\begin{note}[加群と表現]\label{note:加群と表現} %{
		$\Mod{R}$での射は$R$-線形写像となる。
		\begin{equation*}\begin{split}
			fr = rf \quad\text{for all } f\in\Mod{R}_A,\; r\in R
		\end{split}\end{equation*}
		係数環$R$を複素数上の代数$V$とすると、$\Mod{V}$の対象は$V$の表現空間、
		射はintertwinerとなる。Schurの補題から、既約表現の間のintertwinerは、
		互いに$\fukuso$-同型となる既約表現の間に限られる。
		圏$\Mod{V}$の様子を絵にすると次のようになる。
		\begin{equation*}\xymatrix{
			\text{可約表現} \ar[d] \ar[dr] \ar@<1ex>[rr] 
			&& \text{可約表現} \ar[d] \ar[dl] \ar[ll] \\
			\text{既約表現} \ar@<1ex>[u]
			& \text{既約表現} \ar@<1ex>[ul] \ar@<1ex>[ur]
			& \text{既約表現} \ar@<1ex>[u]
		}\end{equation*}
	\end{note} %note:加群と表現}

\subsection{加群の生成系}\label{s2:加群の生成系} %{
	まず、加群の自由直積を定義してから、加群の生成系と基底系を順に定義する。

	\begin{definition}[自由加群その一]\label{def:自由加群その一} %{
		$X$を空でない集合、$X$から$R$への写像全体の集合$\mybf{Set}(X,R)$の
		部分集合$\coprod_XR$を次のように定義する。
		\begin{equation*}\begin{split}
			\coprod_XR := \set{f:X\to R
			\bou fx\neq 0 \text{ only finitely many }x\in X}
		\end{split}\end{equation*}
		$\coprod_XR$に加法$+$と係数の作用$\myspace$を次のように定義する。
		\begin{equation*}\begin{array}{lrcll}
			\text{加法} & (f + g)x &:=& (fx) + (gx)
			& \quad\text{for all }f,g\in\coprod_XR,\; x\in X \\
			\text{係数} & (rf)x &:=& r(fx)
			& \quad\text{for all }f\in\coprod_XR,\; r\in R,\; x\in X \\
		\end{array}\end{equation*}
		すると、次の事が成り立つことがわかる。
		\begin{itemize}\setlength{\itemsep}{-1mm} %{
			\item $0\in R$への恒等写像$0\in\coprod_XR$が加法の単位元となる。
			\begin{equation*}\begin{split}
				f + 0 = f = 0 + f & \quad\text{for all } f\in\coprod_XR \\
				rf = 0 & \quad\text{for all } f\in\coprod_XR,\; r\in R
			\end{split}\end{equation*}
			\item 分配則が成り立つ
			\begin{equation*}\begin{split}
				r(f + g) = rf + rg \quad\text{for all } f,g\in \coprod_XR,\; r\in R
			\end{split}\end{equation*}
		\end{itemize} %}
		したがって、$(\prod_XR,\myspace,+,0)$は$R$-加群となる。
		$(\prod_XR,\myspace,+,0)$を$X$から生成された自由$R$-加群といい、
		$RX^\tran$と書く。
	\end{definition} %def:自由加群その一}

	\begin{definition}[自由加群その二]\label{def:自由加群その二} %{
		$A$を$R$-加群とする。$A$がある空でない集合$X$から生成された自由$R$-加群
		と$R$-加群同型になることき、$A$を自由$R$-加群という。
	\end{definition} %def:自由加群その二}

	$X$を集合とし、$x\in X$の双対元$x^\tran\in RX^\tran$を次のように
	定義すると、
	\begin{equation*}\begin{split}
		x^\tran y = \jump{x=y} \quad\text{for all } y\in X
	\end{split}\end{equation*}
	任意の写像$f:X\to R$は双対元の線形結合で書くことができる。
	\begin{equation*}\begin{split}
		f = \sum_{x\in X}(fx)x^\tran
	\end{split}\end{equation*}
	また、写像$-^\tran\in\mybf{Set}(X,RX^\tran)$を次のように定義すると、
	\begin{equation}\label{eq:転置の定義その一}\begin{split}
		(-^\tran)x = x^\tran \quad\text{for all } x\in X
	\end{split}\end{equation}
	$-^\tran$は$1:1$になる。また、$\vec{-}:\mybf{Set}(X,\Hom_R(RX^\tran,R))$
	を次のように定義すると、
	\begin{equation}\label{eq:転置の定義その二}\begin{split}
		\vec{x}f := fx \quad\text{for all } f\in RX^\tran,\; x\in X 
	\end{split}\end{equation}
	次の式が成り立つ。
	\begin{equation}\label{eq:自由加群の恒等射}\begin{split}
		\sum_{x\in X} (\vec{x}-)x^\tran = \id_{RX^\tran}
	\end{split}\end{equation}

	\begin{definition}[転置]\label{def:転置} %{
		$X$を集合とする。式\eqref{eq:転置の定義その一}で定義された写像
		$-^\tran:X\to RX^\tran$を転置ということにする。
		また、誤解の恐れがないときは、式\eqref{eq:転置の定義その二}で
		定義された写像$\vec{-}:X\to\Hom_R(RX^\tran,R)$を用いて、
		次の$R$-加群準同型$-^\tran:RX^\tran\tofrom\Hom_R(RX^\tran,R)$
		にも同じ記号を用いる。
		\begin{equation*}\begin{split}
			x \xto{-^\tran} x^\tran \udset{-^\tran}{-^\tran}{\tofrom} \vec{x}
			\quad\text{for all } x\in X
		\end{split}\end{equation*}
	\end{definition} %def:転置}

	\begin{proposition}[自由加群の普遍性]\label{prop:自由加群の普遍性} %{
		$X$を空でない集合、$A$を加群とすると、任意の写像$f:X\to A$に対して
		次の図を可換にする$R$-加群準同型$f_*:RX^\tran\to A$が唯一つ定まる。
		\begin{equation*}\xymatrix{
			X \ar[r]^{-^\tran} \ar[rd]_f & RX^\tran \ar@{.>}[d]^{f_*} \\
			& A
		}\end{equation*}
	\end{proposition} %prop:自由加群の普遍性}
	\begin{proof} 
	\begin{description}\setlength{\itemsep}{-1mm} %{
		\item[存在] 写像$f_*\in\mybf{Set}(X^\tran,A)$を次のように定義する。
		\begin{equation*}\begin{split}
			f_*x^\tran := fx \quad\text{for all } x\in X
		\end{split}\end{equation*}
		$f_*$を$R$-線形に拡張して、$f_*\in\Hom_R(RX^\tran,A)$とすると、
		次のようになる。
		\begin{equation*}\begin{split}
			f_*g = \sum_{x\in X}(fx)(\vec{x}g) \quad\text{for all } g\in RX^\tran
		\end{split}\end{equation*}
		$f_*(-^\tran)=f$となるから、命題の$f_*$が存在することがわかる。
		\item[唯一] 写像$-^\tran:X\to RX^\tran$が$1:1$だから、
		任意の$f_*,g_*\in\Hom_R(RX^\tran,A)$に対して次の式が成り立つ。
		\begin{equation*}\begin{split}
			g_*(-^\tran) = f = f_*(-^\tran)
			\implies (f_* - g_*)(-^\tran) = 0
			\implies f_* = g_*
		\end{split}\end{equation*}
		したがって、命題の$f_*$が存在するならば唯一つであることがわかる。
	\end{description} %}
	\end{proof}

	\begin{definition}[加群の生成系]\label{def:加群の生成系} %{
		$A$を$R$-加群、$E$を$A$の空でない部分集合とする。
		$E$の有限の$R$-線形結合で張られる$A$の部分集合を$RE$と書く。
		\begin{equation*}\begin{split}
			RE := \set{\sum_{e\in E}r_ee\in A\bou r_e\in R \text{ and }
			r_e\neq 0 \text{ for only finitely many } e\in E}
		\end{split}\end{equation*}
		$RE=E$となるとき、$E$を$A$の生成系という。
	\end{definition} %def:加群の生成系}

	生成系に一意性の条件を課したものが基底系となる。

	\begin{definition}[加群での線形独立]\label{def:加群での線形独立} %{
		$A$を$R$-加群とする。$A$の空でない部分集合$E$が次の条件を満たすとき、
		$E$を$R$-線形独立な部分集合という。
		\begin{equation*}\begin{split}
			\sum_{e\in E}(fe)e = 0 \implies f = 0
			\quad\text{for all } f\in RE^\tran
		\end{split}\end{equation*}
	\end{definition} %def:加群での線形独立}

	\begin{definition}[加群の基底系]\label{def:加群の基底系} %{
		$A$を$R$-加群、$E$を$A$の生成系とする。$E$が$R$-線形独立なとき、
		$E$を$A$の基底系という。
	\end{definition} %def:加群の基底系}

	\begin{proposition}[基底系による一意的な表現]
	\label{prop:基底系による一意的な表現} %{
		$A$を$R$-加群、$E$を$A$の基底系とする。
		任意の$A$の元は$E$の元の線形結合で一意に書くことができる。
	\end{proposition} %prop:基底系による一意的な表現}
	\begin{proof} 次の式が成り立つ。
		\begin{equation*}\begin{split}
			\sum_{e\in E}(fe)e = \sum_{e\in E}(ge)e
			\implies \bigl((fe) - (ge)\bigr)e = 0 \implies f = g \\
			\quad\text{for all } f,g\in RE^\tran
		\end{split}\end{equation*}
	\end{proof}

	自由加群と基底系を持つことは同じことになる。

	\begin{proposition}[自由加群と基底系]\label{prop:自由加群と基底系} %{
		$A$を$R$-加群とする。
		\begin{itemize}\setlength{\itemsep}{-1mm} %{
			\item $A$が自由加群であることと、
			\item $A$が基底系を持つこと
		\end{itemize} %}
		は同値である。
	\end{proposition} %prop:自由加群と基底系}
	\begin{proof} 
	\begin{description}\setlength{\itemsep}{-1mm} %{
		\item[自由$\implies$基底] $F$を集合$X$から生成された自由$R$-加群とし、
		同型射$\phi:F\to A$が成り立つとする。
		次の写像で$\phi X^\tran$が$A$の基底系となることを示す。
		\begin{equation*}\begin{split}
			X\xto{-^\tran} F\xto{\;\phi\;} A
		\end{split}\end{equation*}
		\begin{description}\setlength{\itemsep}{-1mm} %{
			\item[生成系] $\phi X^\tran$が$A$の生成系となることを示す。
			$\phi$が全射だから、任意の$a\in A$に対してある$f\in F$が存在して、
			$a=\phi f$とすることができる。すると、式\eqref{eq:自由加群の恒等射}
			より、次の式が成り立つことがわかる。
			\begin{equation*}\begin{split}
				a = \phi f = \phi \sum_{x\in X} (fx)x^\tran
				= \sum_{x\in X} (fx)(\phi x^\tran)
			\end{split}\end{equation*}
			したがって、$\phi X^\tran$が$A$の生成系となることがわかる。
			%
			\item[基底系] $\phi X^\tran$が$R$-線形独立となることを示す。
			$\phi$と$-^\tran$が単射だから、任意の$f\in RX^\tran$に対して
			次の式が成り立つ。
			\begin{equation*}\begin{array}{rcll}
				\sum_{x\in X} (fx)(\phi x^\tran) = 0
				&\implies& \phi\sum_{x\in X} (fx)x^\tran = 0
					& \quad\because\;\text{$\phi$ is morphism} \\
				&\implies& \sum_{x\in X} (fx)x^\tran = 0
					& \quad\because\;\text{$\phi$ is $1:1$} \\
				&\implies& f = 0
					& \quad\because\;\text{$-^\tran$ is $1:1$} \\
			\end{array}\end{equation*}
			したがって、$\phi X^\tran$が$R$-線形独立になることがわかる。
		\end{description} %}
		%
		\item[基底$\implies$自由] $X$を$A$の基底系とする。
		$R$-加群準同型$\phi:RX^\tran\to A$を次のように定義する。
		\begin{equation*}\begin{split}
			\phi f = \sum_{x\in X} (fx)x
		\end{split}\end{equation*}
		$X$が$A$の生成系であることから$\phi$が$\onto$になることが示され、
		$X$が$R$-線形独立であることから$\phi$が$1:1$になることが示される。
	\end{description} %}
	\end{proof}


	\begin{example}[生成系だが基底系でない例]
	\label{eg:生成系だが基底系でない例} %{
		加群の基底系でない生成系の例を挙げる。
		\begin{itemize}\setlength{\itemsep}{-1mm} %{
			\item $\sei$-加群$\sei_2$は、生成系$E:=\set{1_2\in\sei_2}$を持つが、
			$21_2=0$だから、$E$は基底系ではない。任意の$2\le n\in\sei$に対して、
			$\sei$-加群$\sei_n$も同様である。
			%
			\item $\sei$-加群$\bun$は、生成系
			$E:=\set{1/p^n\bou p\in\text{primes},\; n\in\sizen_+}$
			を持つが、$1\cdot1/1=2\cdot1/2$だから、$E$は基底系ではない。
			%
			\item $\sei_6$-加群$\sei_6$は、生成系$E:=\set{4_6,3_6}$を持つが、
			\begin{equation*}\begin{split}
				1_6 = 4_6 + 3_6 \implies n_6 = n_6\cdot4_6 + n_6\cdot3_6
				\quad\text{for all } n_6\in \sei_6
			\end{split}\end{equation*}
			$3_6\cdot4_6=2_6\cdot3_6=0_6$だから、$E$は生成系ではない。
		\end{itemize} %}
	\end{example} %eg:生成系だが基底系でない例}
%s2:加群の生成系}

\subsection{加群の直和}\label{s2:加群の直和} %{
	\begin{definition}[加群の直和]\label{def:加群の直和} %{
		$A,B$を$R$-加群とする。直積$A\times B$に次のように加法$+$と係数の作用
		$\myspace$を定義したものを$A$と$B$の直和といい、$A\oplus B$と書く。
		\begin{equation*}\begin{array}{rcll}
			(a_1,b_1) + (a_2,b_2) &=& (a_1+a_2,b_1+b_2)
				& \quad\text{for all }a_1,a_2\in A,\; b_1,b_2\in B \\
			r(a,b) &=& (ra,rb)
				& \quad\text{for all }a\in A,\; b\in B,\; r\in R \\
		\end{array}\end{equation*}
	\end{definition} %def:加群の直和}
%s2:加群の直和}

\subsection{加群のテンソル積}\label{s2:加群のテンソル積} %{
	加群のテンソル積を定義して、テンソル積を使ってテンソル代数を定義する。

	\begin{definition}[双線形写像]\label{def:双線形写像} %{
		$A,B$を$R$-加群とする。次の性質を満たす写像$f:A\times B\to R$を
		$A$から$B$への$R$-双線形写像という。
		\begin{equation*}\begin{array}{rcll}
			f(a_1+a_2,b) &=& f(a_1,b) + f(a_2,b)
				& \quad\text{for all }a_1,a_2\in A,\; b\in B \\
			f(a,b_1+b_2) &=& f(a,b_1) + f(a,b_2)
				& \quad\text{for all }a\in A,\; b_1,b_2\in B \\
			f(ra,b) &=& f(a,rb)
				& \quad\text{for all }a\in A,\; b\in B,\; r\in R \\
		\end{array}\end{equation*}
	\end{definition} %def:双線形写像}

	一般の加群に対するテンソル積の定義は面倒なことになる。

	\begin{definition}[テンソル積]\label{def:テンソル積} %{
		$A,B$を$R$-加群とする。
		\begin{enumerate}\setlength{\itemsep}{-1mm} %{
			\item\label{item:テンソル積その一} 
			$\sei(A\times B)$を直積$A\times B$から生成される自由アーベル群
			(形式和)とする。
			%
			\item\label{item:テンソル積その二} 
			$\sei(A\times B)$に次の同値関係$\sim$を定義する。
			\begin{equation*}\begin{array}{rcll}
				(a_1 + a_2, b) &\sim& (a_1, b) + (a_2, b)
					& \quad\text{for all }a_1,a_2\in A,\; b\in B \\
				(a, b_1 + b_2) &\sim& (a, b_1) + (a, b_2)
					& \quad\text{for all }a\in A,\; b_1,b_2\in B \\
				(ra, b) &\sim& (a,rb)
					& \quad\text{for all }a\in A,\; b\in B,\; r\in R \\
			\end{array}\end{equation*}
			%
			\item\label{item:テンソル積その三} 
			$\sei(A\times B)$を$\sim$で割ったアーベル群を
			$A\otimes B:=\sei(A\times B)/\sim$と書く。
			%
			\item\label{item:テンソル積その四} 
			$A\otimes B$に係数の作用を次のように定義すると、$A\otimes B$は
			$R$-加群となる。
			\begin{equation*}\begin{split}
				(ra,b) = r(a,b) = (a,rb)
				\quad\text{for all }a\in A,\; b\in B,\; r\in R \\
			\end{split}\end{equation*}
		\end{enumerate} %}
		$R$-加群$A\otimes B$を$A$と$B$のテンソル積という。また、慣習で
		\ref{item:テンソル積その三}のアーベル群準同型とその像も同一の記号
		$\otimes$を用いる。
		\begin{equation*}\begin{split}
			\otimes: A\times B &\to A\otimes B \\
			(a,b) &\mapsto  a\otimes b = \bigl[i_\sei(a,b)\bigr]
		\end{split}\end{equation*}
		ここで、$i_\sei: A\times B\to\sei(A\times B)$を標準入射、$[(a,b)]$を
		$(a,b)\in A\times B$を代表元とする$A\times B/\sim$の同値類とする。
	\end{definition} %def:テンソル積}

	\begin{proposition}[テンソル積の普遍性]\label{prop:テンソル積の普遍性} %{
		$A,B,C$を$R$-加群とする。任意の$R$-双線形写像$f:A\times B\to C$に
		対して次の図を可換にする$R$-加群準同型$f_*:A\otimes B\to C$が一意に
		定まる。
		\begin{equation*}\xymatrix{
			A\times B \ar[r]^{\otimes} \ar[dr]_f & A\otimes B \ar@{.>}[d]^{f_*} \\
			& C
		}\end{equation*}
	\end{proposition} %prop:テンソル積の普遍性}
	\begin{proof} テンソル積の定義より、写像$f_*:A\otimes B\to C$を、
		任意の$\in A,\;b\in B$に対して$f_*(a\otimes b):=f(a,b)$と定義することが
		できる。すると、$f$が$R$-双線形写像だから、は$R$加群準同型となる。
		\begin{equation*}\begin{split}
			f(a_1+a_2,b) = f(a_1,b) + f(a_2,b) \\
			\implies f_*\bigl((a_1 + a_2)\otimes b\bigr)
			=  f_*(a_1\otimes b + a_2\otimes b)
			= f_*(a_1\otimes b) + f_*(a_2\otimes b) \\
			\quad\text{for all } a_1,a_2\in A,\; b\in B \\
			f(ra,b) = rf(a,b) 
			\implies f_*\bigl((ra)\otimes b\bigr) 
			= f_*\bigl(r(a\otimes b)\bigr) = rf_*(a\otimes b) \\
			\quad\text{for all } a\in A,\; b\in B,\; r\in R \\
			\text{same as for $B$}
		\end{split}\end{equation*}
		また、
		$\phi:A\otimes B\to C$を$R$-加群準同型とすると、次の式が成り立つ。
		\begin{equation*}\begin{split}
			\biggl(\phi(a\otimes b) = f(a,b)
				\quad\text{for all }a\in A,\; b\in B \biggr) \\
			\implies \biggl( f_*(a\otimes b) = f(a,b) = \phi(a\otimes b)
				\quad\text{for all }a\in A,\; b\in B \biggr) \implies f_* = \phi
		\end{split}\end{equation*}
	\end{proof}

	次の命題はテンソル積の普遍性によってテンソル積の基底系を保証する。

	\begin{proposition}[テンソル積の基底系]\label{prop:テンソル積の基底系} %{
		$A,B$を$R$-加群とする。$E_A$を$A$の生成系、$E_B$を$B$の生成系
		とすると、次の部分集合$E_{AB}\subseteq A\otimes B$は$A\otimes B$の
		生成系となる。
		\begin{equation*}\begin{split}
			E_{AB} := \set{a\otimes b\in A\otimes B \bou a\in E_A,\; b\in E_B}
		\end{split}\end{equation*}
		特に、$E_A$と$E_B$がそれぞれ$A$と$B$の基底系となるとき、
		$E_{AB}$は$A\otimes B$の基底系となる。
	\end{proposition} %prop:テンソル積の基底系}
	\begin{proof} 
	\begin{description}\setlength{\itemsep}{-1mm} %{
		\item[生成系] $A\otimes B\subseteq RE_{AB}$を示す。
		$E_A$と$E_B$がの生成系だから、任意の$f\in A$と$g\in B$に対して、
		\begin{equation*}\begin{split}
			f = \sum_{a\in E_A} (f_*a)a,\quad g = \sum_{b\in E_B} (g_*b)b
		\end{split}\end{equation*}
		となる$f_*\in RE_A^\tran$と$g_*:RE_B^\tran$が存在する。したがって、
		次の式から$f\otimes g\in RE_{AB}$となることがわかる。
		\begin{equation*}\begin{split}
			f\otimes g = \biggl(\sum_{a\in E_A} (f_*a)a\biggr)
				\otimes \biggl(\sum_{b\in E_B} (g_*b)b\biggr)
			= \sum_{a\in E_A,\;b\in E_B}(f_*a)(g_*b)(a\otimes b)
		\end{split}\end{equation*}
		%
		\item[基底系] $E_A$と$E_B$が基底系であるとき、
		$E_{AB}$が$R$-線形独立であることが示せれば命題が証明される。
		つまり、任意の$\phi\in\myop{Set}(A\times B,R)$に対して
		次の式が成り立つことが示せればよい。
		\begin{equation*}\begin{split}
			\sum_{a\in E_A,\;b\in E_B}\bigl(\phi(a,b)\bigr)(a\otimes b) = 0 
			\implies \phi = 0
		\end{split}\end{equation*}
		$E_A$と$E_B$が共に基底系だから、任意の$a\in E_A$と$b\in E_B$に対して
		$R$-双線形写像$\psi_{ab}:A\times B\to R$を次のように定義する\footnote{
			$E_A,\;E_B$が共に基底系であるから、$R$-双線形写像$\psi_{ab}$
			\eqref{eq:双線形写像の基底系}を定義できる。
			例えば、$E_A$が基底系でなければ、ある$r\neq0\in R,\;a\neq0\in E_A$で
			$ra=0$となることがあり得る。このとき、次のようになり、
			$\psi_{ab}$は$R$-双線形写像とはならない。
			\begin{equation*}\begin{split}
				\psi_{ab}(ra,b) = 0 \neq r = r\psi_{ab}(a,b)
			\end{split}\end{equation*}
		}。
		\begin{equation}\label{eq:双線形写像の基底系}\begin{split}
			\psi_{ab}(a_1,b_1) = \jump{a=a_1}\jump{b=b_1}
			\quad\text{for all } a_1\in E_A,\; b_1\in E_B
		\end{split}\end{equation}
		すると、テンソル積の普遍性\ref{prop:テンソル積の普遍性}により、
		次の式を満たす$R$-加群準同型$(a\otimes b)^\tran:A\otimes B\to R$
		が唯一つ定まり、
		\begin{equation*}\begin{split}
			(a\otimes b)^\tran(-\otimes-) = \psi_{ab}
			\quad\text{for all } a\in E_A,\; b\in E_B
		\end{split}\end{equation*}
		任意の$\phi\in\myop{Set}(A\times B, R)$と$a_0\in E_A,\; b_0\in E_B$
		に対して次の式が成り立つ。
		\begin{equation*}\begin{split}
			& \sum_{a\in E_A,\;b\in E_B}\bigl(\phi(a,b)\bigr)(a\otimes b) = 0 \\
			& \implies (a_0\otimes b_0)^\tran
			\sum_{a\in E_A,\;b\in E_B}\bigl(\phi(a,b)\bigr)(a\otimes b) = 0 \\
			& \implies \bigl(\phi(a,b)\bigr)
			\sum_{a\in E_A,\;b\in E_B}\phi_{a_0b_0}(a,b) = 0 \\
			& \implies \phi(a_0,b_0) = 0 \\
		\end{split}\end{equation*}
		したがって、命題が成り立つことがわかる。
	\end{description} %}
	\end{proof}

	この命題と命題\ref{prop:自由加群と基底系}を使うと、
	任意の集合$X,Y$に対して次の可換図が成り立つことがわかる。
	\begin{equation*}\xymatrix{
		& X\times Y \ar[d]^{-^\tran} \ar[dl] \ar[dr] \\
		X \ar[d]_{-^\tran} & R(X\times Y)^\tran \ar@{<->}[d]^{\simeq}
			& Y \ar[d]^{-^\tran} \\
		RX^\tran & RX^\tran\otimes RY^\tran & RY^\tran \\
		& RX^\tran\times RY^\tran \ar[u]_\otimes \ar[ul] \ar[ur] \\
	}\end{equation*}

	直和とテンソル積は分配則を満たす。

	\begin{proposition}[直和とテンソル積の分配則]
	\label{prop:直和とテンソル積の分配則} %{
		任意の$R$-加群$A,B,C$に対して次の分配則が成り立つ。
		\begin{equation*}\begin{split}
			A\otimes(B\oplus C)\simeq(A\otimes B)\oplus(A\otimes B)
		\end{split}\end{equation*}
	\end{proposition} %prop:直和とテンソル積の分配則}
	\begin{proof} 次の射$p_A,p_B,i_A,i_B$を、
		\begin{equation}\label{eq:分配則の双積}\begin{split}
			A\otimes C \udset{p_A}{i_A}{\fromto} (A\oplus B)\otimes C
			\udset{p_B}{i_B}{\tofrom} B\otimes C \\
		\end{split}\end{equation}
		任意の$a\in A,\;b\in B,\;c\in C$に対して次のように定義する。
		\begin{equation*}\begin{array}{rclrcl}
			p_A(a\oplus b\otimes c) &=& a\otimes c
			,& p_B(a\oplus b\otimes c) &=& b\otimes c \\
			i_A(a\otimes c) &=& (a\oplus 0)\otimes c
			,& i_B(b\otimes c) &=& (0\oplus b)\otimes c \\\
		\end{array}\end{equation*}
		すると、$p_Ai_A=\id_{A\otimes C}$と$p_Bi_B=\id_{B\otimes C}$が成り立ち、
		任意の$a\in A,b\in B,c\in C$に対して次の式が成り立つから、
		\begin{equation*}\begin{split}
			(i_Ap_A + i_Bp_B)\bigl((a\oplus b)\otimes c\bigr)
			&= i_A\bigl(a\otimes c\bigr) + i_B\bigl(b\otimes c\bigr) \\
			&= \bigl((a\oplus 0)\otimes c\bigr)
				+ \bigl((0\oplus b)\otimes c\bigr) \\
			&= (a\oplus b)\otimes c \\
		\end{split}\end{equation*}
		$i_Ap_A+i_Bp_B=\id_{(A\oplus B)\otimes C}$が成り立つ。
		したがって、式\eqref{eq:分配則の双積}が双積となり、命題が成り立つこと
		がわかる。
	\end{proof}

	加群の直和とテンソル積は分配則を満たすから代数を定義することができる。

	\begin{definition}[テンソル代数]\label{def:テンソル代数} %{
		$A$を$R$-加群とする。$n\in\sizen$に対して$\T_nA$を次のように定義し、
		\begin{equation*}\begin{array}{rcll}
			\T_0A &:=& R \\
			\T_1A &:=& A \\
			\T_nA &:=& \underbrace{A\otimes A\otimes\cdots\otimes A}_{n\text{ times}}
				& \quad\text{for all } 2\le n\in\sizen \\
		\end{array}\end{equation*}
		$\T_*A\subset\oplus_{n\in\sizen} \T_nA$を次のように有限和で定義する。
		\begin{equation*}\begin{split}
			\T_*A := \Set{\sum_{n\in\sizen}t_n\in\oplus_{n\in\sizen} \T_nA
			\;\left|\; \begin{array}{l}
				t_n\in \T_n A \text{ for all } n\in \sizen \text{ and} \\
				t_n\neq 0\text{ for only finitely many } n\in \sizen
			\end{array}\right.}
		\end{split}\end{equation*}
		$\T_*A$は直和$\oplus$によって$R$-加群となり、テンソル積$\otimes$に
		よって$R$-代数となる。
		\begin{equation*}\begin{array}{lrcl}
			\text{加法} & t_1 + t_2 &:=& t_1\oplus t_2 \\
			\text{乗法} & t_1t_2 &:=& t_1\otimes t_2 \\
		\end{array}
		\quad\text{for all } t_1,t_2\in T_*A
		\end{equation*}
		この$R$-代数$\T_*A=(\T_*A,+,0,\myspace,R)$を$A$のテンソル代数という
		\footnote{
			通常は$A$のテンソル代数を$\T A$と書くが、ここでは$\T_1A$との混同を
			避けるために$\T_*A$と書く事にする。
		}。
	\end{definition} %def:テンソル代数}

	自由加群のテンソル代数はベクトル空間のテンソル代数とほぼ同じものになる。

	\begin{proposition}[自由加群のテンソル代数]
	\label{prop:自由加群のテンソル代数} %{
		$F$を自由$R$-加群とする、
		\begin{itemize}\setlength{\itemsep}{-1mm} %{
			\item $\T_*F$をのテンソル代数、
			\item $E$を$F$の基底系、
			\item $\W_*E$を$E$から生成された自由モノイド、
			\item $R\W_*E$を$\W_*E$から生成された自由加群
		\end{itemize} %}
		とすると、$R$-代数同型$\T_*F\simeq R\W_*F$が成り立ち、
		$\W_*F$の元が$\T_*F$の基底系となる。
	\end{proposition} %prop:自由加群のテンソル代数}
	\begin{proof} と$R\W_*F$が$R$-代数同型になることが示されれば、
		$R\W_*F$が$\W_*A$から生成された自由加群だから、命題が成り立つことが
		示される。写像$\ket{-}:\W E\to\T_*F$を次のように定義する。
		\begin{equation*}\begin{split}
			\ket{1} &= 1 \\
			\ket{e_1\cdots e_p} &= e_1\otimes\cdots\otimes e_p
			\quad\text{for all } e_1,\dots,e_p\in E
		\end{split}\end{equation*}
		すると、$\ket{-}$はモノイド準同型となることがわかる。
		\begin{equation*}\begin{split}
			\ket{w_1w_2} = \ket{w_1}\otimes\ket{w_2}
			\quad\text{for all } w_1,w_2\in\W_*E
		\end{split}\end{equation*}
		したがって、$\ket{-}$を$R$-線形に拡張すると$R$-代数準同型
		$\ket{-}:R\W_*E\to\T_*F$となる。
		\begin{equation*}\begin{array}{rcll}
			\ket{f_1 + f_2} &=& \ket{f_1} + \ket{f_2}
			&\quad\text{for all } f_1,f_2\in R\W_*E \\
			\ket{rf} &=& r\ket{f} &\quad\text{for all } f\in R\W_*E,\; r\in R
		\end{array}\end{equation*}
	\end{proof}
%s2:加群のテンソル積}

\subsection{加群の外積}\label{s2:加群の外積} %{
	加群のテンソル代数の剰余代数として外積代数を定義する。

	\begin{definition}[加群の外積代数]\label{def:加群の外積代数} %{
		$A$を$R$-加群とする。 部分集合$\dup A\subseteq A\otimes A$を
		次のように定義する。
		\begin{equation*}\begin{split}
			\dup A:=\set{a\otimes a\in A\otimes A\bou a\in A}
		\end{split}\end{equation*}
		$\dup A$から生成されたイデアル$\dup_*A:=(\dup A)\T_*A\subseteq\T_*A$
		による剰余代数を$A$の外積代数といい、$\Wedge_*A$と書く。
		\begin{equation*}\begin{split}
			\Wedge_*A := \T_*A/\dup_*A
		\end{split}\end{equation*}
	\end{definition} %def:加群の外積代数}

	自由加群の外積代数はほとんどベクトル空間の外積代数と同じものになる。

	\begin{proposition}[自由加群の外積代数]\label{prop:自由加群の外積代数} %{
		$F$を自由$R$-加群、$E=\set{e_1,\dots,e_n}$を$F$の基底系とする。
		$\Wedge F$を$F$の外積代数とすると、次の代数同型が成り立つ。
		\begin{equation*}\begin{split}
			\Wedge F\simeq R[x_1,\dots,x_n]/I
		\end{split}\end{equation*}
		ここで、$I$は部分集合
		$\set{x_ix_j\bou i,j\in1..n}\subset R[x_1,\dots,x_n]$から生成された
		イデアルとする。
	\end{proposition} %prop:自由加群の外積代数}
	\begin{proof} 
	\end{proof}

	\begin{proposition}[自由加群の外積]\label{prop:自由加群の外積} %{
		自由$R$-加群の外積空間は次のようになる。
		\begin{equation*}\begin{split}
			\Lambda_pR^m &\simeq \left\{\begin{split}
				p\le m &\implies R^{\binom{m}{p}} \\
				\text{else} &\implies 0 \\
			\end{split}\right. \quad\text{for all } p,m\in\sizen
		\end{split}\end{equation*}
	\end{proposition} %prop:自由加群の外積}
	\begin{proof} 帰納法で証明する。$m=0,1$のときは次の式より、命題が成り立つ
	ことがわかる。
	\begin{equation*}\begin{split}
		\Lambda_pR \simeq \left\{\begin{split}
			p=0,1 &\implies R \\
			\text{else} &\implies 0 \\
		\end{split}\right. \\ %\}
	\end{split}\end{equation*}
	ある$2\le m\in\sizen$かで命題が成り立つと仮定する。すると、
	$\Lambda_0R^{m+1}=R$が成り立ち、任意の$p\in0..m$に対して次の式が成り立つ。
	\begin{equation*}\begin{split}
		\Lambda_{p+1}R^{m+1} \simeq (\Lambda_{p+1}\oplus\Lambda_p)R^m
		\simeq R^{\binom{m}{p+1}}\oplus R^{\binom{m}{p}}
		\simeq R^{\binom{m}{p+1}+\binom{m}{p}}
	\end{split}\end{equation*}
	ここで、$\binom{m}{p+1}+\binom{m}{p}=\binom{m+1}{p+1}$となるから、
	$\Lambda_{p+1}R^{m+1}\simeq R^{\binom{m+1}{p+1}}$が成り立つことがわかり、
	$m+1$でも命題が成り立つことがわかる。
	\end{proof}

	この命題から、論文\cite{Grayson1978Gro}の次の式が導かれる。
	\begin{equation*}\begin{split}
		\lambda_t[R^m] := \sum_{p\in\sizen}t^p[\Lambda_pR^m] = (1 + t)^m[R] \\
		\because\; \sum_{p\in\sizen}t^p[\Lambda_pR^m]
		= \sum_{p=0}^mt^p[\Lambda_pR^m]
		= \sum_{p=0}^mt^p[R^{\binom{m}{p}}]
		= \sum_{p=0}^mt^p\binom{m}{p}[R]
	\end{split}\end{equation*}
%s2:加群の外積}
%s1:加群}
	%
	} %\setlength\arraycolsep{2pt}
	%
\endgroup %}
